%!TEX root = index.tex
\chapter[Cronograma]{Cronograma}
\label{chap:cronograma}

O cronograma de atividades da dupla busca seguir o cronograma proposto pela banca avaliadora dos trabalhos de conclusão de curso, estando sempre à frente das entregas em pelo menos uma semana. Dessa maneira, é possível apresentar a entrega antecipadamente ao orientador e falar sobre possíveis mudanças ou correções.

Além disso, semanalmente os alunos se reunem com o orientador a fim de conversar sobre o andamento do projeto, apresentar-lhe o esboço dos relatórios e discutir a implementação dos algoritmos. 

Para o segundo semestre, trabalharemos na implementação do sistema de recomendação já no período de férias escolares, para poder ter uma amostra funcional no início das aulas. Em seguida, daremos início ao relatório final em paralelo com os testes de performance do sistema de recomendação, e esperamos finalizar o projeto dentro do prazo estipulado.

O cronograma detalhado da dupla está descrito a seguir:

\begin{description}
 	\item[09/04] Análise do banco de dados e determinação das medidas de similaridade
 	\item[16/04] Esboço do relatório final
 	\item[23/04] Validação I do relatório final
 	\item[07/05] Validação II do relatório final
 	\item[14/05] Proposta dos Algoritmos de Recomendação 
 	\item[28/05] Validação III do relatório final
 	\item[04/06] Esboço do resumo final e da apresentação
 	\item[09/06] Validação do resumo final
 	\item[11/06] Validação da apresentação
 	\item[13/06] Ensaio da apresentação
 	\item[23/06] Apresentação para o orientador
 	\item[]
 	\item[09/07] Pré-tratamento do banco de dados
 	\item[16/07] Programação do método \textit{FW} e variantes, descritos na Seção \ref{sec:algoritmo_baseado_na_pondera_o_de_atributos_}
 	\item[23/07] Programação do método \textit{UP} e variantes, descritos na Seção \ref{sec:algoritmo_baseado_no_perfil_de_usu_rios_}
 	\item[30/07] Análise comparativa dos dois algoritmos
 	\item[13/08] Relatório de atividades de implementação
 	\item[27/08] Primeiros testes com o sistema (precisão e acurácia para uma base de testes)
 	\item[03/09] Testes com o sistema (validação cruzada)
 	\item[24/09] Melhorias incrementais e relatório de atividades
 	\item[15/10] Relatório aprofundado de atividades
 	\item[05/11] Elaboração da apresentação e finalização dos relatórios
 	\item[12/11] Melhorias incrementais
 \end{description} 