%!TEX root = index.tex
\chapter[Metodologia]{Metodologia}
\label{chap:metodologia}

O presente Trabalho de Conclusão de Curso se fundamenta na metodologia de gestão de projetos adaptada de \cite{pmbok}. Por se tratar de um projeto de Engenharia de Software, alguns desses passos são modificados a fim de levar em conta o desenvolvimento do código computacional. Como o projeto de um software é um processo cíclico com etapas de especificação, desenvolvimento, validação e manutenção, a criação do produto ocorre de maneira incremental, diferentemente de certos projetos de outras áreas da engenharia \cite{iterative-development}. A metodologia de trabalho proposta pode ser, então, consolidada da seguinte maneira: 

\section{Definição da Necessidade} % (fold)
\label{sec:defini_o_da_necessidade}

% section defini_o_da_necessidade (end)

Com o crescente número de lojas de comércio online, tornou-se necessário a criação de sistemas que pudessem entender e prever o comportamento de consumidores, a fim de oferecer produtos específicos para cada um deles e aumentar o número de vendas e a satisfação do cliente. Observa-se atualmente que o número de sistemas de recomendação gratuitos, de fácil integração e de código aberto (\textit{open source}) são limitados e não correspondem às necessidades do mercado ou da academia. Existe, pois, a necessidade da criação de um sistema que possa ser utilizado por e-commerces que desejem estabelecer seu próprio sistema de recomendação ou mesmo por indivíduos interessados na temática da recomendação de itens.

\section{Definição dos Parâmetros de Sucesso} % (fold)
\label{sec:defini_o_dos_par_metros_de_sucesso}

% section defini_o_dos_par_metros_de_sucesso (end)

O sucesso do projeto poderá ser medido em duas frentes: a primeira, quantitativa, medirá a precisão e acurácia das recomendações; a segunda, qualitativa, avaliará se o sistema responde bem aos problemas recorrentes desse tópico de pesquisa, tais como a escalabilidade, o excesso de especialização e outros. 

\section{Síntese de Soluções} % (fold)
\label{sec:s_ntese_de_solu_es}

% section s_ntese_de_solu_es (end)

Nesta fase do projeto serão propostas possíveis soluções para o desafio da recomendação. Aqui o problema principal deverá ser dividido em partes menores, mutuamente exclusivas e coletivamente exaustivas (cobrem todos os pontos uma só vez), que serão resolvidas individualmente. Por exemplo, o algoritmo que calcula as medidas de similaridade entre dois itens influencia na taxa de sucesso da recomendação, e o método que expande este cálculo para todos os outros itens influencia na escalabilidade do sistema.

\section{Processo de escolha} % (fold)
\label{sec:processo_de_escolha}

% section processo_de_escolha (end)

O processo de escolha da solução deverá levar em conta dois pontos. O primeiro, eliminatório, é a viabilidade técnica da solução -- não será levado em conta soluções de execução inviável. O segundo, classificatório, leva em conta os parâmetros de sucesso do projeto, devendo assim avaliar qual alternativa maximiza a precisão e acurácia e ao mesmo tempo lida com os problemas qualitativos.

\section{Detalhamento da Solução} % (fold)
\label{sec:detalhamento_da_solu_o}

% section detalhamento_da_solu_o (end)

No detalhamento da solução, a estrutura dos algoritmos que gerarão as recomendações será descrita exaustivamente. Serão analisados fatores como complexidade computacional, linguagem de programação e entrada e saída de dados.

\section{Projeto Básico} % (fold)
\label{sec:projeto_b_sico}

% section projeto_b_sico (end)

Aqui serão codificados os métodos escolhidos. Esta etapa do projeto é incremental e ocorre em ciclos, e é acompanhada de testes unitários e testes de integração. 

\section{Modelamento e Simulação} % (fold)
\label{sec:modelamento_e_simula_o}

% section modelamento_e_simula_o (end)

Para o modelamento e simulação, utilizaremos um banco de dados de teste, que contenha apenas alguns milhares de dados, e posteriormente uma base maior. Serão feitas simulações e melhoramentos até que tenhamos resultados que antedam aos requisitos de projeto.

\section{Projeto Executivo} % (fold)
\label{sec:projeto_executivo}

% section projeto_executivo (end)

O Projeto Executivo conterá os métodos escolhidos e a forma de implementá-los em um e-commerce, a fim que seja possível aplicá-lo independentemente da área de atuação da empresa.

\section{Protótipos/Testes} % (fold)
\label{sec:prot_tipos_testes}

% section prot_tipos_testes (end)

A realização de testes será feita com os bancos de dados de centenas de milhares de itens ou de avaliações. Visto que será feita uma validação cruzada, será necessário descartar os dados e reformular a solução caso as recomendações não atinjam os requisitos funcionais. Isso evita que o sistema seja moldado para operar somente com aquele banco de dados específico.

\section{Produto} % (fold)
\label{sec:produto}

% section produto (end)

Assim que a fase de testes for concluída com êxito, o Projeto Executivo se torna o Produto, já que este irá conter o código computacional, o ``manual de operação'' e a forma de integração com uma loja online genérica.