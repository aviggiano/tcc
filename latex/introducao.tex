%!TEX root = index.tex
\chapter[Introdução]{Introdução}
\label{chap:introducao}

O comércio on-line se torna cada vez mais importante na vida das pessoas, a adoção deste método é cada vez mais comum. Estima-se que em 2013 1 bilhão de pessoas compraram online \cite{emarketerB2CEcommerceClimbs}, gerando uma receita anual de 1,25 trilhão de dólares com expectativas de crescer 17\% ao ano até 2017. \cite{emarketerGlobalB2CSales}. Para exemplificar, a gigante chinesa Alibaba está se preparando para abrir o seu capital na bolsa de valores americana. Esta oferta pública inicial poderá arrecadar até 25 bilhões de dólares, a tornando a maior oferta pública inicial do mercado acionário americano, e tornará a Alibaba o maior varejista online do mundo, com um valor avaliado em 158 bilhões de doláres \cite{ForbesAlibabaBoostsIPO}.

Ao analisarmos o mercado brasileiro, percebemos que é um mercado jovem e com bom potencial, porém o brasileiro não desenvolveu o hábito de fazer comprar pela internet, sendo que apenas 19\% dos compradores, usam esse serviço semanalmente, enquanto em países mais desenvolvidos estes números chegam à 35\% na Alemanha e 39\% no Reino Unido. Outro indicador de que o mercado brasileiro ainda é jovem é que 61\% dos consumidores de varejo online utilizaram o serviço pela primeira vez nos últimos 4 anos \cite{PWCTotalRetail}.

Como o mercado de varejo online é novo como um todo, este ainda passará por algumas mudanças drásticas em um curto espaço de tempo. Um dos itens-chave destas mudançaé a capacidade de analisar os dados gerados pelos consumidores. Com estes dados será possível segmentar os consumidores muito facilmente e as empresas poderão direcionar suas investidas de forma mais eficiente, chegando ao ponto em que campanhas de marketing, precificação serão customizadas. \cite{BCGThegotomarketrevolution} Uma das maneiras de se usar estes dados são os sistemas de recomendação.

``Sistemas de recomendação são ferramentas e técnicas de software destinadas a prover sugestões de itens para usuários'' \cite{ricci2011introduction-chap1}. O sistema tem o propósito de automatizar o processo de recomendação e auxiliar na tomada de decisão, podendo ser aplicado em diversas áreas da indústria, tais como na indicação de notícias, músicas, relações de amizade ou artigos científicos.

Estes sistemas são utilizados por diversos serviços online e geram um grande impacto quando utilizados corretamente. Em 2012, 75\% dos vídeos assistidos através do site NetFlix foram acessados por meio de recomendações \cite{netflix75}, em 2006, as recomendações representaram 35\% dos livros vendidos pela Amazon \cite{amazon35} e em 2007, 38\% das notícias lidas no Google News \cite{das2007google}.

De modo geral, um sistema de recomendação possui três etapas: a aquisição dos dados de entrada, a determinação das recomendações e finalmente a apresentação dos resultados ao usuário. A aquisição dos dados de entrada pode ser feita tanto de forma automática quanto manual, e em geral utiliza-se um banco de dados para armazenar essas informações. As sugestões são feitas segundo uma estratégia de recomendação determinada a priori, que pode ser fundamentada nas preferências do usuário, nas características dos itens ou em alguma formulação mista. Finalmente, os resultados são apresentados na interface sob variadas formas, como por exemplo em uma lista dos $N$ itens mais relevantes para o usuário.   

Conforme o tipo específico de itens recomendados, o design do sistema, a interface homem-máquina e o tipo de técnica de recomendação são construídos a fim de prover sugestões mais adequadas.

Os sistemas de recomendação são destinados primeiramente aos indivíduos que não possuem competência ou experiência suficiente para avaliar o grande número de opções do conjunto total de itens. Dessa forma, a interface homem-máquina é adaptada a cada um dos usuários, de maneira que eles recebam recomendações adequadas ao seu perfil. Essa ideia, amplamente divulgada por um antigo diretor executivo do e-commerce \textit{Amazon.com}, se resume à sua fala de que ``se você possui 2 milhões de clientes na web, você precisa ter 2 milhões de lojas na web'' \cite{schafer1999recommender}. 

Motivados pela importância econômica crescente de lojas de varejo online, bem como pela possibilidade de criar um conjunto de ferramentas \textit{open source} que possam ser utilizadas abertamente pela comunidade, propomos como Trabalho de Conclusão de Curso o desenvolvimento de um sistema de recomendação de produtos de e-commerces.  

A contribuição científica e tecnológica do trabalho para a Engenharia Mecatrônica estão sobretudo nos campos de sistemas de informação, de automação de processos e de inteligência artificial. As competências acadêmicas necessárias para a sua execução envolvem algoritmos e estruturas de dados, aprendizado de máquina e modelagem de bancos de dados. As competências técnicas abrangem programação estatística e orientada a objetos (R ou Java, por exemplo) e em linguagem de consulta estruturada (SQL).