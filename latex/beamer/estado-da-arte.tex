%!TEX root = index.tex
\section[Estado da Arte]{Estado da Arte}
%\subsection*{Estado da arte dos problemas} % (fold)

\begin{frame}{Estado da Arte}{Problema}
\begin{description}
	\item[$\mathcal{U}$] Conjunto dos usuários $u$ 
	\item[$\mathcal{I}$] Conjunto dos itens $i$ 
	\item[$r_{ui}$] Histórico avaliações \par{~}
	\item[$\ell$] Função de utilidade 
	\begin{itemize}
		\item $\ell: \mathcal{U} \times \mathcal{I} \rightarrow \mathcal{R}$ ~p.ex. $\left\{-1, 0, +1\right\}$ ou $ \left[1, 5\right]$
	\end{itemize}
\end{description}


\begin{block}{Objetivo}
Determinar o item $\tilde{\imath}_u$ que maximize a utilidade $\ell_{ui}$ do usuário $u$:
$$
\forall u \in \mathcal{U}, ~ \tilde{\imath}_u = \argmax_{i \in \mathcal{I}}{\ell_{ui}}
$$
\end{block}

\begin{alertblock}{Problema}
$\ell$ desconhecida
\end{alertblock}
\end{frame}


%\subsection*{Estado da arte das soluções}

\begin{frame}{Estado da Arte}{Soluções}
\begin{columns}[c]
\column{.4\textwidth} % Right column and width
\begin{block}{Estratégias de recomendação}
\begin{itemize}
	\item Colaborativas
	\item Conteúdo
	\item Híbridas
\end{itemize}
\end{block}



\column{.6\textwidth} % Right column and width


\textbf{Utilização comercial} \\ \cite{chiang2012networked}

\begin{description}
\item[Amazon] Filtragem baseada \\ em conteúdo
\item[YouTube] Contagem de visitas mútuas
\item[Pandora] Experts + votos positivos/negativos
\item[Netflix] Filtragem colaborativa
\end{description}
\end{columns}
\end{frame}


\begin{frame}{Estado da Arte}{Soluções}
\begin{columns}[c]
\column{.65\textwidth}
\textbf{Filtragem colaborativa (FC)}
\begin{itemize}
	\item Usuário-usuário
	\item Item-item
%	\item Fatorização de matrizes
\end{itemize}


\textbf{Filtragem baseada em conteúdo (BC)}
\par{~}

\textbf{Métodos híbridos}
\begin{itemize}
	\item FC + BC
\end{itemize}


\column{.35\textwidth}

\begin{table}[hp]
\begin{center}
	\caption{Avaliações $r_{ui}$}
    \begin{tabular}{ | c | c | c | c | c | c |} 
    \hline
     & $i_1$ & $i_2$ & $i_3$ & $i_4$ \\ \hline
     $u_1$ & - & 4 & 3 & - \\ \hline
     $u_2$ & - & 4 & 3 & 5 \\ \hline
     $u_3$ & 2 & 5 & - & 1 \\ \hline
    \end{tabular}
\end{center}
\end{table}


\begin{table}[hp]
\begin{center}
    \caption{Atributos $a_{if}$}
    \begin{tabular}{ | c | c | c | c | } 
    \hline
     & $f_1$ & $f_2$ & $f_3$ \\ \hline
     $i_1$ & 1 & 0 & 0 \\ \hline
     $i_2$ & 0 & 1 & 0 \\ \hline
     $i_3$ & 0 & 1 & 1 \\ \hline
    \end{tabular}
\end{center}
\end{table}

\end{columns}
\end{frame}

%\subsection*{Desafios científicos e tecnológicos}

\begin{frame}{Estado da Arte}{Desafios}
\begin{columns}[c]
\column{.4\textwidth}

\begin{block}{Desafios científicos e tecnológicos}
\begin{itemize}
	\item Escalabilidade
	\item Esparsidade
	\item \textit{Cold start}
	\item Excesso de especialização
\end{itemize}
\end{block}

\end{columns}
\end{frame}
