%!TEX root = index.tex
\section[Estado da Arte]{Estado da Arte}
%\subsection{Estado da arte dos problemas} % (fold)
%\label{sub:estado_da_arte_dos_problemas}
% subsection estado_da_arte_dos_problemas (end)
\begin{frame}
\frametitle{Estado da Arte}
\begin{itemize}
	\item $\mathcal{U}$ conjunto dos usuários $u$ 
	\item $\mathcal{I}$ conjunto dos itens $i$ 
	\item $r_{ui}$ histórico avaliações \par{~}
	\item $\ell$ função de utilidade 
	\begin{itemize}
		\item $\ell: \mathcal{U} \times \mathcal{I} \rightarrow \mathcal{R}$ $\left(\left\{-1, 0, +1\right\}, \left[1, 5\right], \cdots\right)$
	\end{itemize}
\end{itemize}


\begin{block}{Objetivo}
Determinar o item $\tilde{\imath}_u$ que maximize a utilidade $\ell_{ui}$ do usuário $u$:


\begin{equation} 
\label{eq:utilidade}
\forall u \in \mathcal{U}, ~ \tilde{\imath}_u = \argmax_{i \in \mathcal{I}}{\ell_{ui}}
\end{equation}
\end{block}

\begin{alertblock}{Problema}
$\ell$ desconhecida
\end{alertblock}
\end{frame}




\begin{frame}{Estado da Arte}
\begin{columns}[c]
\column{.5\textwidth} % Right column and width
\begin{block}{Estratégias de recomendação}
\begin{itemize}
	\item Baseadas em conteúdo
	\item Colaborativas
	\item Híbridas
\end{itemize}
\end{block}

\column{.5\textwidth} % Right column and width
\begin{block}{Desafios tecnológicos}
\begin{itemize}
	\item Escalabilidade
	\item Esparsidade
	\item \textit{Cold start}
	\item Excesso de especialização
\end{itemize}
\end{block}
\end{columns}
\end{frame}


\begin{frame}
\frametitle{Estado da Arte}
\textbf{colocar 1 slide para cada um dos métodos}
\end{frame}

\begin{frame}
\frametitle{Estado da Arte}

Utilização comercial \cite{chiang2012networked}

\begin{description}
\item[Amazon] Filtragem baseada em conteúdo
\item[YouTube] Contagem de visitas mútuas
\item[Pandora] Experts + votos positivos/negativos
\item[Netflix] Filtragem colaborativa
\end{description}
\end{frame}

%\subsection{Estado da arte das soluções} % (fold)
%\label{sub:estado_da_arte_das_solu_es_}
% subsection estado_da_arte_das_solu_es_ (end)
\begin{frame}
\frametitle{Estado da Arte}

\textbf{colocar gráficos da ref 11}
\end{frame}


%\subsection{Desafios científicos e tecnológicos} % (fold)
%\label{sub:desafios_cient_ficos_e_tecnol_gicos_}
% subsection desafios_cient_ficos_e_tecnol_gicos_ (end)
