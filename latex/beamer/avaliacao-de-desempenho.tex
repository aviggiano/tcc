%!TEX root = index.tex
\section[Avaliação de Desempenho]{Avaliação de Desempenho}
\begin{frame}
\frametitle{Avaliação de Desempenho}


\begin{table}[hp]
\begin{center}
    \caption{Avaliação de sistemas de predição}
    \label{tab:avaliacao-predicao}
    \begin{tabular}{  | p{2cm} | p{4cm} | p{4cm} | }
    \hline
    \textbf{Medida} & \textbf{Fórmula} & \textbf{Significado} \\ \hline
    Precisão &  $\frac{VP}{VP+FP}$ & Porcentagem de casos positivos corretamente preditos. \\ \hline                            
    Abrangência & $\frac{VP}{VP+FN}$ & Porcentagem de casos positivos sobre aqueles que foram marcados como positivos. \\ \hline
    Acurácia & $\frac{VP+VN}{VP+VN+FP+FN}$ & Porcentagem de predições corretas. \\ \hline
    $F_1$ &  $2 \cdot \frac{\mathrm{Precisão}~\cdot~\mathrm{Abrangência}}{\mathrm{Precisão}~+~\mathrm{Abrangência}}$ & Média harmônica entre precisão e abrangência. \\ \hline
    \end{tabular}
\end{center}
\end{table}

\end{frame}


\begin{frame}
\frametitle{Avaliação de Desempenho}

\begin{itemize}
	\item medida de distância entre  $\hat{\textbf{\i}}$ e  $\textbf{i}$
	\begin{itemize}
		\item distância $L_1$ (erro médio absoluto, $\left|\hat{\textbf{\i}} - \textbf{i}\right|$) 
		\item   distância $L_2$ (erro quadrático médio,  $\sqrt{\left|\hat{\textbf{\i}} - \textbf{i}\right|^2}$)
	\end{itemize}
	\item Desempenho do sistema mediante a mudança nas variáveis 
	\begin{itemize}
		\item  Quantidade de atributos utilizados na recomendação
	\end{itemize}
	\item Tempo de execução em função do algoritmo e do tamanho do banco de dados.
\end{itemize}
\end{frame}