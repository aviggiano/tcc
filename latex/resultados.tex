%!TEX root = index.tex
\chapter[Resultados]{Resultados}
\label{chap:resultados}

\section{Primeira etapa do Trabalho de Conclusão de Curso} % (fold)
\label{sec:primeira_etapa_do_trabalho_de_conclus_o_de_curso}

% section primeira_etapa_do_trabalho_de_conclus_o_de_curso (end)
Os resultados da primeira etapa deste Trabalho de Conclusão de Curso, realizadas na disciplina PMR2500, concentram-se na definição de necessidades e de parâmetros de sucesso e elaboração de possíveis soluções. 

Definimos que a aquisição de dados será feita a partir de uma base arbitrária, que deverá alimentar o sistema por meio de arquivos de texto com valores separados por vírgulas (\texttt{.csv}).

A fim de facilitar o pré-processamento dos dados, exigem-se dois arquivos. Um deles deve conter a matriz de atributos $\mathbf{A}$ e o outro, a matriz de avaliações  $\mathbf{R}$. 

\begin{equation} 
\mathbf{A} = 
\begin{bmatrix} 
 a_{i_1 f_1} &  a_{i_1 f_2} &  a_{i_1 f_3}  & \dots   \\
 a_{i_2 f_1} &  a_{i_2 f_2} &  a_{i_2 f_3}  & \dots   \\
 a_{i_3 f_1} &  a_{i_3 f_2} &  a_{i_3 f_3}  & \dots  \\ 
 \vdots &  \vdots &  \vdots  & \ddots   \\
 \end{bmatrix}
\end{equation}


\begin{equation}
	  \mathbf{R} = 
\begin{bmatrix} 
  r_{u_1 i_1} &  r_{u_1 i_2} &  r_{u_1 i_3}  & \dots   \\
 r_{u_2 i_1} &  r_{u_2 i_2} &  r_{u_2 i_3}  & \dots   \\
 r_{u_3 i_1} &  r_{u_3 i_2} &  r_{u_3 i_3}  & \dots  \\ 
 \vdots &  \vdots &  \vdots  & \ddots   \\
\end{bmatrix}
\end{equation}

Em alguns bancos de dados relacionais, a tabela de histórico também contém outras informações adicionais $\theta$, tais como método de pagamento, data da compra, data de entrega, etc., e é denominada $\mathbf{H}$. Optamos nessa etapa do projeto por não aceitar esse tipo de informação, e nem tampouco informações de  características de clientes (matriz $\mathbf{B}$). Para o emprego do sistema de recomendação em um e-commerce real, deverão ser efetuados ajustes no algoritmo a fim de tratar de particularidades envolvendo informações adicionais e arquivos suplementares.

\begin{equation} 
\mathbf{H} =
\begin{bmatrix} 
 r_{u_1 i_1} &  \theta_{h_1 1} &  \theta_{h_1 2} & \dots   \\
 r_{u_1 i_2} &  \theta_{h_2 1} &  \theta_{h_2 2} & \dots   \\
 r_{u i} &  \theta_{h 1} &  \theta_{h 2} & \dots   \\
 \vdots &  \vdots &  \vdots  & \ddots   \\
 \end{bmatrix} \\
\end{equation}

\begin{equation}
	\mathbf{B} = 
\begin{bmatrix} 
 b_{u_1 c_1} &  b_{u_1 c_2} &  b_{u_1 c_3}  & \dots   \\
 b_{u_2 c_1} &  b_{u_2 c_2} &  b_{u_2 c_3}  & \dots   \\
 b_{u_3 c_1} &  b_{u_3 c_2} &  b_{u_3 c_3}  & \dots  \\ 
 \vdots &  \vdots &  \vdots  & \ddots   \\
 \end{bmatrix}
\end{equation}

Uma vez determinada a forma de entrada de dados, definiu-se o conjunto de dados a serem utilizados. O primeiro conjunto de dados abertos é proveniente do website de recomendações de filmes MovieLens (\url{http://movielens.umn.edu}). Nessa base de dados, o catálogo de filme faz o papel de catálogo de produtos pelos quais os usuários possam se interessar, e o histórico de compras se refere à avaliação dos filmes feita por cada usuário. Outro conjunto de dados abertos que serão utilizados é do Internet Movie Database (IMDB), e poderá ser utilizado em complemento ou independência do primeiro banco.

Na primeira etapa do projeto, buscamos parcerias com e-commerces que estivessem dispostos a doar anonimamente seu banco de dados. Como as negociações ainda não foram estabelecidas, essa possibilidade está em aberto.

\section{Segunda etapa do Trabalho de Conclusão de Curso} % (fold)
\label{sec:segunda_etapa_do_trabalho_de_conclus_o_de_curso}

% section segunda_etapa_do_trabalho_de_conclus_o_de_curso (end)

A partir da síntese de soluções estabelecida na primeira etapa do projeto, implementamos os três algoritmos de recomendação e as medidas de recomendação na linguagem de programação estatística R.

Confirmamos a validade dos métodos aplicando-os nas matrizes de referência das Tabelas \ref{tab:rui_ref} e \ref{tab:aif_ref} e obtendo os mesmos resultados dos artigos-base.

Tratamos todos os dados das bases que serão utilizadas e iniciamos o processo de desenvolvimento da análise de \textit{cross-validation}.

Uma vez que a análise desempenho estiver finalizada para todos os métodos, será possível avaliar se os requisitos funcionais foram estabelecidos. Em seguida, procuraremos melhorar o sistema de recomendação a fim de torná-lo mais genérico. Buscaremos deixar a entrada e saída de informações arbitrária, de modo que o usuário possa informar ao sistema como é formado sua base e tabelas. Trabalharemos também na construção de um \textit{driver} que possibilite a conexão entre o sistema de recomendação e um banco de dados SQL, sem que seja necessária a etapa intermediária de arquivos \texttt{csv} para aquisição de dados. Caso haja tempo, planejamos elaborar um \textit{website} para o sistema de recomendação e exportar a lógica de recomendação para um servidor dedicado. Dessa forma, esse serviço de ``sistema de recomendação nas nuvens'' estaria completo e poderia ser utilizado por diferentes ramos de e-commerces.