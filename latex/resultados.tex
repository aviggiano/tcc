%!TEX root = index.tex
\chapter[Resultados]{Resultados}
\label{chap:resultados}

Até o presente momento, os resultados deste Trabalho de Conclusão de Curso concentram-se na definição de necessidades, de parâmetros de sucesso e de síntese de possíveis soluções. 

Visto que a primeira etapa de um sistema de recomendação é a coleta e manipulação de dados, definimos que a aquisição de dados será feita a partir de um banco de dados genérico, que deverá alimentar o sistema por meio de arquivos de texto com valores separados por vírgulas (\texttt{.csv}). A fim de facilitar o pré-processamento dos dados, exigem-se três arquivos, cada um com uma tabela de itens e seus atributos $\mathbf{A}$, clientes e suas características $\mathbf{B}$ e histórico de compras ou avaliações $\mathbf{R}$. Caso existam outras tabelas no banco de dados, o sistema deverá ser alterado para levar em conta o processamento dos arquivos suplementares.

\begin{equation} 
\mathbf{A} = 
\begin{bmatrix} 
 a_{i_1 f_1} &  a_{i_1 f_2} &  a_{i_1 f_3}  & \dots   \\
 a_{i_2 f_1} &  a_{i_2 f_2} &  a_{i_2 f_3}  & \dots   \\
 a_{i_3 f_1} &  a_{i_3 f_2} &  a_{i_3 f_3}  & \dots  \\ 
 \vdots &  \vdots &  \vdots  & \ddots   \\
 \end{bmatrix}
\end{equation}

\begin{equation}
	\mathbf{B} = 
\begin{bmatrix} 
 b_{u_1 c_1} &  b_{u_1 c_2} &  b_{u_1 c_3}  & \dots   \\
 b_{u_2 c_1} &  b_{u_2 c_2} &  b_{u_2 c_3}  & \dots   \\
 b_{u_3 c_1} &  b_{u_3 c_2} &  b_{u_3 c_3}  & \dots  \\ 
 \vdots &  \vdots &  \vdots  & \ddots   \\
 \end{bmatrix}
\end{equation}

\begin{equation}
	  \mathbf{R} = 
\begin{bmatrix} 
  r_{u_1 i_1} &  r_{u_1 i_2} &  r_{u_1 i_3}  & \dots   \\
 r_{u_2 i_1} &  r_{u_2 i_2} &  r_{u_2 i_3}  & \dots   \\
 r_{u_3 i_1} &  r_{u_3 i_2} &  r_{u_3 i_3}  & \dots  \\ 
 \vdots &  \vdots &  \vdots  & \ddots   \\
\end{bmatrix}
\end{equation}

Em alguns bancos de dados, a tabela de histórico também contém outras informações adicionais $\theta$, tais como o método de pagamento, a data da compra, data de entrega, etc., e é denominada $\mathbf{H}$.

\begin{equation} 
\mathbf{H} =
\begin{bmatrix} 
 r_{u_1 i_1} &  \theta_{h_1 1} &  \theta_{h_1 2} & \dots   \\
 r_{u_1 i_2} &  \theta_{h_2 1} &  \theta_{h_2 2} & \dots   \\
 r_{u i} &  \theta_{h 1} &  \theta_{h 2} & \dots   \\
 \vdots &  \vdots &  \vdots  & \ddots   \\
 \end{bmatrix} \\
\end{equation}

Uma vez determinada a forma de entrada de dados, definiu-se a escolha do conjunto de dados a serem utilizados. O primeiro conjunto de dados abertos é proveniente do website de recomendações de filmes MovieLens (\url{http://movielens.umn.edu}). Nessa base de dados, o catálogo de filme faz o papel de catálogo de produtos pelos quais os usuários possam se interessar, e o histórico de compras se refere à avaliação dos filmes feita por cada usuário. Outros conjuntos de dados também serão  explorados pela dupla, tais como os dados de classificação de músicas do serviço Yahoo! Music (\url{http://webscope.sandbox.yahoo.com}) ou de dados anônimos de e-commerces.
