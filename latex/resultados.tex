%!TEX root = index.tex
\chapter[Resultados]{Resultados}
\label{chap:resultados}

Até o presente momento, os resultados deste Trabalho de Conclusão de Curso concentram-se na definição de necessidades, de parâmetros de sucesso e de síntese de possíveis soluções. 

Visto que a primeira etapa de um sistema de recomendação é a coleta e manipulação de dados, definimos que a aquisição de dados será feita a partir de um banco de dados genérico, que deverá alimentar o sistema por meio de arquivos de texto com valores separados por vírgulas (\texttt{.csv}). A fim de facilitar o pré-processamento dos dados, exigem-se três arquivos, cada um com uma tabela de itens e seus atributos $\mathbf{A}$, clientes e suas características $\mathbf{B}$ e histórico de compras ou avaliações $\mathbf{R}$. Caso existam outras tabelas no banco de dados, o sistema deverá ser alterado para levar em conta o processamento dos arquivos suplementares.

\begin{equation} 
\begin{split} 
\mathbf{A} &= 
\begin{bmatrix} 
 a_{i_1 f_1} &  a_{i_1 f_2} &  a_{i_1 f_3}  & \dots   \\
 a_{i_2 f_1} &  a_{i_2 f_2} &  a_{i_2 f_3}  & \dots   \\
 a_{i_3 f_1} &  a_{i_3 f_2} &  a_{i_3 f_3}  & \dots  \\ 
 \vdots &  \vdots &  \vdots  & \ddots   \\
 \end{bmatrix} \\
\mathbf{B} &= 
\begin{bmatrix} 
 b_{u_1 c_1} &  b_{u_1 c_2} &  b_{u_1 c_3}  & \dots   \\
 b_{u_2 c_1} &  b_{u_2 c_2} &  b_{u_2 c_3}  & \dots   \\
 b_{u_3 c_1} &  b_{u_3 c_2} &  b_{u_3 c_3}  & \dots  \\ 
 \vdots &  \vdots &  \vdots  & \ddots   \\
 \end{bmatrix}\\
  \mathbf{R} &= 
\begin{bmatrix} 
  r_{u_1 i_1} &  r_{u_1 i_2} &  r_{u_1 i_3}  & \dots   \\
 r_{u_2 i_1} &  r_{u_2 i_2} &  r_{u_2 i_3}  & \dots   \\
 r_{u_3 i_1} &  r_{u_3 i_2} &  r_{u_3 i_3}  & \dots  \\ 
 \vdots &  \vdots &  \vdots  & \ddots   \\
\end{bmatrix}
\end{split} 
\end{equation}

Em alguns bancos de dados, a tabela de histórico também contém outras informações adicionais $\theta$, tais como o método de pagamento, a data da compra, data de entrega, etc., e é denominada $\mathbf{H}$.

\begin{equation} 
\mathbf{H} =
\begin{bmatrix} 
 r_{u_1 i_1} &  \theta_{h_1 1} &  \theta_{h_1 2} & \dots   \\
 r_{u_1 i_2} &  \theta_{h_2 1} &  \theta_{h_2 2} & \dots   \\
 r_{u_1 i_3} &  \theta_{h_3 1} &  \theta_{h_3 2} & \dots   \\
 \vdots &  \vdots &  \vdots  & \ddots   \\
 \end{bmatrix} \\
\end{equation}


Uma vez determinada a forma de entrada de dados, definiu-se a escolha do conjunto de dados a serem utilizados. O primeiro conjunto de dados abertos é proveniente do website de recomendações de filmes MovieLens (\url{http://movielens.umn.edu}). Nessa base de dados, o catálogo de filme faz o papel de catálogo de produtos pelos quais os usuários possam se interessar, e o histórico de compras se refere à avaliação dos filmes feita por cada usuário. Outros conjuntos de dados também serão  explorados pela dupla, tais como os dados de classificação de músicas do serviço Yahoo! Music (\url{http://webscope.sandbox.yahoo.com}) ou de dados anônimos de e-commerces.

Por fim, as possíveis soluções do projeto abrangem o cálculo das medidas de similaridade entre itens, para os conjuntos de dados que não foram previamente tratados. Determinar um valor numérico entre dois produtos distintos, tais como uma camiseta e uma prancha de \textit{surf}, é uma tarefa complexa e sujeita a erros humanos. Após reflexão e leitura de referências, definimos possíveis maneiras de realizar esse cálculo: 

\begin{itemize} 
	\item Grupos de similaridade: a similaridade dos itens seria definida por pelas características do item (como ``esporte radical'', ``corrida'', ``filme de aventura'', etc). Seria necessário classificar manualmente esses atributos (por exemplo, determinar que ``esporte radical'' tem similaridade de $3/5$ com ``corrida'' e $1/5$ com ``produtos de limpeza'').
	\item Histórico de compra da comunidade: quanto mais usuários comprarem a mesma dupla de itens, maior a similaridade entre estes itens. A classificação se faz pelo histórico mas o índice de similaridade pertence aos itens e não entre os usuários.
	\item Ranking por arestas: adaptado do sistema \textit{edge rank}, de classificação de posts no Facebook, este sistema levaria em conta a multiplicação de diversas características do item. Características como: popularidade da marca, número de compras por visualização do item, popularidade da marca para o usuário em questão (quantos itens ele comprou desta marca), há quanto tempo o item foi lançado e a relação do item com as compras anteriores do usuário (se este tipo de item já fez sucesso com este usuário).
\end{itemize}

Para os itens que já foram avaliados, como no caso dos conjuntos de dados de classificação de filmes ou músicas, essa etapa não é necessária, pois a similaridade entre dois itens provém diretamente da avaliação do usuário. Dizer que um filme $A$ tem avaliação $4/5$ e um filme $B$ tem avaliação $5/5$ equivale a dizer que os dois são interessantes para aquele usuários, e por isso vale recomendar filmes similares a $A$ e $B$. Nesse caso, passa-se  diretamente para a etapa das recomendações.

Os resultados práticos de cálculo de similaridade ou descrição completa do banco de dados serão apresentados no relatório final do trabalho, em conjunto com os demais resultados da evolução do projeto. 