%!TEX root = index.tex
\chapter[Objetivos]{Objetivos}
\label{chap:objetivos}

O objetivo do presente Trabalho de Conclusão de Curso é o desenvolvimento de um Sistema de Recomendação de produtos para lojas de comércio online, e respectiva análise de desempenho das recomendações propostas. 

Serão propostos diferentes algoritmos de recomendação, e será feita uma avaliação comparativa entre cada um deles. A explicação detalhada dos métodos se encontra no Capítulo \ref{chap:sintese_de_solucoes}.

O modelo a ser desenvolvido será, do ponto de vista da taxonomia tradicional dos sistemas de recomendação \cite{schafer1999recommender}, automático e persistente. Isso significa que o usuário não precisará informar suas preferências ao sistema e que seu histórico de compras será levado em conta nas sugestões. Essas características aproximam o entregável das ferramentas de marketing via e-mail das lojas online, que sugerem produtos com uma determinada frequência aos usuários com base em seu perfil e últimas compras. 

A qualidade das recomendações será avaliada quanto à acurácia e precisão. Será medida a distância entre os itens efetivamente comprados pelo cliente e aqueles previstos pelo sistema, além de indicadores de erro tipo I e erro tipo II. Uma descrição detalhada da avaliação do sistema de recomendação está descrita no Capítulo \ref{cha:avalia_o_de_desempenho}.

Por meio de uma validação cruzada, analisaremos a influência dos principais parâmetros do problema na qualidade das recomendações, como o tamanho do banco de dados ou a quantidade de informações de itens e clientes utilizadas na recomendação.

Será discutido o impacto dos principais desafios tecnológicos e científicos dos sistemas de recomendação na nossa proposta de solução, tais como a escalabilidade, a adaptação a novos usuários e a esparsidade dos dados \cite{sarwar2000analysis}.

Ao final, será possível extrair uma validação experimental das diretrizes fundamentais a serem seguidas por e-commerces que desejem desenvolver um sistema de recomendação próprio, a partir de um banco de dados de clientes, produtos e histórico de compras. 