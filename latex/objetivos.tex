%!TEX root = index.tex
\chapter[Objetivos]{Objetivos}
\label{chap:objetivos}

O objetivo do presente Trabalho de Conclusão de Curso é o desenvolvimento de um Sistema de Recomendação de produtos para lojas de comércio online, e respectiva análise de desempenho das recomendações propostas. 

Serão propostos diferentes algoritmos de recomendação, e será feita uma avaliação comparativa entre cada um dos métodos. A explicação detalhada de cada um deles se encontra no Capítulo \ref{chap:sintese_de_solucoes}.

O sistema a ser desenvolvido será, do ponto de vista da taxonomia tradicional dos sistemas de recomendação \cite{schafer1999recommender}, automático e persistente. Isso significa que as sugestões serão dadas sem a interação do usuário e que as compras anteriores serão levadas em conta. Essas características aproximam o entregável das ferramentas de marketing via e-mail, que sugerem produtos com uma determinada frequência aos usuários com base em seu histórico de compras. 

A qualidade das recomendações será avaliada tanto em termos da similaridade entre os itens efetivamente comprados pelo cliente com aqueles previstos pelo sistema de recomendação, quanto em termos de indicadores de erro tipo I e erro tipo II, como a medida F \cite{sarwar2000analysis}. 

Por meio de uma validação cruzada, analisaremos a influência dos principais parâmetros do problema na qualidade das recomendações, como o tamanho do banco de dados ou a quantidade de informações de itens e clientes utilizadas na recomendação. 

Será discutido o impacto dos principais desafios tecnológicos e científicos dos sistemas de recomendação na nossa proposta de solução, tais como a escalabilidade, a adaptação a novos usuários e a dispersão dos dados \cite{wei2007survey}. Também serão avaliadas as diferentes medidas de similaridade e modelos de predição na qualidade das recomendações. 

Ao final, será possível extrair uma validação experimental das diretrizes fundamentais a serem seguidas por e-commerces que desejem desenvolver um sistema de recomendação próprio, a partir de um banco de dados de clientes, produtos e histórico de compras. 