%!TEX root = index.tex
\chapter[Requisitos]{Requisitos}
\label{chap:requisitos}

A partir dos objetivos deste Trabalho de Conclusão de Curso, é possível extrair os requisitos funcionais do sistema de recomendação. Esses requisitos ditam principalmente sobre a escalabilidade e o desempenho das recomendações do sistema.

Como as sugestões serão calculadas com antecedência, não há necessidade para uma elevada taxa de recomendações por período de tempo (\textit{throughput}). Deseja-se contudo que o sistema possa gerar todas as recomendações para um banco de dados de cem mil clientes em uma hora, isto é, que tenha \textit{throughput} mínimo de 28 recomendação por segundo. Os sistemas de recomendação tradicionais possuem \textit{throughput} de cerca de 500 recomendações por segundo, mas operam em servidores dedicados de maior potência computacional \cite{sarwar2001item}. 

A fim de poder estabelecer uma base comparativa entre o sistema proposto \textit{UI} e os sistemas de referência \textit{FW} e \textit{UP}, serão utilizados os mesmos indicadores de desempenho dos artigos-base: precisão, abrangência e medida $F_1$ \cite{symeonidis2007feature,debnath2008feature}. Precisão é a porcentagem de casos corretamente preditos em relação ao tamanho da lista de recomendações. Abrangência é a razão entre o número de itens corretamente preditos e daqueles que foram efetivamente avaliados pelo usuário. A medida $F_1$, por sua vez, é a média harmônica entre precisão e
abrangência.

Todas essas métricas são dependentes dos diversos parâmetros do problema, como do tamanho da lista de recomendações $N$, da quantidade de vizinhos mais próximos $k$, e principalmente do banco de dados de teste. Como os artigos de referência não os disponibilizaram integralmente, serão estimados os valores de precisão, abrangência e medida $F_1$ para o banco de dados da dupla. 

Espera-se que a precisão, abrangência e consequentemente a medida $F_1$ sejam maiores que 20\%. Esses valores foram escolhidos por serem superiores aos de algoritmos puramente baseados em conteúdo ou em filtragem colaborativa \cite{symeonidis2007feature,debnath2008feature}. Na prática, o resultado mais importante é a comparação entre os três métodos para um banco de dados de referência. 

Neste trabalho o \textit{benchmark} é feito por meio de dois bancos amplamente utilizados na comunidade científica de Sistemas de Recomendação. O primeiro, denominado MovieLens 100k, é composto de 100 000 avaliações (valores inteiros de 1 a 5) de 943 usuários para 1682 filmes \cite{movielensdataset}. Além disso, cada usuário (idade, sexo, profissão, logradouro) avaliou pelo menos 20 filmes (categoria, ano de publicação). O segundo banco de dados é extraído do Internet Movie Database (IMDB), e possui 28 819 filmes. Esse banco está presente na biblioteca \texttt{ggplot2} da linguagem de programação R \cite{moviesggplot2dataset}.  

Os requisitos funcionais são suportados por requisitos não-funcionais, e estes são determinados pelas restrições sobre o projeto ou execução, tais como desenvolvimento e confiabilidade.

O sistema de recomendação deverá poder ser utilizado por qualquer e-commerce que disponha de um banco de dados de clientes, produtos e histórico de compras, desde que o formato de entrada, a ser especificado no Capítulo \ref{chap:resultados}, seja seguido.

Além disso o sistema deverá ser desenvolvido em tecnologias abertas (\textit{open source}) que tenham um alto número de colaboradores, como o sistema de gestão de banco de dados MySQL ou a linguagem de programação estatística R, a fim de torná-lo reutilizável por alunos ou e-commerces interessados.

Por fim, o sistema de recomendação deverá ser escalável e flexível no sentido de poder operar igualmente bem tanto em pequenas quanto em grandes bases de dados.

Apesar serem importantes parâmetros de um sistema de recomendação, a taxa de recomendações por período de tempo e a escalabilidade estão intimamente relacionados ao orçamento do projeto. Pode-se obter virtualmente qualquer \textit{throughput} desejado, contanto que haja investimento equivalente em infra-estrutura computacional. O mesmo não é válido para os parâmetros de qualidade da recomendação, que dependem tão somente dos algoritmos de sugestão. Neste trabalho, assumimos que o sistema de operará em microcomputadores pessoais, e por isso o requisito funcional \textit{throughput} se faz necessário.