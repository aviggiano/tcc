%!TEX root = index.tex
\chapter[Requisitos]{Requisitos}
\label{chap:requisitos}

A partir dos casos de uso propostos e dos objetivos deste Trabalho de Conclusão de Curso, é possível extrair os requisitos funcionais do sistema de recomendação. Esses requisitos ditam principalmente sobre a escalabilidade e acurácia do sistema.

Como as recomendações serão calculadas com antecedência e dadas de forma automática, não há necessidade para um elevado \textit{throughput} ou taxa de transferência (quantidade de recomendações feitas por período de tempo). Deseja-se contudo que o sistema possa gerar todas as recomendações para um banco de dados de cem mil clientes em uma hora, isto é, que tenha \textit{throughput} mínimo de $28$ recomendação por segundo. Os sistemas de recomendação tradicionais possuem \textit{throughput} de cerca de 500 recomendações por segundo, mas operam em servidores dedicados de maior potência computacional \cite{sarwar2001item}.  

O sistema também deve ser suficientemente acurado para prover recomendações úteis para os clientes. Espera-se que o desvio médio entre todas as previsões de qualidade de itens e os produtos efetivamente avaliados pelos clientes, ou seja, o erro absoluto médio, seja de no máximo $1.00$, para avaliações que variam de $1.00$ a $5.00$. No caso de bancos de dados que não contém a avaliação dos produtos por parte dos clientes, esse requisito pode ser substituído pelo desvio médio entre as similaridades dos produtos sugeridos e aqueles verdadeiramente comprados. Para os sistemas de recomendação tradicionais, esse valor é de cerca de $0.85$ \cite{sarwar2002recommender}.

Os requisitos funcionais são suportados por requisitos não-funcionais, e estes são determinados pelas restrições sobre o projeto ou execução, tais como requisitos de desempenho, de segurança ou confiabilidade. 

O sistema de recomendação deverá poder ser utilizado por qualquer e-commerce que disponha de um banco de dados de clientes, produtos e histórico de compras, desde que o formato de entrada, a ser especificado no Capítulo \ref{chap:resultados}, seja seguido.

Além disso o sistema deverá ser desenvolvido em tecnologias abertas (\textit{open source}) que tenham um alto número de colaboradores, como o sistema de gestão de banco de dados MySQL ou a linguagem de programação orientada a objetos Java, a fim de torná-lo genérico e reutilizável.

Por fim, o sistema de recomendação deverá ser escalável e flexível no sentido de poder operar igualmente bem tanto em pequenas quanto em grandes bases de dados.