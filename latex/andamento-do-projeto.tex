%!TEX root = index.tex
\chapter[Andamento do Projeto]{Andamento do Projeto}
\label{chap:andamento_do_projeto}

%Tal relatório deverá conter 
%	descrição minuciosa 
%		do andamento do projeto
%		das etapas cumpridas, 
%		do planejamento do trabalho no segundo semestre, 
%		dos resultados alcançados  , 
%		das modificações em relação à proposta inicial, 
%	bem como uma avaliação precisa e detalhada 
%		dos resultados alcançados neste semestre , 
%		dos objetivos do próximo semestre 
%		da viabilidade de alcance do resultado programado  no fim do segundo semestre. 

De início pensamos fazer um sistema de recomendação utilizando algoritmos de filtragem colaborativa baseada em itens, principalmente motivados pela leitura inicial de \cite{linden2003amazon}, que mostrava as vantagens desse método comparado à filtragem colaborativa baseada em usuários. 

Todavia, percebemos grande parte dos e-commerces estruturam seus bancos de dados em torno da descrição dos itens vendidos e das informações dos clientes. Pouco detalhe é dado à interação entre esses dois grupos, com exceção da tabela de compras, que se limita a informações como data e método de pagamento. Dessa forma concluímos que os métodos de filtragem colaborativa, fundamentados na avaliação dos itens por parte dos usuários, teriam pior desempenho que métodos baseados em conteúdo.

Métodos de recomendação baseados em conteúdo podem explorar a classificação dos produtos no banco de dados a fim de determinar as sugestões. Os atributos podem ser diversos, dependendo do ramo de negócios do e-commerce, tais como marca, esporte, categoria, sexo, idade, cor, preço, etc.

...

Utilizaremos um banco de dados de e-commerce fornecido anonimamente. Para o segundo semestre, utilizaremos também outros bancos, como o TODO MovieLens, TODO ClickBus. Ele está estruturado da seguinte maneira: TODO 

...


Para o segundo semestre deste ano, desenvolveremos um sistema de recomendação a partir de diferentes algoritmos e faremos uma análise de desempenho  para cada um deles. Utilizaremos algoritmos inspirados em \cite{debnath2008feature} e \cite{symeonidis2007feature}. O primeiro artigo determina a similaridade de dois itens a partir de medidas de distância para cada um dos atributos dos itens, ponderadas por pesos determinados na regressão linear de uma equação descrita pelo interesse dos usuários em cada \textit{feature}. O segundo texto parte do princípio que os usuários estão interessados nos atributos dos itens, traçando correlações entre esses dois elementos até chegar nos pesos que servirão de base para a matriz de similaridade de usuários, utilizada na recomendação pelo método da vizinhança (\textit{nearest neighbors}).

...

A aquisição de dados será feita a partir de um banco de dados genérico, que deverá alimentar o sistema por meio de arquivos de texto com valores separados por vírgulas (\texttt{.csv}). A fim de facilitar o pré-processamento dos dados, exigem-se três arquivos, cada um com uma tabela de itens, clientes e histórico de compras. Caso existam outras tabelas no banco de dados, o sistema deverá ser alterado para levar em conta o processamento dos arquivos suplementares.

...

Os resultados das recomendações serão entregues, da mesma forma, por meio de um arquivo \texttt{.csv} contendo o identificador de cada usuário com as \textit{top-N} recomendações de produtos, assim como o valor numérico associado à recomendação. Esse resultado é o mais importante do ponto de vista do e-commerce, que o utilizará como estratégia de marketing na sugestão de produtos.

...

Uma das maiores dificuldades do sistema de recomendação é a escala, isto é, o fato de os sistemas lidarem com quantidades de itens e clientes da ordem de centenas de milhares, exigindo algoritmos eficientes e inviabilizando implementações computacionalmente complexas. Outra grande dificuldade é a esparsidade dos dados, ou seja, o fato de a maioria dos clientes nunca ter interagido com mais de algumas unidades de itens, fazendo com que a matriz de relação usuário-item seja tenha uma quantidade muito pequena de valores preenchidos, da ordem de 1\% \cite{fennell2009collaborative}.

...

De modo geral os sistemas de recomendação tem o objetivo de apresentar ao usuário itens pelos quais ele possa se interessar e, no caso de um e-commerce, que ele vá adquirir. O desempenho de um sistema de recomendação se mede, portanto, na qualidade com a qual ele executa essa tarefa. Essa qualidade pode ser medida de diferentes maneiras, tal como pela medida de distância entre os produtos recomendados ($\widetilde{\mathbf{x}}$) e aqueles que seriam efetivamente comprados ($\mathbf{x}$) pelo cliente em uma validação cruzada (\textit{cross validation}). Essa medida pode ser, por exemplo, a distância $L_1$ (erro médio absoluto, $\left|\widetilde{\mathbf{x}} - \mathbf{x}\right|$) ou a distância $L_2$ (erro quadrático médio,  $\sqrt{\left|\widetilde{\mathbf{x}} - \mathbf{x}\right|^2}$)

Outras medidas de predição também serão utilizadas, como acurácia (\textit{accuracy}), especificidade (\textit{specificity}), precisão (\textit{precision}), abrangência (\textit{recall}) e a medida $F_1$ (\textit{F-score}). Elas estão sumarizadas na tabela \ref{tab:avaliacao-predicao}.

%\begin{table}[H]
%\begin{center}
%    \caption{Matriz de confusão}
%    \label{tab:avaliacao-predicao}
%    \begin{tabular}{ | l | l | p{5cm} | p{5cm} | }
%    \hline
%    & & \multicolumn{2}{|c|}{Caso predito} \\ \hline 
%    & & \textbf{Positivo} & \textbf{Negativo} \\ \hline
%    \multirow{2}{*}{Caso real} 
%        & Positivo & Verdadeiro Positivo &  $(VP)$ & Falso Negativo $(FN)$ \\ \hline
%        & Negativo & Falso Positivo $(FP)$ & Verdadeiro Negativo $(VN)$ \\ \hline
%    \end{tabular}
%\end{center}
%\end{table}


%\begin{tabular}{cc|c|c|c|c|l}
%\cline{3-4}
%& & \multicolumn{2}{ c| }{Caso predito} \\ \cline{3-4}
%& & 2 & 3  \\ \cline{1-4}
%\multicolumn{1}{ |c| }{\multirow{2}{*}{Powers} } &
%\multicolumn{1}{ |c| }{504} & 3 & 2 &      \\ \cline{2-4}
%\multicolumn{1}{ |c  }{}                        &
%\multicolumn{1}{ |c| }{540} & 2 & 3 &      \\ \cline{1-4}
%\multicolumn{1}{ |c  }{\multirow{2}{*}{Powers} } &
%\multicolumn{1}{ |c| }{gcd} & 2 & 2   \\ \cline{2-4}
%\multicolumn{1}{ |c  }{}                        &
%\multicolumn{1}{ |c| }{lcm} & 3 & 3   \\ \cline{1-4}
%\end{tabular}


\begin{table}[H]
\begin{center}
    \caption{Avaliação de sistemas de predição}
    \label{tab:avaliacao-predicao}
    \begin{tabular}{ | l | c | p{5cm} | }
    \hline
    \textbf{Medida} & \textbf{Fórmula} & \textbf{Significado} \\ \hline
    Precisão &  $\frac{VP}{VP+FP}$ & Porcentagem de casos positivos corretamente preditos. \\ \hline                            
    Abrangência & $\frac{VP}{VP+FN}$ & Porcentagem de casos positivos sobre aqueles que foram marcados como positivos. \\ \hline
    Especificidade & $\frac{VN}{VN+FP}$ &  Porcentagem de casos negativos sobre aqueles que foram marcados como negativos. \\ \hline
    Acurácia & $\frac{VP+VN}{VP+VN+FP+FN}$ & Porcentagem de predições corretas. \\ \hline
    Medida $F_1$ &  $2 \cdot \frac{\mathrm{Precisão}~\cdot~\mathrm{Abrangência}}{\mathrm{Precisão}~+~\mathrm{Abrangência}}$ & Média harmônica entre precisão e abrangência. \\ \hline
    \end{tabular}
\end{center}
\end{table}

Por fim, avaliaremos o desempenho do sistema mediante a mudança nas variáveis de importância do problema, como por exemplo na quantidade de atributos utilizados na recomendação e na capacidade de lidar com problemas como o \textit{cold start}. O tempo de execução também será avaliado em função do tamanho do banco de dados.

...

Problemas que o recsys enfrentará: cold start

...

Algoritmos

O primeiro algoritmo que utilizamos no sistema de recomendação, adaptado de  \cite{symeonidis2007feature} e doravante denominado de \textit{feature weighting}, ponderação de atributos ou \textit{FW}, se trata de um híbrido entre filtragem colaborativa e filtragem baseada em conteúdo. A partir da regressão linear de dados de uma rede social (\textit{Internet Movie Database, IMDB}), extraem-se os pesos que determinam a importância de cada atributo dos itens. Essa rede social permite determinar o julgamento humano de similaridade entre itens, e por isso se trata de uma filtragem colaborativa dos usuários. Após obtenção dos pesos, realiza-se a filtragem baseada em conteúdo e posteriormente os itens com maior similaridade são recomendados.

Na filtragem baseada em conteúdo, cada item é representado por um vetor de atributos ou \textit{features}. A similaridade $s_{ij}$ entre dois itens $i$ e $j$ é dada pela média ponderada das distâncias entre das \textit{features} dos itens:

\begin{equation} 
\label{eq:similaridade}
    s_{ij} = \sum_{f}{w_{f} ~ \left(1-d_{fij}\right)}
\end{equation}

As distâncias entre os atributos $d_f$ são determinadas conforme o tipo de dado avaliado e seu domínio, normalizadas no intervalo $\left[0,1\right]$. Para dados literais, como categoria, marca, cor, etc., uma possível medida de distância é o delta de Kronecker descrito em \ref{eq:delta}. É possível considerar a correlação entre atributos (marcas de calçado entre si tem maior similaridade que duas marcas de categorias distintas), mas em uma primeira análise utilizaremos para a maior parte dos atributos $f$ a medida de distância $d_{fij}=1-\delta_{ij}^f$. Isso significa que se as \textit{features} de dois itens são idênticas, a distância é nula e logo a similaridade é máxima. O sumário das medidas de distância estão na Tabela \ref{tab:medidas-distancia}.

\begin{equation}
\label{eq:delta}
\delta_{ij}^f = 
\begin{cases}
1, &\text{se }i=j \\
0, &\text{se }i \neq j
\end{cases} 
\end{equation}

\begin{table}[H]
\begin{center}
    \caption{Medidas de distância entre atributos}
    \label{tab:medidas-distancia}
    \begin{tabular}{ | l | c | p{5cm} | }
    \hline
    \textbf{Medida} & \textbf{Fórmula} & \textbf{Significado} \\ \hline
    \end{tabular}
\end{center}
\end{table}
 
Os pesos $w_f$ são a priori desconhecidos. A referência \cite{symeonidis2007feature} os determina a partir de um conjunto de equações do tipo \ref{eq:regressao-linear}, onde $e_{ij}$ é o número de usuários que se interessam tanto por $i$ quanto por $j$. 

\begin{equation}
\label{eq:regressao-linear} 
    e_{ij} = w_0 + \sum_{f}{w_{f} ~ d_{fij}}
\end{equation} 

A partir da matriz de avaliações $\mathbf{R}$, pode-se determinar $e_{ij}$, conforme a equação \ref{eq:determinacao-eij}, onde $r_{ui}$ é a avaliação do item $i$ feita pelo usuário $u$ e $\mathrm{b_0}$ é o operador booleano descrito por \ref{eq:b0}.

\begin{equation}
\label{eq:determinacao-eij} 
    e_{ij} = \sum_{u}{\mathrm{b_0}\left(r_{ui} ~ r_{uj}\right)}
\end{equation} 

\begin{equation}
\label{eq:b0}
\mathrm{b}_y\left(x\right) = 
\begin{cases}
1, &\text{se }x>y \\
0, &\text{se }x\leq y
\end{cases} 
\end{equation}

Desta forma, os pesos $w_f$ são determinados a partir resolução do sistema de equações lineares \ref{eq:determinacao-wf}. Apenas os pesos positivos e com maior valor absoluto são utilizados na recomendação. Calcula-se a matriz de similaridade $\mathbf{S}$ pela equação \ref{eq:similaridade} e recomenda-se os itens mais similares àqueles já comprados.  

\begin{equation}
\label{eq:determinacao-wf} 
    w_0 + \sum_{f}{w_{f} ~ d_{fij}} = \sum_{u}{\mathrm{b_0}\left(r_{ui} ~ r_{uj}\right)},~\forall i \neq j 
\end{equation} 

...

O segundo algoritmo, adaptado de \cite{debnath2008feature}, é um hibrido entre filtragem colaborativa e filtragem baseada em conteúdo. Os atributos dos itens são ponderados no cálculo de similaridade, com pesos extraídos de um modelo de perfil de usuários, denominado \textit{user profile} ou \textit{UP}. Esse perfil leva em consideração o interesse dos usuários por \textit{features}, indiretamente calculado a partir de seu interesse pelos itens. Se o usuário avaliou positivamente algum item $r_{ui}$, tal que $r_{ui}$ é superior a um valor mínimo $M$, considera-se que $u$ tem interesse $a_{if}$ nos atributos $f$ do item $i$. A matriz de correlação $t_{uf}$ entre usuários e \textit{features} é descrita por \ref{eq:puf}.

\begin{equation}
\label{eq:puf} 
    t_{uf} = \sum_{i}{\mathrm{b}_M\left(r_{ui}~a_{if}\right)} 
\end{equation} 

Os pesos $w_{uf}$ que mostram a relevância de $f$ para $u$ são determinados a partir da estatística TF-IDF (\textit{term frequency--inverse document frequency}), presente em formulações de recuperação de informação e mineração de dados. Em nosso caso, TF ou \textit{feature frequency} é a similaridade intra-usuários $p_{uf}$ -- número de vezes em que a \textit{feature} $f$ aparece no perfil do usuário $u$ (equação \ref{eq:tf}). IDF ou \textit{inverse user frequency} é a dissimilaridade inter-usuários $q_{f}$ -- relacionada com o inverso da frequência $\hat{q}_{f}$ de um atributo $f$ dentro de todos os usuários (equações \ref{eq:uf} e \ref{eq:iuf}).

\begin{equation}
\label{eq:tf} 
    p_{uf} = t_{uf}
\end{equation} 


\begin{equation}
\label{eq:uf} 
    \hat{q}_{f} = \sum_{u}{\mathrm{b}_0\left(t_{uf}\right)}
\end{equation} 

\begin{equation}
\label{eq:iuf} 
    q_{f} = \log \left( \frac{\left|~\mathcal{U}~\right|}{\hat{q}_{f}} \right)
\end{equation} 

Os pesos, descritos por \ref{eq:w-tfidf}, são utilizados para calcular a similaridade $s_{uv}$ entre dois usuários $u$ e $v$, conforme \ref{eq:suv}.

\begin{equation}
\label{eq:w-tfidf} 
    w_{uf} = p_{uf}~q_{f}
\end{equation} 


\begin{equation}
\label{eq:suv}
\begin{split}
    s_{uv} &= \frac{\sum\limits_{f \in \mathcal{F}_{uv}}{w_{uf}~w_{vf}}}{\sqrt{\sum\limits_{f \in \mathcal{F}_{uv}}w_{uf}^2} \sqrt{\sum\limits_{f \in \mathcal{F}_{uv}}w_{vf}^2}} \\
    \mathcal{F}_{uv} &= \mathcal{F}_u \cap \mathcal{F}_v \\
    \mathcal{F}_u &= \left\{ f \in \mathcal{F}~|~t_{uf} > 0 \right\}
\end{split}    
\end{equation} 

Dispondo-se de $\mathbf{S}$, selecionam-se os $k$ vizinhos mais próximos $v_k$ de $u$ com maior similaridade $s_{uv}$.  Posteriormente, determina-se o conjunto $I_{v_k} = \left\{ i ~|~ r_{v_ki} > 0\right\}$ de itens $i$ avaliados por $v_k$. Avalia-se a frequência total $\mathrm{fr}_f$ dos atributos $f$ para os itens de $I_{v_k}$, por \ref{eq:frf}. Por fim, a partir da equação \ref{eq:wi} calcula-se o peso $\omega_i$ de cada item e gera-se a lista dos \textit{top-N} itens a serem recomendados para o usuário $u$. 


\begin{equation}
\label{eq:frf} 
\mathrm{fr}_f = \sum_{i \in I_{v_k}}{\mathrm{b}_0\left(a_{if}\right)}
\end{equation} 

\begin{equation}
\label{eq:wi} 
    \omega_{i} = \sum_{f}{a_{if}~\mathrm{fr}_f}
\end{equation} 






%Objetivos
%desenvolvimento de um recsys
%Recomendação
%com base em features (ref 28)
%ditando os pesos w_i = 1 (facil e tosco)
%“aprender” os w_i de uma fonte externa (rede social)
%“aprender” os w_i com a base → como?
%outras??? como?
%aprender os pesos com base em grafo de pessoas que foram pra A e B ⇒ destino A e destino B sao similares
%como determinar os pesos de macro/micro categorias? ex: Vestuário { Camiseta {Basquete} , Sunga {Natacao} } 
%otimizacao dos pesos → “chuta” e vai mudando
%modelo do video do netflix!!! 
%
%
%Problemas elementares:
%Estruturar entrada / saida de dados
%Correlação de 2 features em R (delta, distancia euclidiana, etc)
%Similaridade entre dois itens quaisquer baseada em features (porque é mais facil)
%Similaridade para toda a matriz de itens
%Neighborhood e recomendar top N
%Nesse ponto ja temos um recsys primario, bem generico e que vale para qualquer e-commerce (passagens de onibus, livros, filmes, etc)
%Problemas mais dificeis
%Pensar no problema da data e sazonalidade (inspirar-se de netflix)
%Ref (25) diz que CF é melhor que CB e sugere melhorias. Estruturar e fazer um CF
%Qual algoritmo indicará que SP-RJ é similar a SP-BH??? Similar em quanto, de 0 a 10? Adaptar algum CF
%Sempre ter em mente o geral / especifico (ex: aplicado para a clickbus) do recsys, e escrever o relatório focando no “geral” e mostrando a aplicacao %“especifica”
