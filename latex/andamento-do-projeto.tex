%!TEX root = index.tex
\chapter[Andamento do Projeto]{Andamento do Projeto}
\label{chap:andamento_do_projeto}

%Tal relatório deverá conter 
%	descrição minuciosa 
%		do andamento do projeto
%		das etapas cumpridas, 
%		do planejamento do trabalho no segundo semestre, 
%		dos resultados alcançados  , 
%		das modificações em relação à proposta inicial, 
%	bem como uma avaliação precisa e detalhada 
%		dos resultados alcançados neste semestre , 
%		dos objetivos do próximo semestre 
%		da viabilidade de alcance do resultado programado  no fim do segundo semestre. 

De início pensamos fazer um sistema de recomendação baseado em filtragem colaborativa baseada em itens, principalmente motivados pela leitura inicial de \cite{linden2003amazon}, que mostrava as vantagens desse método comparado à filtragem colaborativa baseada em usuários. 

Todavia, percebemos grande parte dos e-commerces estruturam seus bancos de dados em torno da descrição dos itens vendidos e das informações dos clientes. Pouco detalhe é dado à interação entre esses dois grupos, com exceção da tabela de compras, que se limita a informações como data e método de pagamento. Dessa forma concluímos que os métodos de filtragem colaborativa, fundamentados na avaliação dos itens por parte dos usuários, teriam pior desempenho que métodos baseados em conteúdo.

Métodos de recomendação baseados em conteúdo podem explorar a classificação dos produtos no banco de dados a fim de determinar as sugestões. Os atributos podem ser diversos, dependendo do ramo de negócios do e-commerce, tais como marca, esporte, categoria, sexo, idade, cor, preço, etc.

...

Para o segundo semestre deste ano, desenvolveremos um sistema de recomendação a partir de diferentes algoritmos e faremos uma análise de desempenho  para cada um deles. Utilizaremos algoritmos inspirados em \cite{debnath2008feature} e \cite{symeonidis2007feature}. O primeiro artigo determina a similaridade de dois itens a partir de medidas de distância para cada um dos atributos dos itens, ponderadas por pesos determinados na regressão linear de uma equação descrita pelo interesse dos usuários em cada \textit{feature}, extraída de uma rede social. O segundo texto parte do princípio que os usuários estão interessados nos atributos dos itens, traçando correlações entre esses dois elementos até chegar nos pesos que servirão de base para a matriz de similaridade de usuários, utilizada na recomendação pelo método da vizinhança (\textit{nearest neighbors}).

...

A aquisição de dados será feita a partir de um banco de dados genérico, que deverá alimentar o sistema por meio de arquivos de texto com valores separados por vírgulas (\texttt{.csv}). A fim de facilitar o pré-processamento dos dados, exigem-se três arquivos, cada um com uma tabela de itens, clientes e histórico de compras. Caso existam outras tabelas no banco de dados, o sistema deverá ser alterado para levar em conta o processamento dos arquivos suplementares.

...

Os resultados das recomendações serão entregues, da mesma forma, por meio de um arquivo \texttt{.csv} contendo o identificador de cada usuário com as \textit{top-N} recomendações de produtos, assim como o valor numérico associado à recomendação. Esse resultado é o mais importante do ponto de vista do e-commerce, que o utilizará como estratégia de marketing na sugestão de produtos.

...

Uma das maiores dificuldades do sistema de recomendação é a escala, isto é, o fato de os sistemas lidarem com quantidades de itens e clientes da ordem de centenas de milhares, exigindo algoritmos eficientes e inviabilizando implementações computacionalmente complexas. Outra grande dificuldade é a esparsidade dos dados, ou seja, o fato de a maioria dos clientes nunca ter interagido com mais de algumas unidades de itens, fazendo com que a matriz de relação usuário-item seja tenha uma quantidade muito pequena de valores preenchidos, da ordem de 1 \% \cite{fennell2009collaborative}.

...

A análise de desempenho

...

Problemas que o recsys enfrentará: cold start


%Objetivos
%desenvolvimento de um recsys
%análise de desempenho
%erro tipo I e tipo II (medida F -- ver referencias)
%cross validation (MSE da recomendação e do efetivamente comprado)
%influencia do tamanho do BD no recsys (exemplo: 100 mil compras ou 100 mil itens)
%quantidade de attr utilizados no recsys
%comparar cold start dos 3 metodos de rec
%Recomendação
%com base em features (ref 28)
%ditando os pesos w_i = 1 (facil e tosco)
%“aprender” os w_i de uma fonte externa (rede social)
%“aprender” os w_i com a base → como?
%outras??? como?
%aprender os pesos com base em grafo de pessoas que foram pra A e B ⇒ destino A e destino B sao similares
%como determinar os pesos de macro/micro categorias? ex: Vestuário { Camiseta {Basquete} , Sunga {Natacao} } 
%otimizacao dos pesos → “chuta” e vai mudando
%modelo do video do netflix!!! 
%
%
%Problemas elementares:
%Estruturar entrada / saida de dados
%Correlação de 2 features em R (delta, distancia euclidiana, etc)
%Similaridade entre dois itens quaisquer baseada em features (porque é mais facil)
%Similaridade para toda a matriz de itens
%Neighborhood e recomendar top N
%Nesse ponto ja temos um recsys primario, bem generico e que vale para qualquer e-commerce (passagens de onibus, livros, filmes, etc)
%Problemas mais dificeis
%Pensar no problema da data e sazonalidade (inspirar-se de netflix)
%Ref (25) diz que CF é melhor que CB e sugere melhorias. Estruturar e fazer um CF
%Qual algoritmo indicará que SP-RJ é similar a SP-BH??? Similar em quanto, de 0 a 10? Adaptar algum CF
%Sempre ter em mente o geral / especifico (ex: aplicado para a clickbus) do recsys, e escrever o relatório focando no “geral” e mostrando a aplicacao %“especifica”
