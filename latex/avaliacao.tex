%!TEX root = index.tex
\chapter{Avaliação de Desempenho} % (fold)
\label{cha:avalia_o_de_desempenho}

% chapter avalia_o_de_desempenho (end)

% chapter avalia_o_do_sistema_de_recomenda_o (end)

% section avalia_o_do_sistema_de_recomenda_o (end)

De modo geral os sistemas de recomendação tem o objetivo de apresentar ao usuário itens pelos quais ele possa se interessar e que, no caso de um e-commerce,  ele vá adquirir. O desempenho de um sistema de recomendação se mede, portanto, na qualidade com a qual ele executa essa tarefa. Essa qualidade pode ser medida de diferentes maneiras, tal como pela medida de distância entre os produtos recomendados $\hat{\textbf{\i}}$ e aqueles que seriam efetivamente comprados $\textbf{i}$ pelo cliente em uma validação cruzada (\textit{cross validation}). Essa medida pode ser, por exemplo, a distância $L_1$ (erro médio absoluto, $\left|\hat{\textbf{\i}} - \textbf{i}\right|$) ou a distância $L_2$ (erro quadrático médio,  $\sqrt{\left|\hat{\textbf{\i}} - \textbf{i}\right|^2}$).

Outras medidas de predição também serão utilizadas, tais como acurácia (\textit{accuracy}), especificidade (\textit{specificity}), precisão (\textit{precision}), abrangência (\textit{recall}) e a medida $F_1$ (\textit{$F_1$-score}). Elas estão sumarizadas na Tabela \ref{tab:avaliacao-predicao}.

%\begin{table}[H]
%\begin{center}
%    \caption{Matriz de confusão}
%    \label{tab:avaliacao-predicao}
%    \begin{tabular}{ | l | l | p{5cm} | p{5cm} | }
%    \hline
%    & & \multicolumn{2}{|c|}{Caso predito} \\ \hline 
%    & & \textbf{Positivo} & \textbf{Negativo} \\ \hline
%    \multirow{2}{*}{Caso real} 
%        & Positivo & Verdadeiro Positivo &  $(VP)$ & Falso Negativo $(FN)$ \\ \hline
%        & Negativo & Falso Positivo $(FP)$ & Verdadeiro Negativo $(VN)$ \\ \hline
%    \end{tabular}
%\end{center}
%\end{table}


%\begin{tabular}{cc|c|c|c|c|l}
%\cline{3-4}
%& & \multicolumn{2}{ c| }{Caso predito} \\ \cline{3-4}
%& & 2 & 3  \\ \cline{1-4}
%\multicolumn{1}{ |c| }{\multirow{2}{*}{Powers} } &
%\multicolumn{1}{ |c| }{504} & 3 & 2 &      \\ \cline{2-4}
%\multicolumn{1}{ |c  }{}                        &
%\multicolumn{1}{ |c| }{540} & 2 & 3 &      \\ \cline{1-4}
%\multicolumn{1}{ |c  }{\multirow{2}{*}{Powers} } &
%\multicolumn{1}{ |c| }{gcd} & 2 & 2   \\ \cline{2-4}
%\multicolumn{1}{ |c  }{}                        &
%\multicolumn{1}{ |c| }{lcm} & 3 & 3   \\ \cline{1-4}
%\end{tabular}


\begin{table}[hp]
\begin{center}
    \caption{Avaliação de sistemas de predição}
    \label{tab:avaliacao-predicao}
    \begin{tabular}{  | >{\arraybackslash} m{3cm} | >{\centering\arraybackslash} m{4cm} | >{\arraybackslash} m{6cm} | }
    \hline
    \textbf{Medida} & \textbf{Fórmula} & \textbf{Significado} \\ \hline
    Precisão &  $\frac{VP}{VP+FP}$ & Porcentagem de casos positivos corretamente preditos. \\ \hline                            
    Abrangência & $\frac{VP}{VP+FN}$ & Porcentagem de casos positivos sobre aqueles que foram marcados como positivos. \\ \hline
    Especificidade & $\frac{VN}{VN+FP}$ &  Porcentagem de casos negativos sobre aqueles que foram marcados como negativos. \\ \hline
    Acurácia & $\frac{VP+VN}{VP+VN+FP+FN}$ & Porcentagem de predições corretas. \\ \hline
    Medida $F_1$ &  $2 \cdot \frac{\mathrm{Precisão}~\cdot~\mathrm{Abrangência}}{\mathrm{Precisão}~+~\mathrm{Abrangência}}$ & Média harmônica entre precisão e abrangência. \\ \hline
    \end{tabular}
\end{center}
\end{table}

Por fim, avaliaremos o desempenho do sistema mediante a mudança nas variáveis de importância do problema, como por exemplo na quantidade de atributos utilizados na recomendação. O tempo de execução também será avaliado em função do tamanho do banco de dados e do algoritmo utilizado.

%e na capacidade de lidar com problemas como o \textit{cold start}