%!TEX root = index.tex
\chapter[Síntese de Soluções]{Síntese de Soluções}
\label{chap:sintese_de_solucoes}

A simbologia utilizada neste texto é adaptada de \cite{symeonidis2007feature}, e está descrita na Tabela \ref{tab:simbologia}. As terminologias \textit{cliente} e \textit{usuário} serão intercambiáveis e sem distinção semântica, mesmo que na prática essas duas entidades possam ser diferentes. Da mesma forma, \textit{item} e \textit{produto} terão o mesmo significado neste trabalho. 

A fim de tornar a formulação mais genérica, também não faremos distinção entre \textit{avaliação positiva} de um item e \textit{compra} de um item. Avaliação positiva é toda avaliação $r_{ui}$ do item $i$ feital pelo usuário $u$ tal que $r_{ui} > M$, e avaliação negativa tal que $r_{ui} \leq M$, sendo $M$ um valor mínimo escolhido a priori, indicador de que o usuário $u$ ``gostou'' do item $i$. No caso de um banco de dados sem avaliações dos produtos, será levada em conta a compra dos itens e será admitida avaliação unitária e valor mínimo nulo. Desta forma, os bancos de dados que contenham informações do tipo ``usuário $u$ avaliou o item $i$ em $r_{ui} = 3.54 > M$'' e aqueles que contenham ``usuário $u$ comprou o item $i$, logo $r_{ui} = 1 > 0$'' serão tratados equivalentemente.

\begin{table}[hp]
\begin{center}
    \caption{Simbologia}
    \label{tab:simbologia}
    \begin{tabular}{ | l | l | }
    \hline
    \textbf{Símbolo} & \textbf{Definição} \\ \hline \hline
    $k$ & Número de vizinhos mais próximos \\ \hline
    $N$ & Tamanho da lista de recomendação \\ \hline \hline
    $\mathcal{U}$ & Conjunto de todos os usuários \\ \hline
    $\mathcal{F}$ & Conjunto  de todos os atributos \\ \hline
    $\mathcal{I}$ & Conjunto de todos os itens \\ \hline \hline
    $u, v$ & Usuários \\ \hline
    $i, j$ & Itens \\ \hline
    $f$ & Atributos dos itens \\ \hline
    $c $ & Características dos usuários \\ \hline \hline
    $\mathbf{X}_{M \times N},~\mathbf{X}$ & Matriz de elementos $x_{mn}$ \\ \hline
    $\mathbf{x}_{N},~\mathbf{x}$ & Vetor de elementos $x_{n}$ \\ \hline
    $\tilde{x}$ & Valor ótimo de $x$ \\ \hline    
    $\hat{x}$ & Valor estimado de $x$ \\ \hline    
    $|\mathcal{X}|$ & Número de elementos do conjunto $\mathcal{X}$ \\ \hline \hline
    $\mathbf{R}, r_{ui}$ & Avaliação feita pelo usuário $u$ do item $i$ \\ \hline
    $\mathbf{A}, a_{if}$ & Quantificação do atributo $f$ presente no item $i$ \\ \hline
    $\mathbf{B}, b_{uc}$ & Quantificação da característica $c$ do usuário $u$ \\ \hline    
    $\mathbf{T}, t_{uf}$ & Correlação entre usuário $u$ e atributo $f$ \\ \hline
    $\mathbf{w}, w_{f}$ & Peso do atributo $f$ \\ \hline
    $\mathbf{W}, w_{uf}$ & Correlação ponderada entre usuário $u$ e atributo $f$ \\ \hline
    $\mathbf{S}, s_{ij}, s_{uv}$ & Similaridade entre itens $i$ e $j$ ou entre usuários $u$ e $v$\\ \hline
    \end{tabular}
\end{center}
\end{table}


\section{Algoritmo baseado na ponderação de atributos (FW)} % (fold)
\label{sec:algoritmo_baseado_na_pondera_o_de_atributos_}

% section algoritmo_baseado_na_pondera_o_de_atributos_ (end)

O primeiro algoritmo que utilizaremos no sistema de recomendação, adaptado de  \cite{symeonidis2007feature} e doravante denominado ponderação de atributos, \textit{feature weighting} ou \textit{FW}, trata-se de um híbrido entre filtragem colaborativa e filtragem baseada em conteúdo. A partir da regressão linear de dados de uma rede social (\textit{Internet Movie Database, IMDB}), extraem-se os pesos que determinam a importância de cada atributo dos itens. Essa rede social permite determinar ``o julgamento humano de similaridade entre itens'', e é onde ocorre a filtragem colaborativa dos usuários. Após obtenção dos pesos, realiza-se a filtragem baseada em conteúdo para determinar os itens com maior similaridade, que são finalmente recomendados.

Na filtragem baseada em conteúdo, ``cada item é representado por um vetor de atributos ou \textit{features}''. A similaridade $s_{ij}$ entre dois itens $i$ e $j$ é dada pela média ponderada das distâncias entre as \textit{features} dos itens:

\begin{equation} 
\label{eq:sij}
    s_{ij} = \sum_{f}{w_{f} \left(1-d_{fij}\right)}
\end{equation}

As distâncias entre os atributos $d_f$ são determinadas conforme o tipo de dado avaliado e seu domínio, normalizadas no intervalo $\left[0,1\right]$. Para atributos literais, como categoria, marca, cor, etc., uma possível medida de distância é o delta de Kronecker descrito em \ref{eq:delta}. Nas medidas de distância, é interessante considerar a correlação entre atributos (a similaridade de duas marcas de calçado é maior que a de duas marcas de produtos de categorias distintas, mesmo que as marcas sejam diferentes), mas em uma primeira análise utilizaremos para a maior parte das \textit{features} a medida de distância \ref{eq:dfij}. Isso significa que se os atributos de dois itens são idênticos, a distância é nula e portanto a similaridade é máxima. O sumário de possíveis medidas de distância que serão utilizadas estão na Tabela \ref{tab:medidas-distancia}.

\begin{equation}
\label{eq:delta}
\delta_{mn} = 
\begin{cases}
1, &\text{se }m=n \\
0, &\text{se }m \neq n
\end{cases} 
\end{equation}

\begin{equation}
\label{eq:dfij}
\begin{split}
d_{fij} =& 1-\delta_{ij}^f \\
    =& 1-\delta_{a_{if} a_{jf}}
\end{split} 
\end{equation}

\begin{table}[hp]
\begin{center}
    \caption{Medidas de distância entre alguns atributos}
    \label{tab:medidas-distancia}
    \begin{tabular}{  | >{\arraybackslash} m{3cm} | >{\arraybackslash} m{3cm} | >{\centering\arraybackslash} m{3cm} | } 
    \hline
    \textbf{Atributo} $f$ & \textbf{Domínio} $\mathrm{F}$ & \textbf{Distância} $d_f$ \\ \hline
    Marca & Literal & $1-\delta^f_{ij}$ \\ \hline    
    Esporte & Literal & $1-\delta^f_{ij}$ \\ \hline
    Categoria & Literal & $1-\delta^f_{ij}$ \\ \hline            
    Gênero & Literal & $1-\delta^f_{ij}$ \\ \hline            
    Cor & $\left(\mathbb{N}\backslash \mathbb{N}_{255}\right)^3$  tripla ordenada RGB & $ \frac{\lVert f_i-f_j \rVert_2}{\max_{i,j}{\lVert f_i-f_j \rVert_2}} $ \\ \hline
    Preço & $\mathbb{R}$ & $ \frac{\left| f_i-f_j \right|}{\max_{i,j}{\left| f_i-f_j \right|}} $ \\ \hline                
    \end{tabular}
\end{center}
\end{table}
 
Os pesos $w_f$ são a priori desconhecidos. A referência \cite{symeonidis2007feature} os determina a partir de um conjunto de equações do tipo \ref{eq:regressao-linear}, onde $e_{ij}$ é o número de usuários que se interessam tanto por $i$ quanto por $j$. 

\begin{equation}
\label{eq:regressao-linear} 
    e_{ij} = w_0 + \sum_{f}{w_{f} \left(1-d_{fij}\right)}
\end{equation} 

A partir da matriz de avaliações $\mathbf{R}$, pode-se determinar $e_{ij}$, conforme a equação \ref{eq:determinacao-eij}, onde $\mathrm{b_0}$ é o operador booleano descrito por \ref{eq:b0}.

\begin{equation}
\label{eq:determinacao-eij} 
    e_{ij} = \sum_{u}{\mathrm{b_0}\left(r_{ui} ~ r_{uj}\right)}
\end{equation} 

\begin{equation}
\label{eq:b0}
\mathrm{b}_y\left(x\right) = 
\begin{cases}
1, &\text{se }x>y \\
0, &\text{se }x\leq y
\end{cases} 
\end{equation}

Desta forma, os pesos $w_f$ são determinados a partir resolução do sistema de equações lineares \ref{eq:determinacao-wf}. Apenas os pesos positivos e com valor absoluto expressivo (maior que um piso arbitrariamente escolhido a posteriori) são utilizados na recomendação. Calcula-se a matriz de similaridade $\mathbf{S}$ pela equação \ref{eq:sij} e recomenda-se os itens similares àqueles já comprados.  

\begin{equation}
\label{eq:determinacao-wf} 
    w_0 + \sum_{f}{w_{f}  \left(1-d_{fij}\right)} = \sum_{u}{\mathrm{b_0}\left(r_{ui} ~ r_{uj}\right)},~\forall i \neq j 
\end{equation} 

\subsection{Variante: pesos unitários (FW$_1$)} % (fold)
\label{sub:variante_pesos_unit_rios}

Além do algoritmo tradicional de ponderação de atributos, avaliaremos também a influência dos pesos $w_f$ na recomendação. Para tanto, as recomendações serão feitas considerando-se $w_f = 1~\forall f$ na variante denominada FW$_1$. 

Essa simplificação reduz grandemente a complexidade do algoritmo, pois a similaridade entre os itens passa a ser calculada por \ref{eq:sij1}, não sendo mais necessário  resolver o sistema linear \ref{eq:determinacao-wf}.

\begin{equation} 
\label{eq:sij1}
    s_{ij} = \sum_{f}{\left(1-d_{fij}\right)}
\end{equation}

Espera-se que o algoritmo \textit{FW$_1$} apresente menor tempo de execução que \textit{FW}, mas que a qualidade das recomendações seja muito inferior. O trabalho final discutirá se esta simplificação é interessante para os bancos de dados analisados, avaliando o compromisso entre custo computacional e qualidade de recomendação.

% subsection variante_pesos_unit_rios (end)

\section{Algoritmo baseado no perfil de usuários (UP)} % (fold)
\label{sec:algoritmo_baseado_no_perfil_de_usu_rios_}

% section algoritmo_baseado_no_perfil_de_usu_rios_ (end)

O segundo algoritmo, adaptado de \cite{debnath2008feature}, é um hibrido entre filtragem colaborativa e filtragem baseada em conteúdo. Os atributos dos itens são ponderados no cálculo de similaridade, com pesos extraídos de um modelo de perfil de usuários, denominado \textit{user profile} ou \textit{UP}. Esse perfil leva em consideração o interesse dos usuários por \textit{features}, indiretamente calculado a partir de seu interesse pelos itens. 

Se o usuário avaliou \textit{positivamente} algum item $r_{ui}$, tal que $r_{ui}$ é superior a um valor mínimo $M$, considera-se que $u$ tem interesse $t_{uf}$ nos atributos $f$ dos itens $i$, representados por $a_{if}$. A correlação $t_{uf}$ entre usuários e \textit{features} é descrita por \ref{eq:puf}.

\begin{equation}
\label{eq:puf} 
    t_{uf} = \sum_{i}{\mathrm{b}_M\left(r_{ui}~a_{if}\right)} 
\end{equation} 

Os pesos $w_{uf}$, que mostram a relevância de $f$ para $u$, são determinados a partir da estatística TF-IDF (\textit{term frequency--inverse document frequency}), presente em formulações de recuperação de informação e mineração de dados. Em nosso caso, TF ou \textit{feature frequency} é a ``similaridade intra-usuários'' $p_{uf}$ -- número de vezes em que a \textit{feature} $f$ aparece no perfil do usuário $u$ (equação \ref{eq:tf}). IDF ou \textit{inverse user frequency} é a ``dissimilaridade inter-usuários'' $q_{f}$ -- relacionada com o inverso da frequência $\check{q}_{f}$ de um atributo $f$ dentro de todos os usuários (equações \ref{eq:uf} e \ref{eq:iuf}).

\begin{equation}
\label{eq:tf} 
    p_{uf} = t_{uf}
\end{equation} 


\begin{equation}
\label{eq:uf} 
    \check{q}_{f} = \sum_{u}{\mathrm{b}_0\left(t_{uf}\right)}
\end{equation} 

\begin{equation}
\label{eq:iuf} 
    q_{f} = \log \left( \frac{\left|~\mathcal{U}~\right|}{\check{q}_{f}} \right)
\end{equation} 

Os pesos $w_{uf}$, obtidos na TF-IDF \ref{eq:w-tfidf}, são utilizados para calcular a similaridade $s_{uv}$ entre dois usuários $u$ e $v$, conforme \ref{eq:suv}.

\begin{equation}
\label{eq:w-tfidf} 
    w_{uf} = p_{uf}~q_{f}
\end{equation} 


\begin{equation}
\label{eq:suv}
\begin{split}
    s_{uv} &= \frac{\sum\limits_{f \in \mathcal{F}_{uv}}{w_{uf}~w_{vf}}}{\sqrt{\sum\limits_{f \in \mathcal{F}_{uv}
    }w_{uf}^2} \sqrt{\sum\limits_{f \in \mathcal{F}_{uv}}w_{vf}^2}} \\
    \mathcal{F}_{uv} &= \mathcal{F}_u \cap \mathcal{F}_v \\
    \mathcal{F}_u &= \left\{ f \in \mathcal{F}~|~t_{uf} > 0 \right\}
\end{split}    
\end{equation} 

Dispondo-se de $\mathbf{S}$, selecionam-se os $k$ vizinhos mais próximos $v_k^u$ com maior similaridade $s_{uv}$.  Posteriormente, determina-se o conjunto $\mathcal{I}_{v_k^u} = \left\{ i ~|~ r_{v_k^u i} > M\right\}$ de itens $i$ avaliados positivamente por $v_k^u$. Em \ref{eq:frf} avalia-se a frequência total $\mathrm{f}_{uf}$ dos atributos $f$ para os itens de $\mathcal{I}_{v_k^u}$. Por fim, a partir da equação \ref{eq:wi} calcula-se o peso $\omega_{ui}$ de cada item e gera-se a lista dos \textit{top-N} produtos a serem recomendados para o usuário $u$. 


\begin{equation}
\label{eq:frf} 
\mathrm{f}_{uf} = \sum_{i \in \mathcal{I}_{v_k^u}}{\mathrm{b}_0\left(a_{if}\right)}
\end{equation} 

\begin{equation}
\label{eq:wi} 
    \omega_{ui} = \sum_{f}{a_{if}~\mathrm{f}_{uf}}
\end{equation} 

\subsection{Variante: correlação usuário-item (UI)} % (fold)
\label{sub:variante_correla_o_usu_rio_item_}

A partir da matriz de correlações ponderadas $\mathbf{W}$ entre usuários e atributos, e da matriz de atributos dos itens $\mathbf{A}$, é possível extrair a correlação $\omega_{ui}$ entre usuários $u$ e itens $i$. A lista dos $N$ produtos a serem recomendados decorre portanto da equação \ref{eq:wui}.

\begin{equation}
\label{eq:wui} 
    \omega_{ui} = \sum_{f}{w_{uf}~a_{if}}
\end{equation} 

Ao passo que o método \textit{UP} recomenda itens a partir dos $k$ vizinhos mais próximos, o algoritmo \textit{UI} busca os itens com \textit{features} mais similares aos atributos pelos quais $u$ se interessa. Espera-se que esse tipo de recomendação forneça sugestões de qualidade similar ao algoritmo original, pois os dois estão fundamentados no fato que o usuário se interessa pelos atributos $f$ dos itens $i$. 


% subsection variante_correla_o_usu_rio_item_ (end)

\section{Avaliação do sistema de recomendação} % (fold)
\label{sec:avalia_o_do_sistema_de_recomenda_o}

% section avalia_o_do_sistema_de_recomenda_o (end)

De modo geral os sistemas de recomendação tem o objetivo de apresentar ao usuário itens pelos quais ele possa se interessar e que, no caso de um e-commerce,  ele vá adquirir. O desempenho de um sistema de recomendação se mede, portanto, na qualidade com a qual ele executa essa tarefa. Essa qualidade pode ser medida de diferentes maneiras, tal como pela medida de distância entre os produtos recomendados $\hat{\textbf{\i}}$ e aqueles que seriam efetivamente comprados $\textbf{i}$ pelo cliente em uma validação cruzada (\textit{cross validation}). Essa medida pode ser, por exemplo, a distância $L_1$ (erro médio absoluto, $\left|\hat{\textbf{\i}} - \textbf{i}\right|$) ou a distância $L_2$ (erro quadrático médio,  $\sqrt{\left|\hat{\textbf{\i}} - \textbf{i}\right|^2}$).

Outras medidas de predição também serão utilizadas, tais como acurácia (\textit{accuracy}), especificidade (\textit{specificity}), precisão (\textit{precision}), abrangência (\textit{recall}) e a medida $F_1$ (\textit{$F_1$-score}). Elas estão sumarizadas na Tabela \ref{tab:avaliacao-predicao}.

%\begin{table}[H]
%\begin{center}
%    \caption{Matriz de confusão}
%    \label{tab:avaliacao-predicao}
%    \begin{tabular}{ | l | l | p{5cm} | p{5cm} | }
%    \hline
%    & & \multicolumn{2}{|c|}{Caso predito} \\ \hline 
%    & & \textbf{Positivo} & \textbf{Negativo} \\ \hline
%    \multirow{2}{*}{Caso real} 
%        & Positivo & Verdadeiro Positivo &  $(VP)$ & Falso Negativo $(FN)$ \\ \hline
%        & Negativo & Falso Positivo $(FP)$ & Verdadeiro Negativo $(VN)$ \\ \hline
%    \end{tabular}
%\end{center}
%\end{table}


%\begin{tabular}{cc|c|c|c|c|l}
%\cline{3-4}
%& & \multicolumn{2}{ c| }{Caso predito} \\ \cline{3-4}
%& & 2 & 3  \\ \cline{1-4}
%\multicolumn{1}{ |c| }{\multirow{2}{*}{Powers} } &
%\multicolumn{1}{ |c| }{504} & 3 & 2 &      \\ \cline{2-4}
%\multicolumn{1}{ |c  }{}                        &
%\multicolumn{1}{ |c| }{540} & 2 & 3 &      \\ \cline{1-4}
%\multicolumn{1}{ |c  }{\multirow{2}{*}{Powers} } &
%\multicolumn{1}{ |c| }{gcd} & 2 & 2   \\ \cline{2-4}
%\multicolumn{1}{ |c  }{}                        &
%\multicolumn{1}{ |c| }{lcm} & 3 & 3   \\ \cline{1-4}
%\end{tabular}


\begin{table}[hp]
\begin{center}
    \caption{Avaliação de sistemas de predição}
    \label{tab:avaliacao-predicao}
    \begin{tabular}{  | >{\arraybackslash} m{3cm} | >{\centering\arraybackslash} m{4cm} | >{\arraybackslash} m{6cm} | }
    \hline
    \textbf{Medida} & \textbf{Fórmula} & \textbf{Significado} \\ \hline
    Precisão &  $\frac{VP}{VP+FP}$ & Porcentagem de casos positivos corretamente preditos. \\ \hline                            
    Abrangência & $\frac{VP}{VP+FN}$ & Porcentagem de casos positivos sobre aqueles que foram marcados como positivos. \\ \hline
    Especificidade & $\frac{VN}{VN+FP}$ &  Porcentagem de casos negativos sobre aqueles que foram marcados como negativos. \\ \hline
    Acurácia & $\frac{VP+VN}{VP+VN+FP+FN}$ & Porcentagem de predições corretas. \\ \hline
    Medida $F_1$ &  $2 \cdot \frac{\mathrm{Precisão}~\cdot~\mathrm{Abrangência}}{\mathrm{Precisão}~+~\mathrm{Abrangência}}$ & Média harmônica entre precisão e abrangência. \\ \hline
    \end{tabular}
\end{center}
\end{table}

Por fim, avaliaremos o desempenho do sistema mediante a mudança nas variáveis de importância do problema, como por exemplo na quantidade de atributos utilizados na recomendação. O tempo de execução também será avaliado em função do tamanho do banco de dados e do algoritmo utilizado.

%e na capacidade de lidar com problemas como o \textit{cold start}
