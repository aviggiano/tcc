%!TEX root = index.tex
\chapter[Síntese de Soluções]{Síntese de Soluções}
\label{chap:sintese_de_solucoes}

A simbologia utilizada neste texto é adaptada de \cite{symeonidis2007feature}, e está descrita na Tabela \ref{tab:simbologia}. As terminologias \textit{cliente} e \textit{usuário} serão intercambiáveis e sem distinção semântica, mesmo que na prática essas duas entidades possam ser diferentes. Da mesma forma, \textit{item} e \textit{produto} terão o mesmo significado neste trabalho. 

A fim de tornar a formulação mais genérica, também não faremos distinção entre \textit{avaliação positiva} de um item e \textit{compra} de um item. Avaliação positiva é toda avaliação $r_{ui}$ tal que $r_{ui} > M$, e avaliação negativa tal que $r_{ui} \leq M$, sendo $M$ um valor mínimo escolhido a priori, indicador de que o usuário $u$ ``gostou'' do item $i$. No caso de um banco de dados sem avaliações dos produtos, será levada em conta a compra dos itens e será admitida avaliação unitária. Desta forma, para $M=3$, os bancos de dados que contenham informações do tipo ``usuário $u$ avaliou o item $i$ em $r_{ui} = 3.54 > 3.00$'' e aqueles que contenham ``usuário $u$ comprou o item $i$, logo $r_{ui} = 1$'' serão tratados equivalentemente.

\begin{table}[hp]
\begin{center}
    \caption{Simbologia}
    \label{tab:simbologia}
    \begin{tabular}{ | l | l | }
    \hline
    \textbf{Símbolo} & \textbf{Definição} \\ \hline
    $k$ & Número de vizinhos mais próximos \\ \hline
    $N$ & Tamanho da lista de recomendação \\ \hline
    $\mathcal{U}$ & Conjunto de todos os usuários \\ \hline
    $\mathcal{F}$ & Conjunto  de todos os atributos \\ \hline
    $\mathcal{I}$ & Conjunto de todos os itens \\ \hline
    $u, v$ & Usuários \\ \hline
    $i, j$ & Itens \\ \hline
    $f$ & Atributos dos itens \\ \hline
    $c $ & Características dos usuários \\ \hline
    $\mathbf{X}_{M \times N},~\mathbf{X}$ & Matriz de elementos $x_{mn}$ \\ \hline
    $\mathbf{x}_{N},~\mathbf{x}$ & Vetor de elementos $x_{n}$ \\ \hline
    $|\mathcal{X}|$ & Número de elementos do conjunto $\mathcal{X}$ \\ \hline
    $\mathbf{R}, r_{ui}$ & Avaliação do item $i$ pelo usuário $u$\\ \hline
    $\mathbf{A}, a_{if}$ & Quantificação do atributo $f$ presente no item $i$ \\ \hline
    $\mathbf{B}, b_{uc}$ & Quantificação das características $c$ de cada usuário $u$ \\ \hline    
    $\mathbf{T}, t_{uf}$ & Correlação entre usuário $u$ e atributo $f$ \\ \hline
    $\mathbf{w}, w_{f}$ & Peso do atributo $f$ \\ \hline
    $\mathbf{W}, w_{uf}$ & Correlação ponderada entre usuário $u$ e atributo $f$ \\ \hline
    $\mathbf{S}, s_{ij}, s_{uv}$ & Similaridade entre itens $i$ e $j$ ou usuários $u$ e $v$\\ \hline
    \end{tabular}
\end{center}
\end{table}


\section{Avaliação do sistema de recomendação} % (fold)
\label{sec:avalia_o_do_sistema_de_recomenda_o}

% section avalia_o_do_sistema_de_recomenda_o (end)

De modo geral os sistemas de recomendação tem o objetivo de apresentar ao usuário itens pelos quais ele possa se interessar e, no caso de um e-commerce, que ele vá adquirir. O desempenho de um sistema de recomendação se mede, portanto, na qualidade com a qual ele executa essa tarefa. Essa qualidade pode ser medida de diferentes maneiras, tal como pela medida de distância entre os produtos recomendados $\widetilde{\mathbf{x}}$ e aqueles que seriam efetivamente comprados $\mathbf{x}$ pelo cliente em uma validação cruzada (\textit{cross validation}). Essa medida pode ser, por exemplo, a distância $L_1$ (erro médio absoluto, $\left|\widetilde{\mathbf{x}} - \mathbf{x}\right|$) ou a distância $L_2$ (erro quadrático médio,  $\sqrt{\left|\widetilde{\mathbf{x}} - \mathbf{x}\right|^2}$)

Outras medidas de predição também serão utilizadas, tais como acurácia (\textit{accuracy}), especificidade (\textit{specificity}), precisão (\textit{precision}), abrangência (\textit{recall}) e a medida $F_1$ (\textit{$F_1$-score}). Elas estão sumarizadas na Tabela \ref{tab:avaliacao-predicao}.

%\begin{table}[H]
%\begin{center}
%    \caption{Matriz de confusão}
%    \label{tab:avaliacao-predicao}
%    \begin{tabular}{ | l | l | p{5cm} | p{5cm} | }
%    \hline
%    & & \multicolumn{2}{|c|}{Caso predito} \\ \hline 
%    & & \textbf{Positivo} & \textbf{Negativo} \\ \hline
%    \multirow{2}{*}{Caso real} 
%        & Positivo & Verdadeiro Positivo &  $(VP)$ & Falso Negativo $(FN)$ \\ \hline
%        & Negativo & Falso Positivo $(FP)$ & Verdadeiro Negativo $(VN)$ \\ \hline
%    \end{tabular}
%\end{center}
%\end{table}


%\begin{tabular}{cc|c|c|c|c|l}
%\cline{3-4}
%& & \multicolumn{2}{ c| }{Caso predito} \\ \cline{3-4}
%& & 2 & 3  \\ \cline{1-4}
%\multicolumn{1}{ |c| }{\multirow{2}{*}{Powers} } &
%\multicolumn{1}{ |c| }{504} & 3 & 2 &      \\ \cline{2-4}
%\multicolumn{1}{ |c  }{}                        &
%\multicolumn{1}{ |c| }{540} & 2 & 3 &      \\ \cline{1-4}
%\multicolumn{1}{ |c  }{\multirow{2}{*}{Powers} } &
%\multicolumn{1}{ |c| }{gcd} & 2 & 2   \\ \cline{2-4}
%\multicolumn{1}{ |c  }{}                        &
%\multicolumn{1}{ |c| }{lcm} & 3 & 3   \\ \cline{1-4}
%\end{tabular}


\begin{table}[H]
\begin{center}
    \caption{Avaliação de sistemas de predição}
    \label{tab:avaliacao-predicao}
    \begin{tabular}{ | l | c | p{5cm} | }
    \hline
    \textbf{Medida} & \textbf{Fórmula} & \textbf{Significado} \\ \hline
    Precisão &  $\frac{VP}{VP+FP}$ & Porcentagem de casos positivos corretamente preditos. \\ \hline                            
    Abrangência & $\frac{VP}{VP+FN}$ & Porcentagem de casos positivos sobre aqueles que foram marcados como positivos. \\ \hline
    Especificidade & $\frac{VN}{VN+FP}$ &  Porcentagem de casos negativos sobre aqueles que foram marcados como negativos. \\ \hline
    Acurácia & $\frac{VP+VN}{VP+VN+FP+FN}$ & Porcentagem de predições corretas. \\ \hline
    Medida $F_1$ &  $2 \cdot \frac{\mathrm{Precisão}~\cdot~\mathrm{Abrangência}}{\mathrm{Precisão}~+~\mathrm{Abrangência}}$ & Média harmônica entre precisão e abrangência. \\ \hline
    \end{tabular}
\end{center}
\end{table}

Por fim, avaliaremos o desempenho do sistema mediante a mudança nas variáveis de importância do problema, como por exemplo na quantidade de atributos utilizados na recomendação e na capacidade de lidar com problemas como o \textit{cold start}. O tempo de execução também será avaliado em função do tamanho do banco de dados.