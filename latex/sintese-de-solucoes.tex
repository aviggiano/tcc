%!TEX root = index.tex
\chapter[Síntese de Soluções]{Síntese de Soluções}
\label{chap:sintese_de_solucoes}

A fim de facilitar a compreensão dos métodos propostos neste trabalho, serão utilizadas as matrizes de avaliações $\mathbf{R}$ e de atributos $\mathbf{A}$ abaixo, adaptadas da Referência \citeonline{debnath2008feature}. Em todos os exemplos, considera-se valor mínimo $M=2$. Os logaritmos são expressos em base 10.

\begin{table}[h]
\begin{center}
    \caption{Avaliações $r_{ui}$}
    \label{tab:rui_ref}
    \begin{tabular}{ | c | c | c | c | c | c | c | } 
    \hline
     & $i_1$ & $i_2$ & $i_3$ & $i_4$ & $i_5$ & $i_6$ \\ \hline
     $u_1$ & - & 4 & - & - & 5 & - \\ \hline
     $u_2$ & - & 3 & - & 4 & - & - \\ \hline
     $u_3$ & - & - & - & - & - & 4 \\ \hline
     $u_4$ & 5 & - & 3 & - & - & - \\ \hline
    \end{tabular}
\end{center}
\end{table}

\begin{table}[h]
\begin{center}
    \caption{Atributos $a_{if}$}
    \label{tab:aif_ref}
    \begin{tabular}{ | c | c | c | c | c | } 
    \hline
     & $f_1$ & $f_2$ & $f_3$ & $f_4$  \\ \hline
     $i_1$ & 0 & 1 & 0 & 0  \\ \hline
     $i_2$ & 1 & 1 & 0 & 0  \\ \hline
     $i_3$ & 0 & 1 & 1 & 0  \\ \hline
     $i_4$ & 0 & 1 & 0 & 0  \\ \hline
     $i_5$ & 1 & 1 & 1 & 0  \\ \hline
     $i_6$ & 0 & 0 & 0 & 1  \\ \hline
    \end{tabular}
\end{center}
\end{table}

\section{Algoritmo baseado na ponderação de atributos (FW)} % (fold)
\label{sec:algoritmo_baseado_na_pondera_o_de_atributos_}

% section algoritmo_baseado_na_pondera_o_de_atributos_ (end)

O primeiro algoritmo que utilizaremos no sistema de recomendação, adaptado da Referência \citeonline{symeonidis2007feature} e denominado ponderação de atributos, \textit{feature weighting} ou \textit{FW}, trata-se de um híbrido entre filtragem colaborativa e filtragem baseada em conteúdo. A partir da regressão linear de dados de uma rede social (\textit{Internet Movie Database, IMDB}), extraem-se os pesos que determinam a importância de cada atributo dos itens, e é onde ocorre a filtragem colaborativa dos usuários. Após obtenção dos pesos, realiza-se a filtragem baseada em conteúdo para determinar os itens com maior similaridade, que são finalmente recomendados.

Na filtragem baseada em conteúdo, ``cada item é representado por um vetor de atributos ou \textit{features}''. A similaridade $s_{ij}$ entre dois itens $i$ e $j$ é dada pela média ponderada das distâncias entre as \textit{features} dos itens:

\begin{equation} 
\label{eq:sij}
    s_{ij} = \sum_{f}{w_{f} \left(1-d_{fij}\right)}
\end{equation}

As distâncias entre os atributos $d_f$ são determinadas conforme o tipo de dado avaliado e seu domínio, normalizadas no intervalo $\left[0,1\right]$. 

Para atributos literais, como categoria, marca, cor, etc., uma possível medida de distância é o delta de Kronecker descrito em \ref{eq:delta}. A similaridade entre as cores ``azul'' e ``vermelho'' é, nesse caso, 0, e sua distância é 1. O valor da distância é nulo se e somente se os atributos são idênticos.

Para atributos pertencentes a uma coleção finita de itens, tais como os atores participantes de um filme, é possível estabelecer a similaridade entre dois conjuntos a partir do índice Jaccard, descrito em \ref{eq:jaccard}. Neste caso, a similaridade entre os conjuntos \{Al Pacino, Tom Hanks\} e \{Tom Hanks, Marlon Brando\} é $1/3$, e a sua distância é $2/3$.


\begin{equation}
\label{eq:delta}
\delta_{mn} =  
\begin{cases}
1, &\text{se }m=n \\
0, &\text{se }m \neq n
\end{cases} 
\end{equation}

\begin{equation}
\label{eq:jaccard}
J(A,B) ={{|A \cap B|}\over{|A \cup B|}}
\end{equation}

Vale considerar a correlação entre atributos no cálculo das distâncias: a similaridade de duas marcas de calçado, por exemplo, é maior que a de duas marcas de produtos de categorias diferentes, mesmo que as marcas sejam distintas nos dois casos. Em uma primeira análise, todavia, utilizaremos para a maior parte das \textit{features} as medidas de distância do delta de Kronecker \ref{eq:dfij_delta} e do índice Jaccard \ref{eq:dfij_jaccard}. Isso significa que se os atributos de dois itens são idênticos, a distância é nula e portanto a similaridade é máxima. O sumário de algumas medidas de distância que podem ser utilizadas estão na Tabela \ref{tab:medidas-distancia}.

\begin{equation}
\label{eq:dfij_delta}
\begin{split}
d_{fij} =&~ 1-\delta_{ij}^f \\
    =&~ 1-\delta_{a_{if} a_{jf}}
\end{split} 
\end{equation}

\begin{equation}
\label{eq:dfij_jaccard}
\begin{split}
d_{fij} =&~ 1-J^f(i,j) \\
    =&~ 1-J(a_{if},a_{jf})
\end{split} 
\end{equation}

\begin{table}[h]
\begin{center}
    \caption{Medidas de distância entre alguns atributos}
    \label{tab:medidas-distancia}
    \begin{tabular}{  | >{\arraybackslash} m{3cm} | >{\arraybackslash} m{3cm} | >{\centering\arraybackslash} m{3cm} | } 
    \hline
    \textbf{Atributo} $f$ & \textbf{Domínio} $\mathrm{F}$ & \textbf{Distância} $d_f$ \\ \hline
    Marca & Literal & $1-\delta^f_{ij}$ \\ \hline    
    Esporte & Literal & $1-\delta^f_{ij}$ \\ \hline
    Gênero & Literal & $1-\delta^f_{ij}$ \\ \hline            
    Categoria & Conjunto Literal & $1-J^f(i,j)$ \\ \hline            
    Preço & $\mathbb{R}$ & $ \frac{\left| a_{if}-a_{jf} \right|}{\max_{i,j}{\left| a_{if}-a_{jf} \right|}} $ \\ \hline
    Data & $\mathbb{R}$ milissegundos a partir de \textit{epoch} \cite{epoch} & $ \frac{\left| a_{if}-a_{jf} \right|}{\max_{i,j}{\left| a_{if}-a_{jf} \right|}} $ \\ \hline
    \end{tabular}
\end{center}
\end{table}
 
Os pesos $w_f$ são a priori desconhecidos. A Referência \citeonline{symeonidis2007feature} os determina a partir de uma regressão linear do tipo \ref{eq:regressao-linear}, onde $e_{ij}$ é o número de usuários que se interessam tanto por $i$ quanto por $j$. Esses valores permitem determinar ``o julgamento humano de similaridade entre itens'', e pode ser calculado a partir da matriz de avaliações, conforme a equação \ref{eq:determinacao-eij} (Tabela \ref{tab:eij}). O operador booleano $\mathrm{b}_M$, descrito pela Equação \ref{eq:b0}, nada mais é que uma ferramenta matemática para se poder extrair o número de usuários que avaliaram \textit{positivamente} tanto $i$ quanto $j$ a partir de $\mathbf{R}$. 


\begin{equation}
\label{eq:regressao-linear} 
    e_{ij} = w_0 + \sum_{f}{w_{f} \left(1-d_{fij}\right)}
\end{equation} 


\begin{equation}
\label{eq:determinacao-eij} 
    e_{ij} = \sum_{u}{\mathrm{b_M}\left(r_{ui} ~ r_{uj}\right)}
\end{equation} 

\begin{equation}
\label{eq:b0}
\mathrm{b}_M\left(x\right) = 
\begin{cases}
1, &\text{se }x>M \\
0, &\text{se }x\leq M
\end{cases} 
\end{equation}

\begin{table}[p]
\begin{center}
    \caption{$e_{ij}$}
    \label{tab:eij}
    \begin{tabular}{ | c | c | c | c | c | c | c | } 
    \hline
     & $i_1$ & $i_2$ & $i_3$ & $i_4$ & $i_5$ & $i_6$  \\ \hline
     $i_1$ & 1 & 0 & 1 & 0 & 0 & 0 \\ \hline
     $i_2$ & 0 & 2 & 0 & 1 & 1 & 0  \\ \hline
     $i_3$ & 1 & 0 & 1 & 0 & 0 & 0 \\ \hline
     $i_4$ & 0 & 1 & 0 & 1 & 0 & 0 \\ \hline
     $i_5$ & 0 & 1 & 0 & 0 & 1 & 0 \\ \hline
     $i_6$ & 0 & 0 & 0 & 0 & 0 & 1 \\ \hline
    \end{tabular}
\end{center}
\end{table}

Desta forma, os pesos $w_f$ são determinados a partir resolução do sistema de equações lineares \ref{eq:determinacao-wf}. Apenas os pesos positivos e com valor absoluto expressivo (maior que um piso arbitrariamente escolhido a posteriori) são utilizados na recomendação. 

\begin{equation}
\label{eq:determinacao-wf} 
    w_0 + \sum_{f}{w_{f}  \left(1-d_{fij}\right)} = \sum_{u}{\mathrm{b_0}\left(r_{ui} ~ r_{uj}\right)},~\forall i \neq j 
\end{equation} 

Calcula-se a matriz de similaridade $\mathbf{S}$ pela equação \ref{eq:sij} e recomendam-se os itens similares àqueles já comprados, segundo \ref{eq:ifw}.

\begin{equation}
\label{eq:ifw} 
    \hat{\imath}_u = \argmax_{i \in \left\{i~|~r_{ui} > 0\right\}, j}{s_{ij}}
\end{equation} 



\section{Algoritmo baseado no perfil de usuários (UP)} % (fold)
\label{sec:algoritmo_baseado_no_perfil_de_usu_rios_}

% section algoritmo_baseado_no_perfil_de_usu_rios_ (end)

O segundo algoritmo, adaptado da Referência \citeonline{debnath2008feature}, é um hibrido entre filtragem colaborativa e filtragem baseada em conteúdo. Os atributos dos itens são ponderados no cálculo de similaridade, com pesos extraídos de um modelo de perfil de usuários, denominado \textit{user profile} ou \textit{UP}. Esse perfil leva em consideração o interesse dos usuários por \textit{features}, indiretamente calculado a partir de seu interesse pelos itens. 

Para se determinar a relevância de $f$ para $u$, deve-se levar em conta não somente a frequência com a qual uma característica aparece, mas também o fato de algumas características estarem contidas na maioria dos itens. Determina-se, então, os pesos $w_{uf}$, que mostram a relevância de $f$ para $u$, a partir da medida estatística TF-IDF (\textit{term frequency--inverse document frequency}), presente em formulações de recuperação de informação e mineração de dados (Equação \ref{eq:w-tfidf}). 

Em nosso caso, TF ou \textit{feature frequency} é a ``similaridade intra-usuários'', igual ao número de vezes em que a \textit{feature} $f$ aparece no perfil do usuário $u$ (Equação \ref{eq:tf}, Tabela \ref{tab:tf_uf}). Se o usuário avaliou \textit{positivamente} algum item $r_{ui}$, tal que $r_{ui}$ é superior a um valor mínimo $M$, considera-se que $u$ tem interesse $\mathrm{TF}_{uf}$ nos atributos $f$ dos itens $i$, representados por $a_{if}$. 

\begin{equation}
\label{eq:tf} 
    \mathrm{TF}_{uf}  = \sum_{i}{\mathrm{b}_M\left(r_{ui}~a_{if}\right)} 
\end{equation} 

\begin{table}[p]
\begin{center}
    \caption{$\mathrm{TF}_{uf}$}
    \label{tab:tf_uf}
    \begin{tabular}{ | c | c | c | c | c | } 
    \hline
     & $f_1$ & $f_2$ & $f_3$ & $f_4$   \\ \hline
     $u_1$ & 2 & 2 & 1 & 0  \\ \hline
     $u_2$ & 1 & 2 & 0 & 0  \\ \hline
     $u_3$ & 0 & 0 & 0 & 1  \\ \hline
     $u_4$ & 0 & 2 & 1 & 0  \\ \hline
    \end{tabular}
\end{center}
\end{table}

O termo IDF ou \textit{inverse user frequency} é a ``dissimilaridade inter-usuários'', relacionada com o inverso da frequência de um atributo $f$ dentro de todos os usuários (Equação \ref{eq:iuf}, Tabela \ref{tab:idf_f}).

\begin{equation}
\label{eq:iuf} 
    \mathrm{IDF}_{f} = \log \left( \frac{\left|~\mathcal{U}~\right|}{\sum_{u}{\mathrm{b}_0\left(\mathrm{TF}_{uf}\right)}} \right)
\end{equation} 

\begin{table}[p]
\begin{center}
    \caption{$\mathrm{IDF}_{f}$}
    \label{tab:idf_f}
    \begin{tabular}{ | c | c | c | c | } 
    \hline
     $f_1$ & $f_2$ & $f_3$ & $f_4$   \\ \hline
     0.30 & 0.12 & 0.30 & 0.60  \\ \hline
     \end{tabular}
\end{center}
\end{table}

Os pesos $w_{uf}$, obtidos na TF-IDF \ref{eq:w-tfidf} (Tabela \ref{tab:w_uf}), são utilizados para calcular a similaridade $s_{uv}$ entre dois usuários $u$ e $v$, conforme as Equações \ref{eq:suv} e \ref{eq:fuv} (Tabela \ref{tab:s_uv}).

\begin{equation}
\label{eq:w-tfidf} 
    w_{uf} = \mathrm{TF}_{uf}~\mathrm{IDF}_{f}
\end{equation} 

\begin{table}[p]
\begin{center}
    \caption{$w_{uf}$}
    \label{tab:w_uf}
    \begin{tabular}{ | c | c | c | c | c | } 
    \hline
     & $f_1$ & $f_2$ & $f_3$ & $f_4$   \\ \hline
     $u_1$ & 0.60 & 0.25 & 0.30 & 0  \\ \hline
     $u_2$ & 0.30 & 0.25 & 0 & 0  \\ \hline
     $u_3$ & 0 & 0 & 0 & 0.60  \\ \hline
     $u_4$ & 0 & 0.25 & 0.30 & 0  \\ \hline
    \end{tabular}
\end{center}
\end{table}

\begin{equation}
\label{eq:suv}
    s_{uv} = \frac{\sum\limits_{f \in \mathcal{F}_{uv}}{w_{uf}~w_{vf}}}{\sqrt{\sum\limits_{f \in \mathcal{F}_{uv}
    }w_{uf}^2} \sqrt{\sum\limits_{f \in \mathcal{F}_{uv}}w_{vf}^2}} 
\end{equation} 

\begin{table}[p]
\begin{center}
    \caption{$s_{uv}$}
    \label{tab:s_uv}
    \begin{tabular}{ | c | c | c | c | c | } 
    \hline
     & $u_1$ & $u_2$ & $u_3$ & $u_4$   \\ \hline
     $u_1$ & 1 & 0.96 & 0 & 1  \\ \hline
     $u_2$ & 0.96 & 1 & 0 & 1  \\ \hline
     $u_3$ & 0 & 0 & 1 & 0  \\ \hline
     $u_4$ & 1 & 1 & 0 & 1  \\ \hline
    \end{tabular}
\end{center}
\end{table}

\begin{equation}
\label{eq:fuv}
\begin{split}
    \mathcal{F}_{uv} &= \mathcal{F}_u \cap \mathcal{F}_v \\
    \mathcal{F}_u &= \left\{ f \in \mathcal{F}~|~t_{uf} > 0 \right\}
\end{split}    
\end{equation} 

Dispondo-se de $\mathbf{S}$, selecionam-se os $k$ vizinhos mais próximos $v_k^u$ com maior similaridade $s_{uv}$, dentre todos $v \neq u$.  Posteriormente, determina-se o conjunto $\mathcal{I}_{v_k^u} = \left\{ i ~|~ r_{v_k^u i} > M\right\}$ de itens $i$ avaliados positivamente por $v_k^u$. Em \ref{eq:frf} avalia-se a frequência total $\mathrm{f}_{uf}$ dos atributos $f$ para os itens de $\mathcal{I}_{v_k^u}$ (Tabela \ref{tab:f_uf}). 

\begin{equation}
\label{eq:frf} 
\mathrm{f}_{uf} = \sum_{i \in \mathcal{I}_{v_k^u}}{\mathrm{b}_0\left(a_{if}\right)}
\end{equation} 

\begin{table}[p]
\begin{center}
    \caption{$\mathrm{f}_{uf}$}
    \label{tab:f_uf}
    \begin{tabular}{ | c | c | c | c | c | } 
    \hline
     & $f_1$ & $f_2$ & $f_3$ & $f_4$   \\ \hline
     $u_1$ & 0 & 2 & 1 & 0  \\ \hline
     $u_2$ & 1 & 3 & 2 & 0  \\ \hline
     $u_3$ & 1 & 2 & 0 & 0  \\ \hline
     $u_4$ & 2 & 3 & 1 & 0  \\ \hline
    \end{tabular}
\end{center}
\end{table}

Por fim, a partir da Equação \ref{eq:wi} calcula-se o peso $\omega_{ui}$ (Tabela \ref{tab:omega_ui}) de cada item e gera-se a lista dos \textit{top-N} produtos a serem recomendados para o usuário $u$, conforme \ref{eq:iup} (Tabela \ref{tab:i_u}). 

\begin{equation}
\label{eq:wi} 
    \omega_{ui} = \sum_{f}{a_{if}~\mathrm{f}_{uf}}
\end{equation} 


\begin{table}[p]
\begin{center}
    \caption{$\omega_{ui}$ (UP)}
    \label{tab:omega_ui}
    \begin{tabular}{ | c | c | c | c | c | c | c | } 
    \hline
     & $i_1$ & $i_2$ & $i_3$ & $i_4$ & $i_5$ & $i_6$ \\ \hline
     $u_1$ & 2 & 0 & 3 & 0 & 0 & 0 \\ \hline
     $u_2$ & 3 & 0 & 5 & 0 & 6 & 0 \\ \hline
     $u_3$ & 0 & 3 & 0 & 2 & 0 & 0 \\ \hline
     $u_4$ & 0 & 5 & 0 & 3 & 6 & 0 \\ \hline
    \end{tabular}
\end{center}
\end{table}

\begin{equation}
\label{eq:iup} 
    \hat{\imath}_u = \argmax_{i \in \left\{i~|~r_{ui}~=~0\right\}}{\omega_{ui}}
\end{equation} 

\begin{table}[p]
\begin{center}
    \caption{$\hat{\imath}_u$ (UP)}
    \label{tab:i_u}
    \begin{tabular}{ | c | c | c | c | } 
    \hline
     $u_1$ & $u_2$ & $u_3$ & $u_4$   \\ \hline
     3 & 5 & 2 & 5  \\ \hline
    \end{tabular}
\end{center}
\end{table}

\section{Algoritmo baseado na correlação usuário-item (UI)} % (fold)
\label{sec:algoritmo_baseado_na_correla_o_usu_rio_item_ui_}

Este método se trata de uma variante da solução \textit{UP}, e também está embasado no cálculo da preferência do usuário por \textit{features}, medida através do seu interesse pelos itens. O algoritmo \textit{UI} utiliza as matrizes de correlação ponderada entre usuários e atributos $\mathbf{W}$ e a matriz de atributos dos itens $\mathbf{A}$ no cálculo da correlação usuário-item.

A lista dos $N$ produtos a serem recomendados decorre portanto do cálculo de $\omega_{ui}$ (Equação \ref{eq:wui}, Tabela \ref{tab:omega_ui_ui}) e da escolha dos itens que maximizem essa variável para cada usuário (Equação \ref{eq:iup}, Tabela \ref{tab:i_u_ui}).

\begin{equation}
\label{eq:wui} 
    \omega_{ui} = \sum_{f}{w_{uf}~a_{if}}
\end{equation} 


\begin{table}[p]
\begin{center}
    \caption{$\omega_{ui}$ (UI)}
    \label{tab:omega_ui_ui}
    \begin{tabular}{ | c | c | c | c | c | c | c | } 
    \hline
     & $i_1$ & $i_2$ & $i_3$ & $i_4$ & $i_5$ & $i_6$ \\ \hline
     $u_1$ & 0.25 & 0.85 & 0.55 & 0.25 & 1.15 & 0 \\ \hline
     $u_2$ & 0.25 & 0.55 & 0.25 & 0.25 & 0.55 & 0 \\ \hline
     $u_3$ & 0    & 0    & 0    & 0    & 0    & 0.60 \\ \hline
     $u_4$ & 0.25 & 0.25 & 0.55 & 0.25 & 0.55 & 0 \\ \hline
    \end{tabular}
\end{center}
\end{table}


\begin{table}[p]
\begin{center}
    \caption{$\hat{\imath}_u$ (UI)}
    \label{tab:i_u_ui}
    \begin{tabular}{ | c | c | c | c | } 
    \hline
     $u_1$ & $u_2$ & $u_3$ & $u_4$   \\ \hline
     3 & 5 & - & 5  \\ \hline
    \end{tabular}
\end{center}
\end{table}

Ao passo que o método \textit{UP} recomenda itens a partir dos $k$ vizinhos mais próximos, o algoritmo \textit{UI} busca os itens com \textit{features} mais similares aos atributos pelos quais $u$ se interessa, diretamente através da matriz de atributos. 

Espera-se que esse tipo de recomendação forneça sugestões de qualidade similar ao algoritmo original, pois os dois tem a mesma fundamentação inicial. Pode-se observar que, para o exemplo-base, ambos algoritmos forneceram a mesma recomendação para três de quatro usuários.


% subsection variante_correla_o_usu_rio_item_ (end)
