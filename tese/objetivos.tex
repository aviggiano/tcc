%!TEX root = index.tex
\section[Objetivos]{Objetivos}
\label{chap:objetivos}

O objetivo do presente Trabalho de Conclusão de Curso é o desenvolvimento de um algoritmo e uma biblioteca computacional para sistemas de recomendação de produtos de lojas de comércio online, e respectiva análise de desempenho das recomendações propostas. 

O pacote de software é composto de três diferentes algoritmos de recomendação, além de funções para avaliar a qualidade das sugestões. Neste texto, será feita uma avaliação comparativa entre os três algoritmos. A explicação detalhada dos métodos se encontra no Capítulo \ref{chap:sintese_de_solucoes}.  

A fim de se poder experimentar a influência de diversos parâmetros na qualidade das recomendações, todas as funções foram desenvolvidas integralmente pela dupla, e nenhuma biblioteca externa foi utilizada. O objetivo do trabalho não é, portanto, o uso de ferramentas de recomendação já disponíveis no mercado, mas sim a elaboração de uma biblioteca que possibilite a construção e análise de um sistema de recomendação  próprio. Qualquer e-commerce interessado no assunto pode, portanto, apropriar-se do pacote de software e modificá-lo para atender a suas especificidades e melhorar as sugestões.

Essa ferramenta tem como finalidade a automatização do processo de recomendação de itens, e pode ser aplicada em diversas áreas da indústria, tais como na indicação de notícias, músicas, relações de amizade ou artigos científicos. No nosso trabalho, a biblioteca terá como foco a sugestão de produtos de lojas de comércio online que disponham de um histórico de compras dos usuários e das características dos produtos.

A qualidade das recomendações será avaliada quanto a precisão, abrangência e tempo de execução. Uma descrição detalhada da avaliação do sistema de recomendação está descrita na Seção \ref{cha:avalia_o_de_desempenho}.

Por meio de uma validação cruzada, analisaremos a influência dos principais parâmetros do problema na qualidade das recomendações, como o tamanho do banco de dados, a quantidade de informações de itens e clientes utilizadas na recomendação e outros.

Será discutido o impacto dos principais desafios tecnológicos e científicos dos sistemas de recomendação na nossa proposta de solução, tal como a adaptação a novos usuários \cite{sarwar2000analysis}.

Ao final, será possível extrair uma validação experimental das diretrizes fundamentais a serem seguidas por e-commerces que desejem desenvolver um sistema de recomendação próprio a partir da biblioteca desenvolvida neste trabalho. 