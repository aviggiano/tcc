%!TEX root = index.tex
\chapter[Objetivos]{Objetivos}
\label{chap:objetivos}

motivação

- razoes p/ desenvolver biblioteca e nao gui... mais rapida / mais geral

- vender o produto

- na literatura existe XYZ, o nosso é W


O objetivo do presente Trabalho de Conclusão de Curso é o desenvolvimento de um Sistema de Recomendação de produtos para e-commerces, e respectiva análise de desempenho das recomendações propostas. 

Serão propostos três diferentes algoritmos de recomendação, e será feita uma avaliação comparativa entre eles. A explicação detalhada dos métodos se encontra no Capítulo \ref{chap:sintese_de_solucoes}.

Essa ferramenta tem como finalidade a automatização do processo de sugestão de itens, e pode ser aplicada em diversas áreas da indústria, tais como na indicação de notícias, músicas, relações de amizade ou artigos científicos. No nosso trabalho, o sistema terá como foco a sugestão de produtos de lojas de comércio online que disponham de um histórico de compras dos usuários e das características dos produtos.

A qualidade das recomendações será avaliada quanto a precisão e abrangência. Para os métodos em que se possui uma medida de similaridade entre produtos, será avaliada a distância entre os itens efetivamente comprados pelo cliente e aqueles previstos pelo sistema. Uma descrição detalhada da avaliação do sistema de recomendação está descrita no Capítulo \ref{cha:avalia_o_de_desempenho}.

Por meio de uma validação cruzada, analisaremos a influência dos principais parâmetros do problema na qualidade das recomendações, como o tamanho do banco de dados ou a quantidade de informações de itens e clientes utilizadas na recomendação.

Será discutido o impacto dos principais desafios tecnológicos e científicos dos sistemas de recomendação na nossa proposta de solução, tais como a escalabilidade, a adaptação a novos usuários e a esparsidade dos dados \cite{sarwar2000analysis}.

Ao final, será possível extrair uma validação experimental das diretrizes fundamentais a serem seguidas por e-commerces que desejem desenvolver um sistema de recomendação próprio ou que queiram utilizar o sistema desenvolvido neste trabalho. 