
%----------------------------------------------------------------------------------------
%	PACKAGES AND THEMES
%----------------------------------------------------------------------------------------

\documentclass{beamer}
\usepackage[brazil]{babel}
\mode<presentation> {

% The Beamer class comes with a number of default slide themes
% which change the colors and layouts of slides. Below this is a list
% of all the themes, uncomment each in turn to see what they look like.

%\usetheme{default}
%\usetheme{AnnArbor}
%\usetheme{Antibes}
%\usetheme{Bergen}
%\usetheme{Berkeley}
%%\usetheme{Berlin}
%%\usetheme{Boadilla}
%%%\usetheme{CambridgeUS}
%\usetheme{Copenhagen}
%\usetheme{Darmstadt}
%\usetheme{Dresden}
\usetheme{Frankfurt}
%\usetheme{Goettingen}
%\usetheme{Hannover}
%%\usetheme{Ilmenau}
%\usetheme{JuanLesPins}
%\usetheme{Luebeck}
%\usetheme{Madrid}
%\usetheme{Malmoe}
%\usetheme{Marburg}
%\usetheme{Montpellier}
%\usetheme{PaloAlto}
%%\usetheme{Pittsburgh}
%\usetheme{Rochester}
%\usetheme{Singapore}
%\usetheme{Szeged}
%\usetheme{Warsaw}

% As well as themes, the Beamer class has a number of color themes
% for any slide theme. Uncomment each of these in turn to see how it
% changes the colors of your current slide theme.

%\usecolortheme{albatross}
%\usecolortheme{beaver}
%\usecolortheme{beetle}
%\usecolortheme{crane}
%\usecolortheme{dolphin}
%\usecolortheme{dove}
%\usecolortheme{fly}
%\usecolortheme{lily}
%\usecolortheme{orchid}
%\usecolortheme{rose}
%\usecolortheme{seagull}
%\usecolortheme{seahorse}
%\usecolortheme{whale}
%\usecolortheme{wolverine}

%\setbeamertemplate{footline} % To remove the footer line in all slides uncomment this line
\setbeamertemplate{footline}[page number] % To replace the footer line in all slides with a simple slide count uncomment this line



% pra que serve?
%\setbeamercovered{transparent}

\setbeamertemplate{navigation symbols}{} % To remove the navigation symbols from the bottom of all slides uncomment this line
}

\usepackage{graphicx} % Allows including images
\usepackage{booktabs} % Allows the use of \toprule, \midrule and \bottomrule in tables


% ---
% Pacotes básicos 
% ---
\usepackage{lmodern}			% Usa a fonte Latin Modern			
\usepackage[T1]{fontenc}		% Selecao de codigos de fonte.
\usepackage[utf8]{inputenc}		% Codificacao do documento (conversão automática dos acentos)
\usepackage{color}				% Controle das cores
\usepackage{microtype} 			% para melhorias de justificação
% ---
		
% ---
% Pacotes de citações
% ---
% Paginas com as citações na bibl
\usepackage[alf]{abntex2cite}	% Citações padrão ABNT

% --- 
% CONFIGURAÇÕES DE PACOTES
% --- 
\usepackage{amsmath}
\usepackage{multirow}
% helvetica no titulo dos capitulos
%\usepackage[scaled]{helvet}
\usepackage{tgheros}
%real numbers
\usepackage{amssymb}
%\hyphenation{Recomendação}

\DeclareMathOperator*{\argmax}{arg\,max}
% alterando o aspecto da cor azul
\definecolor{blue}{RGB}{41,5,195}


%% sumario com algarismos romanos
%\defbeamertemplate{subsection in toc}{bullets}{%
  %\leavevmode
  %\parbox[t]{1em}{\textbullet\hfill}%
  %\parbox[t]{\dimexpr\textwidth-1em\relax}{\inserttocsubsection}\par}
%\defbeamertemplate{section in toc}{sections numbered roman}{%
  %\leavevmode%
  %\MakeUppercase{\romannumeral\inserttocsectionnumber}.\ %
  %\inserttocsection\par}
%\setbeamertemplate{section in toc}[sections numbered roman]
%\setbeamertemplate{subsection in toc}[bullets]


%% fonte
    %\usefonttheme{structurebold}

    \setbeamerfont{title}{series=\bfseries,parent=structure}
    \setbeamerfont{subtitle}{size=\scriptsize,series=\bfseries,parent=structure}
    \setbeamerfont{author}{size=\scriptsize,series=\bfseries,parent=structure}
    \setbeamerfont{institute}{size=\scriptsize,series=\bfseries,parent=structure}
    \setbeamerfont{date}{size=\scriptsize,series=\bfseries,parent=structure}

%\hyphenation{corretamente}

%----------------------------------------------------------------------------------------
%	TITLE PAGE
%----------------------------------------------------------------------------------------

%\title[Short title]{Full Title of the Talk} % The short title appears at the bottom of every slide, the full title is only on the title page
\title[Sistema de Recomendação]{\textsc{Desenvolvimento de uma Biblioteca Computacional para Sistemas de Recomendação de Lojas de Comércio Online}} 

%\author[Antônio Viggiano, Fernando Fochi]{Antônio Viggiano, Fernando Fochi}

\institute[USP] % Your institution as it will appear on the bottom of every slide, may be shorthand to save space
{
\\[-1cm]

\includegraphics[width=0.2\textwidth]{../img/logo_poli} \\

Escola Politécnica da Universidade de São Paulo \\[0.5cm]

\begin{minipage}[t]{0.35\textwidth}
\begin{center}
Antônio Viggiano \\
\texttt{agfviggiano@gmail.com}
\end{center}
\end{minipage}
\begin{minipage}[t]{0.35\textwidth}
\begin{center}
Fernando Fochi \\
\texttt{fernando.fochi@gmail.com}
\end{center}
\end{minipage}

\vspace{0.5cm}
Prof. Dr. Fábio Gagliardi Cozman
}
\date{\today} % Date, can be changed to a custom date


% remover icones de bibliografia
\setbeamertemplate{bibliography item}[triangle]

%bibliography sem paginas
\usepackage{appendixnumberbeamer}

\usepackage{listings}
\usepackage{caption}
%numero em tabelas etc
\setbeamertemplate{caption}[numbered]

%espaco entre caption
%\usepackage{caption}
%\captionsetup{skip=0pt,belowskip=0pt}
\newcommand\numbered{\setbeamertemplate{footline}{%
  \raisebox{5pt}{\makebox[\paperwidth]{%
    \hfill\makebox[10pt]{%
      \scriptsize\insertframenumber}}}}}

\newcommand\unnumbered{\setbeamertemplate{footline}{}\addtocounter{framenumber}{-1}}


\begin{document}

%----------------------------------------------------------------------------------------
%	TITLE PAGE
%----------------------------------------------------------------------------------------

\title[Short title]{Full Title of the Talk} % The short title appears at the bottom of every slide, the full title is only on the title page

\author{John Smith} % Your name
\institute[UCLA] % Your institution as it will appear on the bottom of every slide, may be shorthand to save space
{
University of California \\ % Your institution for the title page
\medskip
\textit{john@smith.com} % Your email address
}
\date{\today} % Date, can be changed to a custom date

\begin{document}

\begin{frame}
\titlepage % Print the title page as the first slide
\end{frame}

\begin{frame}
\frametitle{Overview} % Table of contents slide, comment this block out to remove it
\tableofcontents % Throughout your presentation, if you choose to use \section{} and \subsection{} commands, these will automatically be printed on this slide as an overview of your presentation
\end{frame}

%!TEX root = index.tex
\begin{frame}
\frametitle{Sumário} % Table of contents slide, comment this block out to remove it
\tableofcontents % Throughout your presentation, if you choose to use \section{} and \subsection{} commands, these will automatically be printed on this slide as an overview of your presentation
\end{frame}
%!TEX root = index.tex
\section[Introdução]{Introdução}
\begin{frame}
\frametitle{Introdução}
\begin{block}{Definição: Sistemas de recomendação}
``São ferramentas e técnicas de software destinadas a prover sugestões de itens para usuários'' \cite{ricci2011introduction-chap1}
\end{block}
\end{frame}


\begin{frame}
\frametitle{Introdução}
\begin{block}{Etapas principais}
\begin{itemize}
	\item Aquisição dos dados de entrada
	\item Determinação das recomendações
	\item Apresentação dos resultados ao usuário
\end{itemize}
\end{block}
\end{frame}

\begin{frame}
\frametitle{Introdução}
\begin{itemize}
	\item importância econômica de lojas online
	\item criação de ferramentas \textit{open source} para a comunidade
\end{itemize}

\begin{center}
$\Downarrow$ 
\end{center}

\begin{center}
Desenvolvimento de um sistema de recomendação \par{} de produtos para e-commerces
\end{center}

\end{frame}

%!TEX root = index.tex
\chapter[Objetivos]{Objetivos}
\label{chap:objetivos}

O objetivo do presente Trabalho de Conclusão de Curso é o desenvolvimento de um Sistema de Recomendação de produtos para lojas de comércio online, e respectiva análise de desempenho das recomendações propostas. 

Serão avaliados diferentes algoritmos de recomendação, e será feita uma avaliação comparativa entre cada um dos métodos. A explicação detalhada de cada um deles se encontra no Capítulo \ref{chap:sintese_de_solucoes}.

O sistema a ser desenvolvido será, do ponto de vista da taxonomia tradicional dos sistemas de recomendação \cite{schafer1999recommender}, automático e persistente. Isso significa que as sugestões serão dadas sem a interação do usuário e que as compras anteriores serão levadas em conta. Essas características aproximam o entregável das ferramentas de marketing via e-mail, que sugerem produtos com uma determinada frequência aos usuários com base em seu histórico de compras. 

A qualidade das recomendações será avaliada tanto em termos da similaridade entre os itens efetivamente comprados pelo cliente com aqueles previstos pelo sistema de recomendação, quanto em termos de indicadores de erro tipo I e erro tipo II, como a medida F \cite{sarwar2000analysis}. 

Por meio de uma validação cruzada, analisaremos a influência dos principais parâmetros do problema na qualidade das recomendações, como o tamanho do banco de dados ou a quantidade de informações de itens e clientes utilizadas na recomendação. 

Será discutido o impacto dos principais desafios tecnológicos e científicos dos sistemas de recomendação na nossa proposta de solução, tais como a escalabilidade, a adaptação a novos usuários e a dispersão dos dados \cite{wei2007survey}. Também serão avaliadas as diferentes medidas de similaridade e modelos de predição na qualidade das recomendações. 

Ao final, será possível extrair uma validação experimental das diretrizes fundamentais a serem seguidas por e-commerces que desejem desenvolver um sistema de recomendação próprio, a partir de um banco de dados de clientes, produtos e histórico de compras. 
%!TEX root = index.tex
\section[Estado da Arte]{Estado da Arte}
\begin{frame}
\frametitle{Estado da Arte}
\end{frame}

%!TEX root = index.tex
\section[Requisitos]{Requisitos}
\begin{frame}
\frametitle{Requisitos}
\begin{block}{Requisitos funcionais}
\begin{itemize}
	\item EMA máximo: 
	\item $20\%$ para Precisão
	\item $20\%$ para Abrangência
	\item \textit{Throughput} mínimo
	\begin{itemize}
		\item 100 mil recomendações por hora
	\end{itemize}
\end{itemize}
\end{block}

\begin{block}{Requisitos não funcionais}
\begin{itemize}
	\item Escalabilidade
	\item Sistema genérico
	\begin{itemize}
		\item Padronização dos dados de entrada/saída
	\end{itemize}
	\item Código aberto
\end{itemize}
\end{block}
\end{frame}

%!TEX root = index.tex
\section[Metodologia]{Metodologia}
\begin{frame}{Metodologia}{Estruturação do banco de dados}

\begin{description}
	\item[100k] 100 000 avaliações de 943 usuários \\ para 1682 filmes
	\item[IMDB] 28 819 filmes
	\item[\textbf{IMDB-100k}] 943 usuários, 1682 filmes e 25 atributos
\end{description}
\end{frame}

\begin{frame}[fragile]
\frametitle{Metodologia}
\framesubtitle{Desenvolvimento da biblioteca}


\textbf{Ferramenta utilizada}
	\begin{description}
		\item[RStudio] Editor de texto e console
	\end{description}
\textbf{Estrutura da biblioteca}
\begin{columns}
\column{0.4\textwidth}
\begin{verbatim}
recsys/
|-- db
|   `-- ml-100k
|       |-- u.data
|       |-- u.item
|       |-- u.user
|       |-- ...
|-- methods
|   |-- fw.R
|   |-- ui.R
|   `-- up.R
\end{verbatim}
\column{0.5\textwidth}
\begin{verbatim}
|-- results
|   |-- benchmark.R
|   |-- performance.R
|   `-- run_tests.R
`-- setup
    |-- functions.R
    `-- setup.R
\end{verbatim}
\end{columns}
\end{frame}


\begin{frame}{Metodologia}{Validação cruzada}
\begin{columns}[t] 
\column{.5\textwidth} 
\textbf{Avaliação}
\begin{itemize}
	\item $T=75\%$ base de treinamento
	\item $H=75\%$ dados ``escondidos''
\end{itemize}
\begin{table}[h]
\begin{center}
	\caption{Avaliações $r_{ui}$}
    \begin{tabular}{ | c | c | c | c | c | c |} 
    \hline
     & $i_1$ & $i_2$ & $i_3$ & $i_4$ \\ \hline
     $u_1$ & - & 4 & 3 & 5 \\ \hline
     $u_2$ & 2 & 5 & - & 1 \\ \hline
     $u_3$ & 3 & - & - & 2 \\ \hline
     $u_4$ & (5) & (2) & (3) & 4 \\ \hline
    \end{tabular}
    % TODO color table
\end{center}
\end{table}
\column{.5\textwidth}
\textbf{Ambiente de testes} 
\begin{itemize}
    \item Máquina r3.large
    \item 2 vCPU
    \item 15 GB de memória RAM
    \item Amazon Linux AMI release 2014.09 x86\_64
    \item Custo total R\$ 5,70
 \end{itemize}

\end{columns}
\end{frame}
%!TEX root = index.tex
\chapter[Síntese de Soluções]{Síntese de Soluções}
\label{chap:sintese_de_solucoes}

%A fim de facilitar a compreensão dos métodos propostos neste trabalho, serão utilizadas as matrizes de avaliações $\mathbf{R}$ e de atributos $\mathbf{A}$ abaixo, adaptadas da Referência \citeonline{debnath2008feature}. Em todos os exemplos, considera-se valor mínimo $M=2$.

%\begin{table}[h]
%\begin{center}
%    \caption{Avaliações $r_{ui}$}
%%    \label{tab:aif_overspec}
%    \begin{tabular}{ | c | c | c | c | c | c | c | } 
%    \hline
%     & $i_1$ & $i_2$ & $i_3$ & $i_4$ & $i_5$ & $i_6$ \\ \hline
%     $u_1$ & - & 4 & - & - & 5 & - \\ \hline
%     $u_2$ & - & 3 & - & 4 & - & - \\ \hline
%     $u_3$ & - & - & - & - & - & 4 \\ \hline
%     $u_4$ & 5 & - & 3 & - & - & - \\ \hline
%    \end{tabular}
%\end{center}
%\end{table}
%
%\begin{table}[h]
%\begin{center}
%    \caption{Atributos $a_{if}$}
%%    \label{tab:aif_overspec}
%    \begin{tabular}{ | c | c | c | c | c | } 
%    \hline
%     & $f_1$ & $f_2$ & $f_3$ & $f_4$  \\ \hline
%     $i_1$ & 0 & 1 & 0 & 0  \\ \hline
%     $i_2$ & 1 & 1 & 0 & 0  \\ \hline
%     $i_3$ & 0 & 1 & 1 & 0  \\ \hline
%     $i_4$ & 0 & 1 & 0 & 0  \\ \hline
%     $i_5$ & 1 & 1 & 1 & 0  \\ \hline
%     $i_6$ & 0 & 0 & 0 & 1  \\ \hline
%    \end{tabular}
%\end{center}
%\end{table}

\section{Algoritmo baseado na ponderação de atributos (FW)} % (fold)
\label{sec:algoritmo_baseado_na_pondera_o_de_atributos_}

% section algoritmo_baseado_na_pondera_o_de_atributos_ (end)

O primeiro algoritmo que utilizaremos no sistema de recomendação, adaptado da Referência \citeonline{symeonidis2007feature} e denominado ponderação de atributos, \textit{feature weighting} ou \textit{FW}, trata-se de um híbrido entre filtragem colaborativa e filtragem baseada em conteúdo. A partir da regressão linear de dados de uma rede social (\textit{Internet Movie Database, IMDB}), extraem-se os pesos que determinam a importância de cada atributo dos itens, e é onde ocorre a filtragem colaborativa dos usuários. Após obtenção dos pesos, realiza-se a filtragem baseada em conteúdo para determinar os itens com maior similaridade, que são finalmente recomendados.

Na filtragem baseada em conteúdo, ``cada item é representado por um vetor de atributos ou \textit{features}''. A similaridade $s_{ij}$ entre dois itens $i$ e $j$ é dada pela média ponderada das distâncias entre as \textit{features} dos itens:

\begin{equation} 
\label{eq:sij}
    s_{ij} = \sum_{f}{w_{f} \left(1-d_{fij}\right)}
\end{equation}

As distâncias entre os atributos $d_f$ são determinadas conforme o tipo de dado avaliado e seu domínio, normalizadas no intervalo $\left[0,1\right]$. 

Para atributos literais, como categoria, marca, cor, etc., uma possível medida de distância é o delta de Kronecker descrito em \ref{eq:delta}. A similaridade entre as cores ``azul'' e ``vermelho'' é, nesse caso, 0, e sua distância é 1. O valor da distância é nulo se e somente se os atributos são idênticos.

Para atributos pertencentes a uma coleção finita de itens, tais como os atores participantes de um filme, é possível estabelecer a similaridade entre dois conjuntos a partir do índice Jaccard, descrito em \ref{eq:jaccard}. Neste caso, a similaridade entre os conjuntos \{Al Pacino, Tom Hanks\} e \{Tom Hanks, Marlon Brando\} é $1/3$, e a sua distância é $2/3$.


\begin{equation}
\label{eq:delta}
\delta_{mn} =  
\begin{cases}
1, &\text{se }m=n \\
0, &\text{se }m \neq n
\end{cases} 
\end{equation}

\begin{equation}
\label{eq:jaccard}
J(A,B) ={{|A \cap B|}\over{|A \cup B|}}
\end{equation}

Vale considerar a correlação entre atributos no cálculo das distâncias: a similaridade de duas marcas de calçado, por exemplo, é maior que a de duas marcas de produtos de categorias diferentes, mesmo que as marcas sejam distintas nos dois casos. Em uma primeira análise, todavia, utilizaremos para a maior parte das \textit{features} as medidas de distância do delta de Kronecker \ref{eq:dfij_delta} e do índice Jaccard \ref{eq:dfij_jaccard}. Isso significa que se os atributos de dois itens são idênticos, a distância é nula e portanto a similaridade é máxima. O sumário de algumas medidas de distância que podem ser utilizadas estão na Tabela \ref{tab:medidas-distancia}.

\begin{equation}
\label{eq:dfij_delta}
\begin{split}
d_{fij} =&~ 1-\delta_{ij}^f \\
    =&~ 1-\delta_{a_{if} a_{jf}}
\end{split} 
\end{equation}

\begin{equation}
\label{eq:dfij_jaccard}
\begin{split}
d_{fij} =&~ 1-J^f(i,j) \\
    =&~ 1-J(a_{if},a_{jf})
\end{split} 
\end{equation}

\begin{table}[hp]
\begin{center}
    \caption{Medidas de distância entre alguns atributos}
    \label{tab:medidas-distancia}
    \begin{tabular}{  | >{\arraybackslash} m{3cm} | >{\arraybackslash} m{3cm} | >{\centering\arraybackslash} m{3cm} | } 
    \hline
    \textbf{Atributo} $f$ & \textbf{Domínio} $\mathrm{F}$ & \textbf{Distância} $d_f$ \\ \hline
    Marca & Literal & $1-\delta^f_{ij}$ \\ \hline    
    Esporte & Literal & $1-\delta^f_{ij}$ \\ \hline
    Gênero & Literal & $1-\delta^f_{ij}$ \\ \hline            
    Categoria & Conjunto Literal & $1-J^f(i,j)$ \\ \hline            
    Preço & $\mathbb{R}$ & $ \frac{\left| a_{if}-a_{jf} \right|}{\max_{i,j}{\left| a_{if}-a_{jf} \right|}} $ \\ \hline
    Data & $\mathbb{R}$ milissegundos a partir de \textit{epoch} \cite{epoch} & $ \frac{\left| a_{if}-a_{jf} \right|}{\max_{i,j}{\left| a_{if}-a_{jf} \right|}} $ \\ \hline
    \end{tabular}
\end{center}
\end{table}
 
Os pesos $w_f$ são a priori desconhecidos. A Referência \citeonline{symeonidis2007feature} os determina a partir de uma regressão linear do tipo \ref{eq:regressao-linear}, onde $e_{ij}$ é o número de usuários que se interessam tanto por $i$ quanto por $j$. Esses valores permitem determinar ``o julgamento humano de similaridade entre itens'', e pode ser calculado a partir da matriz de avaliações, conforme a equação \ref{eq:determinacao-eij}. O operador booleano $\mathrm{b}_M$, descrito pela Equação \ref{eq:b0}, nada mais é que uma ferramenta matemática para se poder extrair o número de usuários que avaliaram positivamente tanto $i$ quanto $j$ a partir de $\mathbf{R}$. 

\begin{equation}
\label{eq:regressao-linear} 
    e_{ij} = w_0 + \sum_{f}{w_{f} \left(1-d_{fij}\right)}
\end{equation} 


\begin{equation}
\label{eq:determinacao-eij} 
    e_{ij} = \sum_{u}{\mathrm{b_M}\left(r_{ui} ~ r_{uj}\right)}
\end{equation} 

\begin{equation}
\label{eq:b0}
\mathrm{b}_M\left(x\right) = 
\begin{cases}
1, &\text{se }x>M \\
0, &\text{se }x\leq M
\end{cases} 
\end{equation}

Desta forma, os pesos $w_f$ são determinados a partir resolução do sistema de equações lineares \ref{eq:determinacao-wf}. Apenas os pesos positivos e com valor absoluto expressivo (maior que um piso arbitrariamente escolhido a posteriori) são utilizados na recomendação. 

\begin{equation}
\label{eq:determinacao-wf} 
    w_0 + \sum_{f}{w_{f}  \left(1-d_{fij}\right)} = \sum_{u}{\mathrm{b_0}\left(r_{ui} ~ r_{uj}\right)},~\forall i \neq j 
\end{equation} 

Calcula-se a matriz de similaridade $\mathbf{S}$ pela equação \ref{eq:sij} e recomendam-se os itens similares àqueles já comprados, segundo \ref{eq:ifw}.

\begin{equation}
\label{eq:ifw} 
    \hat{\imath}_u = \argmax_{i \in \left\{i~|~r_{ui} > 0\right\}, j}{s_{ij}}
\end{equation} 

%Para o exemplo proposto, o valor da matriz $e_{ij}$ seria


%\begin{table}[h]
%\begin{center}
    %\caption{$e_{ij}$}
%%    \label{tab:aif_overspec}
    %\begin{tabular}{ | c | c | c | c | c | c | c | } 
    %\hline
     %& $i_1$ & $i_2$ & $i_3$ & $i_4$ & $i_5$ & $i_6$  \\ %\hline
     %$i_1$ & 0 & 1 & 0 & 0  \\ \hline
     %$i_2$ & 1 & 1 & 0 & 0  \\ \hline
     %$i_3$ & 0 & 1 & 1 & 0  \\ \hline
     %$i_4$ & 0 & 1 & 0 & 0  \\ \hline
     %$i_5$ & 1 & 1 & 1 & 0  \\ \hline
     %$i_6$ & 0 & 0 & 0 & 1  \\ \hline
    %\end{tabular}
%\end{center}
%\end{table}

\section{Algoritmo baseado no perfil de usuários (UP)} % (fold)
\label{sec:algoritmo_baseado_no_perfil_de_usu_rios_}

% section algoritmo_baseado_no_perfil_de_usu_rios_ (end)

O segundo algoritmo, adaptado de \citeonline{debnath2008feature}, é um hibrido entre filtragem colaborativa e filtragem baseada em conteúdo. Os atributos dos itens são ponderados no cálculo de similaridade, com pesos extraídos de um modelo de perfil de usuários, denominado \textit{user profile} ou \textit{UP}. Esse perfil leva em consideração o interesse dos usuários por \textit{features}, indiretamente calculado a partir de seu interesse pelos itens. 

Se o usuário avaliou \textit{positivamente} algum item $r_{ui}$, tal que $r_{ui}$ é superior a um valor mínimo $M$, considera-se que $u$ tem interesse $t_{uf}$ nos atributos $f$ dos itens $i$, representados por $a_{if}$. A correlação $t_{uf}$ entre usuários e \textit{features} é descrita por \ref{eq:puf}.

\begin{equation}
\label{eq:puf} 
    t_{uf} = \sum_{i}{\mathrm{b}_M\left(r_{ui}~a_{if}\right)} 
\end{equation} 

Os pesos $w_{uf}$, que mostram a relevância de $f$ para $u$, são determinados a partir da estatística TF-IDF (\textit{term frequency--inverse document frequency}), presente em formulações de recuperação de informação e mineração de dados. Em nosso caso, TF ou \textit{feature frequency} é a ``similaridade intra-usuários'' $p_{uf}$, igual ao número de vezes em que a \textit{feature} $f$ aparece no perfil do usuário $u$ (equação \ref{eq:tf}). O termo IDF ou \textit{inverse user frequency} é a ``dissimilaridade inter-usuários'' $q_{f}$, relacionada com o inverso da frequência $\check{q}_{f}$ de um atributo $f$ dentro de todos os usuários (equações \ref{eq:uf} e \ref{eq:iuf}).

\begin{equation}
\label{eq:tf} 
    p_{uf} = t_{uf}
\end{equation} 


\begin{equation}
\label{eq:uf} 
    \check{q}_{f} = \sum_{u}{\mathrm{b}_0\left(t_{uf}\right)}
\end{equation} 

\begin{equation}
\label{eq:iuf} 
    q_{f} = \log \left( \frac{\left|~\mathcal{U}~\right|}{\check{q}_{f}} \right)
\end{equation} 

Os pesos $w_{uf}$, obtidos na TF-IDF \ref{eq:w-tfidf}, são utilizados para calcular a similaridade $s_{uv}$ entre dois usuários $u$ e $v$, conforme \ref{eq:suv}, \ref{eq:fuv}.

\begin{equation}
\label{eq:w-tfidf} 
    w_{uf} = p_{uf}~q_{f}
\end{equation} 


\begin{equation}
\label{eq:suv}
    s_{uv} = \frac{\sum\limits_{f \in \mathcal{F}_{uv}}{w_{uf}~w_{vf}}}{\sqrt{\sum\limits_{f \in \mathcal{F}_{uv}
    }w_{uf}^2} \sqrt{\sum\limits_{f \in \mathcal{F}_{uv}}w_{vf}^2}} 
\end{equation} 

\begin{equation}
\label{eq:fuv}
\begin{split}
    \mathcal{F}_{uv} &= \mathcal{F}_u \cap \mathcal{F}_v \\
    \mathcal{F}_u &= \left\{ f \in \mathcal{F}~|~t_{uf} > 0 \right\}
\end{split}    
\end{equation} 

Dispondo-se de $\mathbf{S}$, selecionam-se os $k$ vizinhos mais próximos $v_k^u$ com maior similaridade $s_{uv}$.  Posteriormente, determina-se o conjunto $\mathcal{I}_{v_k^u} = \left\{ i ~|~ r_{v_k^u i} > M\right\}$ de itens $i$ avaliados positivamente por $v_k^u$. Em \ref{eq:frf} avalia-se a frequência total $\mathrm{f}_{uf}$ dos atributos $f$ para os itens de $\mathcal{I}_{v_k^u}$. 

\begin{equation}
\label{eq:frf} 
\mathrm{f}_{uf} = \sum_{i \in \mathcal{I}_{v_k^u}}{\mathrm{b}_0\left(a_{if}\right)}
\end{equation} 

Por fim, a partir da equação \ref{eq:wi} calcula-se o peso $\omega_{ui}$ de cada item e gera-se a lista dos \textit{top-N} produtos a serem recomendados para o usuário $u$, conforme \ref{eq:iup}. 

\begin{equation}
\label{eq:wi} 
    \omega_{ui} = \sum_{f}{a_{if}~\mathrm{f}_{uf}}
\end{equation} 

\begin{equation}
\label{eq:iup} 
    \hat{\imath}_u = \argmax_{i \in \left\{i~|~r_{ui} > 0\right\}}{\omega_{ui}}
\end{equation} 

\section{Algoritmo baseado na correlação usuário-item (UI)} % (fold)
\label{sec:algoritmo_baseado_na_correla_o_usu_rio_item_ui_}

Este método... TODO variante

A partir da matriz de correlações ponderadas entre usuários e atributos $\mathbf{W}$, e da matriz de atributos dos itens $\mathbf{A}$, é possível extrair a correlação $\omega_{ui}$ entre usuários $u$ e itens $i$. A lista dos $N$ produtos a serem recomendados decorre portanto das equações \ref{eq:wui} e \ref{eq:iup}.

\begin{equation}
\label{eq:wui} 
    \omega_{ui} = \sum_{f}{w_{uf}~a_{if}}
\end{equation} 

Ao passo que o método \textit{UP} recomenda itens a partir dos $k$ vizinhos mais próximos, o algoritmo \textit{UI} busca os itens com \textit{features} mais similares aos atributos pelos quais $u$ se interessa. Espera-se que esse tipo de recomendação forneça sugestões de qualidade similar ao algoritmo original, pois os dois estão fundamentados no fato que o usuário se interessa pelos atributos $f$ dos itens $i$. 


% subsection variante_correla_o_usu_rio_item_ (end)

%%!TEX root = index.tex
\section[Biblioteca]{Biblioteca}
\begin{frame}{Biblioteca}{Ferramentas Utilizadas}

\textbf{IDE RStudio}
	\begin{itemize}
		\item Console
		\item Editor de texto e corretor de sintaxe
		\item Suporte a execução direta de código
		\item Visualização de gráficos
		\item Depuração de erros
		\item Gerenciamento de espaço de trabalho
	\end{itemize}
\end{frame}

\begin{frame}{Biblioteca}{Estrutura}
\textbf{Quatro principais seções}
\begin{description}
	\item DB
	\begin{itemize}
		\item Contém o banco de dados MovieLens 100k
	\end{itemize}
	\item Methods
	\begin{itemize}
		\item Contém os algoritmos de recomendação
	\end{itemize}
	\item Results
	\begin{itemize}
		\item Contém os métodos de avaliação de qualidade
	\end{itemize}
	\item Setup
	\begin{itemize}
		\item Contém funções de suporte
	\end{itemize}
\end{description}
\end{frame}

%%!TEX root = index.tex
\section[Avaliação de Desempenho]{Avaliação de Desempenho}
\begin{frame}{Avaliação de Desempenho}


\begin{itemize}
	\item Distância entre recomendações
	\begin{itemize}
		\item $EMA = \left|\hat{\textbf{\i}} - \textbf{i}\right|$
	\end{itemize}
	\item Desempenho mediante a mudança nas variáveis 
	\begin{itemize}
		\item Quantidade de atributos utilizados
	\end{itemize}
	\item Tempo de execução
	\begin{itemize}
		\item Em função do algoritmo
		\item Em função do tamanho do banco de dados
	\end{itemize}
\end{itemize}

\begin{table}[ht]
\begin{center}
    \label{tab:avaliacao-predicao}
    \caption{Avaliação de sistemas de predição}
    \begin{tabular}{  | p{2cm} | p{2.5cm} | p{4cm} | }
    \hline
    \textbf{Medida} & \textbf{Fórmula} & \textbf{Significado} \\ \hline
    Precisão &  $\frac{VP}{VP+FP}$ & \% Predições corretas de casos positivos \\ \hline                            
    Acurácia & $\frac{VP+VN}{VP+VN+FP+FN}$ & \% Predições corretas  \\ \hline
    \end{tabular}
\end{center}
\end{table}

\end{frame}


%!TEX root = index.tex
\chapter[Resultados]{Resultados}
\label{chap:resultados}

Até o presente momento, os resultados deste Trabalho de Conclusão de Curso concentram-se na definição de necessidades, de parâmetros de sucesso e de síntese de possíveis soluções. 

Visto que a primeira etapa de um sistema de recomendação é a coleta e manipulação de dados, definimos que a aquisição de dados será feita a partir de um banco de dados genérico, que deverá alimentar o sistema por meio de arquivos de texto com valores separados por vírgulas (\texttt{.csv}). A fim de facilitar o pré-processamento dos dados, exigem-se três arquivos, cada um com uma tabela de itens e seus atributos $\mathbf{A}$, clientes e suas características $\mathbf{B}$ e histórico de compras ou avaliações $\mathbf{R}$. Caso existam outras tabelas no banco de dados, o sistema deverá ser alterado para levar em conta o processamento dos arquivos suplementares.

\begin{equation} 
\begin{split} 
\mathbf{A} &= 
\begin{bmatrix} 
 a_{i_1 f_1} &  a_{i_1 f_2} &  a_{i_1 f_3}  & \dots   \\
 a_{i_2 f_1} &  a_{i_2 f_2} &  a_{i_2 f_3}  & \dots   \\
 a_{i_3 f_1} &  a_{i_3 f_2} &  a_{i_3 f_3}  & \dots  \\ 
 \vdots &  \vdots &  \vdots  & \ddots   \\
 \end{bmatrix} \\
\mathbf{B} &= 
\begin{bmatrix} 
 b_{u_1 c_1} &  b_{u_1 c_2} &  b_{u_1 c_3}  & \dots   \\
 b_{u_2 c_1} &  b_{u_2 c_2} &  b_{u_2 c_3}  & \dots   \\
 b_{u_3 c_1} &  b_{u_3 c_2} &  b_{u_3 c_3}  & \dots  \\ 
 \vdots &  \vdots &  \vdots  & \ddots   \\
 \end{bmatrix}\\
  \mathbf{R} &= 
\begin{bmatrix} 
  r_{u_1 i_1} &  r_{u_1 i_2} &  r_{u_1 i_3}  & \dots   \\
 r_{u_2 i_1} &  r_{u_2 i_2} &  r_{u_2 i_3}  & \dots   \\
 r_{u_3 i_1} &  r_{u_3 i_2} &  r_{u_3 i_3}  & \dots  \\ 
 \vdots &  \vdots &  \vdots  & \ddots   \\
\end{bmatrix}
\end{split} 
\end{equation}

Em alguns bancos de dados, a tabela de histórico também contém outras informações adicionais $\theta$, tais como o método de pagamento, a data da compra, data de entrega, etc., e é denominada $\mathbf{H}$.

\begin{equation} 
\mathbf{H} =
\begin{bmatrix} 
 r_{u_1 i_1} &  \theta_{h_1 1} &  \theta_{h_1 2} & \dots   \\
 r_{u_1 i_2} &  \theta_{h_2 1} &  \theta_{h_2 2} & \dots   \\
 r_{u_1 i_3} &  \theta_{h_3 1} &  \theta_{h_3 2} & \dots   \\
 \vdots &  \vdots &  \vdots  & \ddots   \\
 \end{bmatrix} \\
\end{equation}


Uma vez determinada a forma de entrada de dados, definiu-se a escolha do conjunto de dados a serem utilizados. O primeiro conjunto de dados abertos é proveniente do website de recomendações de filmes MovieLens (\url{http://movielens.umn.edu}). Nessa base de dados, o catálogo de filme faz o papel de catálogo de produtos pelos quais os usuários possam se interessar, e o histórico de compras se refere à avaliação dos filmes feita por cada usuário. Outros conjuntos de dados também serão  explorados pela dupla, tais como os dados de classificação de músicas do serviço Yahoo! Music (\url{http://webscope.sandbox.yahoo.com}) ou de dados anônimos de e-commerces.

Por fim, as possíveis soluções do projeto abrangem o cálculo das medidas de similaridade entre itens, para os conjuntos de dados que não foram previamente tratados. Determinar um valor numérico entre dois produtos distintos, tais como uma camiseta e uma prancha de \textit{surf}, é uma tarefa complexa e sujeita a erros humanos. Após reflexão e leitura de referências, definimos possíveis maneiras de realizar esse cálculo: 

\begin{itemize} 
	\item Grupos de similaridade: a similaridade dos itens seria definida por pelas características do item (como ``esporte radical'', ``corrida'', ``filme de aventura'', etc). Seria necessário classificar manualmente esses atributos (por exemplo, determinar que ``esporte radical'' tem similaridade de $3/5$ com ``corrida'' e $1/5$ com ``produtos de limpeza'').
	\item Histórico de compra da comunidade: quanto mais usuários comprarem a mesma dupla de itens, maior a similaridade entre estes itens. A classificação se faz pelo histórico mas o índice de similaridade pertence aos itens e não entre os usuários.
	\item Ranking por arestas: adaptado do sistema \textit{edge rank}, de classificação de posts no Facebook, este sistema levaria em conta a multiplicação de diversas características do item. Características como: popularidade da marca, número de compras por visualização do item, popularidade da marca para o usuário em questão (quantos itens ele comprou desta marca), há quanto tempo o item foi lançado e a relação do item com as compras anteriores do usuário (se este tipo de item já fez sucesso com este usuário).
\end{itemize}

Para os itens que já foram avaliados, como no caso dos conjuntos de dados de classificação de filmes ou músicas, essa etapa não é necessária, pois a similaridade entre dois itens provém diretamente da avaliação do usuário. Dizer que um filme $A$ tem avaliação $4/5$ e um filme $B$ tem avaliação $5/5$ equivale a dizer que os dois são interessantes para aquele usuários, e por isso vale recomendar filmes similares a $A$ e $B$. Nesse caso, passa-se  diretamente para a etapa das recomendações.

Os resultados práticos de cálculo de similaridade ou descrição completa do banco de dados serão apresentados no relatório final do trabalho, em conjunto com os demais resultados da evolução do projeto. 
%!TEX root = index.tex
\chapter{Conclusão}
%\chapter{Trabalhos Futuros}
\label{cha:trabalhos_futuros}

A avaliação de desempenho dos métodos propostos na biblioteca deste Trabalho de Conclusão de Curso verificaram resultados já conhecidos no meio acadêmico.

Em particular, a dependência entre qualidade de recomendação e tamanho da lista de sugestões se verificou (impacto de $N$). Além disso, mostramos que um banco de dados com maior quantidade de avaliações (impacto de $H$) tem mais relevância que um banco de dados com mais usuários (impacto de $T$). 

Outro resultado do trabalho foi a comprovação do fenômeno de \textit{hidden feedback} (impacto de $M$). Mesmo que construamos métodos embasados na ``avaliação positiva'' dos usuários, esse parâmetro pode não ter tanta influência, visto que a maioria das avaliações dos clientes já são de fato positivas.

Para os trabalhos futuros, iremos realizar a validação cruzada e avaliar se os requisitos funcionais foram estabelecidos. Em seguida, procuraremos melhorar o sistema de recomendação a fim de torná-lo mais genérico. Buscaremos eliminar restrições quanto a entrada e saída de dados, de forma que elas sejam completamente arbitrárias. O objetivo é que o usuário possa informar ao sistema como é formado sua base, e que todo o tratamento preliminar seja feito automaticamente. 

Caso haja tempo, trabalharemos também na construção de um \textit{driver} que possibilite a conexão entre o sistema de recomendação e um banco de dados SQL, sem que seja necessária a etapa intermediária de arquivos \texttt{csv} para aquisição de dados. Planejamos elaborar um \textit{website} para o sistema de recomendação e exportar toda a lógica para um servidor dedicado. Outra melhoria desejada é a reconstrução dos métodos na linguagem de programação C, a fim de melhorar a performance computacional. Dessa forma, o serviço de ``sistema de recomendação nas nuvens'' estaria completo e poderia ser utilizado por e-commerces reais.
%%!TEX root = index.tex
\section[Cronograma]{Cronograma}
\begin{frame}{Cronograma}
\begin{itemize}
	\item[09/07] Pré-tratamento do banco de dados
 	\item[16/07] \textbf{Programação} do método Ponderação de Atributos
 	\item[23/07] \textbf{Programação} do método Perfil de Usuários
 	\item[30/07] Análise comparativa dos dois algoritmos
 	\item[]
 	\item[13/08] Relatório de atividades de implementação
 	\item[27/08] Primeiros testes com o sistema \\(\textbf{precisão e acurácia} para uma base de testes)
 	\item[]
 	\item[03/09] Testes com o sistema (\textbf{validação cruzada})
 	\item[24/09] Melhorias incrementais e relatório de atividades
 	\item[]
 	\item[15/10] \textbf{Relatório aprofundado} de atividades
 	\item[]
 	\item[05/11] Elaboração da apresentação e finalização dos relatórios
 	\item[12/11] Melhorias incrementais
\end{itemize}
\end{frame}

\appendix
%!TEX root = index.tex
\printbibliography[heading=bibintoc]
\end{document} 
