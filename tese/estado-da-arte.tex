%!TEX root = index.tex
\chapter[Estado da arte]{Estado da Arte}
\label{chap:estado_da_arte}

As terminologias \textit{cliente} e \textit{usuário} neste texto serão intercambiáveis e sem distinção semântica, mesmo que na prática essas duas entidades possam ser diferentes. Da mesma forma, \textit{item} e \textit{produto} terão o mesmo significado neste trabalho. 

A fim de tornar a formulação mais genérica, também não faremos distinção entre \textit{avaliação positiva} de um item e \textit{compra} de um item. Avaliação positiva é toda avaliação $r_{ui}$ do item $i$ feital pelo usuário $u$ tal que $r_{ui} > M$, e avaliação negativa tal que $r_{ui} \leq M$, sendo $M$ um valor mínimo escolhido a priori, indicador de que o usuário $u$ ``gostou'' do item $i$. No caso de um banco de dados sem avaliações dos produtos, será levada em conta a compra dos itens e será admitida avaliação unitária e valor mínimo nulo. Desta forma, os bancos de dados que contenham informações do tipo ``usuário $u$ avaliou o item $i$ em $r_{ui} = 3.54 > M$'' e aqueles que contenham ``usuário $u$ comprou o item $i$, logo $r_{ui} = 1 > 0$'' serão tratados equivalentemente. Vale observar que essa definição difere da Referência \citeonline{symeonidis2007feature}, em que avaliação positiva é aquela tal que $r_{ui} \geq M$.

\section{Estado da arte dos problemas} % (fold)
\label{sec:estado_da_arte_dos_problemas}

O problema de recomendação pode ser formulado como se segue, adaptado da Referência \citeonline{adomavicius2005toward}, com notação inspirada no artigo \citeonline{symeonidis2007feature}: 

``Seja $\mathcal{U}$ o conjunto de todos os usuários e seja $\mathcal{I}$ o conjunto de todos os itens que podem ser recomendados, tais como livros, filmes ou artigos científicos. Seja $\ell$ uma função de utilidade, que mede a relevância do produto $i$ para usuário $u$. Em notação matemática, $\ell: \mathcal{U} \times \mathcal{I} \rightarrow \mathcal{R}$, onde $\mathcal{R}$ é um  conjunto totalmente ordenado -- por exemplo, números inteiros ou números reais dentro de um determinado intervalo, em geral $\left\{-1, 0, +1\right\}$ ou $[1, 5]$. O objetivo do sistema de recomendação é determinar o item $\tilde{\imath}_u$ que maximize a utilidade $\ell_{ui}$ do usuário $u$.''

\begin{equation} 
\label{eq:utilidade}
\forall u \in \mathcal{U}, ~ \tilde{\imath}_u = \argmax_{i \in \mathcal{I}}{\ell_{ui}}
\end{equation}

O problema central da recomendação é que ``em geral a função $\ell$ é desconhecida ou não é definida para todo o espaço $\mathcal{U} \times \mathcal{I}$'', e portanto determinar $\tilde{\imath}$ através da Equação \ref{eq:utilidade} é inviável. 

Em algumas formulações, ``a utilidade é descrita pela avaliação $r_{ui}$ do item $i$ feita pelo usuário $u$''. Neste caso, o sistema de recomendação busca determinar $\hat{r}_{ui}$ que melhor se aproxime de $r_{ui}$, e a qualidade da recomendação é normalmente descrita pela distância entre esses dois valores. Em outros sistemas, todavia, a utilidade é descrita diferentemente, de forma que o item com maior valor de $\hat{r}_{ui}$ não é necessariamente recomendado. 

Para lidar com o problema da recomendação, existem três grandes grupos de estratégias de sugestão de itens, segundo as Referências \citeonline{adomavicius2005toward,balabanovic97fab}:

\begin{itemize}
\item Recomendações baseadas em conteúdo: o usuário recebe sugestões de itens similares àqueles pelos quais ele se interessou no passado;
\item Recomendações colaborativas: o usuário recebe sugestões de itens que pessoas com preferências semelhantes gostaram no passado;
\item Recomendações híbridas: esses métodos combinam caracteríticas de sistemas colaborativos e baseados em conteúdo.  O usuário recebe sugestões de itens compatíveis com seu perfil e de itens do interesse de usuários com perfil similar.
\end{itemize}

As estratégias de recomendação baseadas em conteúdo exploram os dados dos itens para calcular a sua relevância conforme o perfil do usuário. Suas técnicas de recomendação podem ser classificadas em dois grupos: aquelas baseadas em heurísticas ou memória -- fazem a previsão com base em toda a coleção de itens anteriormente classificados pelos usuários -- e aquelas baseadas em modelos -- utilizam o conjunto de avaliações com o objetivo de descrever a interação entre usuários e itens, tal como em uma regressão linear ou em uma rede Bayesiana. 

Na abordagem de sistemas baseados em conteúdo, a recomendação pode ser vista como um problema de aprendizado de máquina, em que o sistema adquire conhecimento sobre o usuário. Muitas vezes é recomendado que o aprendizado seja feito com base no perfil do usuário em uso continuo, ao invés de forçá-lo a responder diversas perguntas demográficas, como idade, gênero, classe social, etc \cite{wei2007survey}. O objetivo é categorizar novas informações baseadas em informações previamente adquiridas e rotuladas como interessantes ou não pelo usuário. Com estas informações em mão, é possível gerar modelos preditivos que evoluem conforme aparecem novas informações.

Em sistemas baseados em conteúdo, os itens a serem recomendados podem possuir diversos atributos e formas de classificação. Em documentos como e-mails, \textit{websites} ou comentários de usuários, os textos não tem estrutura definida e a abordagem mais comum para escolher o melhor item é a mineração de informações. O usuário procura por uma lista de termos desejados e o sistema retorna os textos de maior relevância, tal como é feito em um motor de busca \cite{schafer2001commerce}. Nesses casos, calcula-se a similaridade entre documentos a partir da importância das palavras ou termos similares, como a TF-IDF ou o classificador Bayesiano \cite{lops2011content-chap3}. 

Em bancos de dados relacionais, os itens possuem uma categorização pré-definida, e sua relevância depende das suas características, descritas pela matriz de atributos $\mathbf{A}$. Cada \textit{feature} pertence a um conjunto distinto, podendo ser booleano (possui ou não possui), inteiro ou real (preço, data, etc.), ou um coleção finita de valores (marca, modelo, gênero, etc.), como exemplifica a Tabela \ref{tab:aif}.

\begin{table}[h]
\begin{center}
    \caption{Atributos $a_{if}$}
    \label{tab:aif}
    \begin{tabular}{ | c | c | c | c | c | } 
    \hline
     & $f_1$ & $f_2$ & $f_3$ & $f_4$ \\ \hline
     $i_1$ & 1 & 50 & 0.8 & P \\ \hline
     $i_2$ & 0 & 75 & 0.3 & M \\ \hline
     $i_3$ & 1 & 30 & 0.4 & G \\ \hline
    \end{tabular}
\end{center}
\end{table}

As recomendações colaborativas, por sua vez, tentam prever a utilidade dos itens para cada cliente com base em itens previamente avaliados por outros usuários. Elas podem ser baseadas em usuários, isto é, na escolha de clientes  que possuam avaliações similares de produtos, quanto baseadas em itens, na escolha de produtos avaliados similarmente \cite{linden2003amazon}.

Mais formalmente, quando a filtragem colaborativa é baseada em usuários, a utilidade $\ell_{ui}$ de um item $i$ para um usuário $u$ é estimada com base nas utilidades $\ell_{v_k^u i}$ dos usuários $v_k^u \in \mathcal{U}$ que são ``similares'' ao usuário $u$.  De maneira análoga, quando baseada em itens, a utilidade $\ell_{ui}$ é prevista com base nas utilidades $\ell_{u j_k^u}$, dado itens $j_k^u \in \mathcal{I}$ que são ``similares'' aos itens $i$. 

Na prática, o cálculo das recomendações para sistemas colaborativos é feito a partir da matriz de avaliações $\mathbf{R}$. Isso pode ser exemplificado pela Tabela \ref{tab:rui}, que possui avaliações de 1 a 5, sendo $M=2$. Em um sistema usuário-usuário, o cliente $u_1$ receberia recomendação do item $i_4$, pois para os itens $i_2$ e $i_3$ suas avaliações foram similares às do cliente $u_2$. Já para um sistema item-item, o usuário $u_3$ receberia recomendação do item $i_3$, pois este tem avaliações similares às do item $i_2$, avaliado positivamente pelo usuário $u_3$.

\begin{table}[h]
\begin{center}
	\caption{Avaliações $r_{ui}$}
	\label{tab:rui}
    \begin{tabular}{ | c | c | c | c | c | c |} 
    \hline
     & $i_1$ & $i_2$ & $i_3$ & $i_4$ \\ \hline
     $u_1$ & - & 4 & 3 & - \\ \hline
     $u_2$ & - & 4 & 3 & 5 \\ \hline
     $u_3$ & 2 & 5 & - & 1 \\ \hline
    \end{tabular}
\end{center}
\end{table}


Por fim, as recomendações híbridas combinam aspectos tanto da filtragem colaborativa (baseada em usuários ou em itens) quanto da filtragem baseada em conteúdo, com o objetivo de atingir uma melhor recomendação ou de superar problemas recorrentes nas técnicas individuais, como a esparsidade (\textit{sparsity}) dos dados ou o \textit{cold start} \cite{burke2007hybrid}.

\section{Estado da arte das soluções} % (fold)
\label{sec:estado_da_arte_das_solu_es}

% section estado_da_arte_das_solu_es (end)
Do ponto de vista do estado da arte das soluções, as variáveis de interesse estão ligadas do número de usuários no sistema, ao número de itens, à medida de qualidade da recomendação e ao custo computacional \cite{lee2012comparative}.

No que se refere à dependência do número de usuários, a filtragem colaborativa baseada em usuários é extremamente efetiva para um baixo número de usuários. A filtragem colaborativa a base de itens é consideravelmente pior para um baixo número de usuários, mas supera todos os outros métodos baseados em memória conforme o número de clientes aumenta.

A dependência do número de itens é, de certa forma, oposta à de usuários: a filtragem colaborativa baseada em itens é extremamente efetiva para poucos itens, enquanto aquela baseada em usuários supera todos os outros métodos baseados em memória para grandes quantidades de itens.

Com relação à medida de qualidade, avaliada a partir da abrangência dos dados, a filtragem baseada em usuários e a baseada em itens mostram uma dependência semelhante. Na análise de menor erro quadrático médio entre o item sugerido e o item efetivamente comprado, todos os métodos de recomendação variam não-linearmente com o número de usuários, itens e acurácia, e de modo geral há um compromisso (\textit{trade-off}) entre a esparsidade (\textit{sparsity}) dos dados e o tempo de processamento. 
% section estado_da_arte_dos_problemas_e_das_solu_es (end)

\section{Desafios científicos e tecnológicos} % (fold)
\label{sec:desafios_cient_ficos_e_tecnol_gicos}

Um dos maiores desafios tecnológicos dos sistemas de recomendação é, atualmente, o da escalabilidade \cite{wei2007survey}. O sistema de recomendação deve ser flexível no sentido de poder operar igualmente bem tanto em pequenas quanto em grandes bases de dados, que podem chegar até centenas de milhões de clientes \cite{amazoncustomers} e de produtos \cite{amazonproducts}. Isso significa que as recomendações devem ser suficientemente rápidas e ainda assim prover sugestões valiosas aos consumidores. 

Um problema muito comum nos sistemas de recomendação é o do \textit{cold start}, que atinge principalmente os sistemas de filtragem colaborativa, grandemente dependentes da matriz de avaliações $\mathbf{R}$. Quando itens ou usuários são inicialmente introduzidos no sistema, existe pouca ou nenhuma informação sobre eles. O sistema é incapaz de realizar inferências sobre quais itens recomendar ao novo usuário ou sobre quais produtos são similares ao novo item. Na Tabela \ref{tab:rui_cold_start}, por exemplo, o item $i_{100}$ não possui nenhuma avaliação, e nunca seria recomendado em sistemas puramente baseados em itens. Analogamente, o usuário $u_3$ também não teria nenhuma sugestão de itens para sistemas puramente baseados em usuários.

\begin{table}[h]
\begin{center}
    \caption{Avaliações $r_{ui}$}
    \label{tab:rui_cold_start}
    \begin{tabular}{ | c | c | c | c | c | c |} 
    \hline
     & $i_1$ & $i_2$ & $\cdots$ & $i_{100}$ \\ \hline
     $u_1$ & 5 & 4 & $\cdots$ & - \\ \hline
     $u_2$ & - & 2 & $\cdots$ & - \\ \hline
     $u_3$ & - & - & $\cdots$ & - \\ \hline
    \end{tabular}
\end{center}
\end{table}

Também é uma grande dificuldade dos algoritmos baseados em filtragem colaborativa a esparsidade dos dados. Como a maioria dos clientes interage  com uma pequena quantidade  de itens, a matriz de avaliações tem em geral menos de 1\%  dos valores preenchidos, e o sistema deve prever os outros valores \cite{fennell2009collaborative}. 

Outro desafio científico é referente à diversidade das recomendações realizadas, também chamado de excesso de especialização (\textit{over-specialization}) \cite{adomavicius2005toward}. Ao mesmo tempo que o sistema deve apresentar itens similares ao que o usuário está procurando, ele também deve sugerir itens que o usuário desconheça ou que nem saiba que poderiam interessá-lo. Esse problema afeta principalmente os sistemas baseados em conteúdo, pois itens com características similares tendem a ser sempre recomendados. Na Tabela \ref{tab:aif_overspec}, o item $i_2$ seria sugerido para um usuário que tenha avaliado $i_1$, apesar de esse não apresentar nenhuma característica diferente do item previamente comprado. Para contornar essa dificuldade, costuma-se introduzir elementos de aleatoriedade na recomendação, por exemplo a partir de algoritmos genéticos \cite{balabanovic97fab}. 

\begin{table}[h]
\begin{center}
    \caption{Atributos $a_{if}$}
    \label{tab:aif_overspec}
    \begin{tabular}{ | c | c | c | c | } 
    \hline
     & $f_1$ & $f_2$ & $f_3$ \\ \hline
     $i_1$ & 1 & 50 & 0.8 \\ \hline
     $i_2$ & 1 & 50 & 0.8 \\ \hline
     $i_3$ & 0 & 75 & 0.3 \\ \hline
    \end{tabular}
\end{center}
\end{table}

Além desse desafio, existe também o da análise rasa do conteúdo (\textit{shallow content analysis}). O sistema, ao avaliar a característica dos itens, não consegue extrair importantes aspectos para o usuário caso eles não estejam explicitamente descritos na categorização do banco de dados. Isso pode ocorrer, por exemplo, com fatores externos ao produto que influenciem na compra do usuário, como em datas comemorativas, em compras induzidas por propaganda, em compras ``impulsivas'', etc. Se um usuário comprou um arranjo de flores no dia das mães, não é necessariamente verdade que ele se interessa por flores. Da mesma maneira, sistemas que ignorem a sazonalidade de certos produtos não recomendariam arranjos de flores para clientes que não tenham comprado itens parecidos, mesmo no dia das mães.

Por fim, um desafio científico que este trabalho enfrentará é a execução de um sistema híbrido do ponto de vista de efemeridade e persistência, ao construir um modelo de recomendação que integre as preferências de curto e longo termo dos usuários \cite{schafer1999recommender}. A análise dos dados de compras anteriores, bem como de dados demográficos, deverá portanto ser incorporada à análise de característica dos produtos, a fim de enriquecer a acurácia do sistema \cite{wei2007survey}.

Esse tópico de pesquisa inclui ainda diversos desafios científicos e tecnológicos que não serão tratados no nosso projeto, tais como a preservação da privacidade dos usuários, a criação de modelos de recomendação inter-domínios, o desenvolvimento de sistemas descentralizados operando em redes computacionais distribuídas, a otimização de sistemas para sequências de recomendações, a otimização de sistemas para dispositivos móveis e outros. Um sistema de recomendação inteligente também deveria prever quando enviar uma determinada recomendação, e não agir apenas mediante requisição dos clientes \cite{lops2011content}.

% section desafios_cient_ficos_e_tecnol_gicos (end)

\section{Soluções propostas} % (fold)
\label{sec:solu_es_propostas}

Este Trabalho de Conclusão de Curso aborda três propostas de solução para o problema da recomendação, sendo duas delas retiradas de referências bibliográficas \cite{symeonidis2007feature,debnath2008feature}, e uma outra apresentada pela dupla. O objetivo é realizar uma análise comparativa entre cada um dos métodos e estabelecer diretrizes para sua aplicação em e-commerces. Os algoritmos propostos estão descritos com maior detalhe no Capítulo \ref{chap:sintese_de_solucoes}.

Todas as soluções são algoritmos híbridos, por utilizarem na recomendação tanto a matriz de avaliações $\mathbf{R}$ quanto a matriz de atributos $\mathbf{A}$. Optou-se por dar importância aos algoritmos híbridos em razão de os e-commerces estruturarem seus bancos de dados em torno da descrição dos itens à venda. De modo geral, as tabelas de itens possuem dezenas de atributos, dependendo do ramo de negócios da loja, e pouco detalhe é dado à interação entre o grupo de usuários e itens. A tabela de histórico de compras se limita a informações como data e método de pagamento, e detém pouca informação adicional que possa ser utilizada na recomendação de produtos. Dessa forma supusemos que métodos puramente colaborativos, fundamentados na avaliação dos itens por parte dos usuários, teriam pior desempenho que métodos baseados em conteúdo, que exploram as características dos itens na recomendação. 

A solução FW determina a similaridade de dois itens a partir de medidas de distância para cada um dos atributos dos itens, ponderadas por pesos determinados na regressão linear de uma equação descrita pelo interesse dos usuários em cada \textit{feature}.

O método UP parte do princípio que os usuários estão interessados nos atributos dos itens, e traça correlações entre esses dois elementos para obter pesos que servirão de base para o cálculo da similaridade inter-usuários, utilizada na recomendação pelo método da vizinhança (\textit{nearest neighbors}). 

A variante UI, elaborada pela dupla, recomenda o melhor item a partir das matrizes de correlação usuário-atributo e atributo-item, a fim de obter a matriz de correlação usuário-item. De antemão espera-se que essa solução tenha desempenho similar ao método de base UP, pois ambos buscam explorar as características dos itens para determinar a preferência do usuário.

% section solu_es_propostas (end)