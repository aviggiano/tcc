%!TEX root = index.tex
\chapter{Conclusão}
%\chapter{Trabalhos Futuros}
\label{cha:trabalhos_futuros}

\section{Discussão} % (fold)
\label{sec:discuss_o}

% section discuss_o (end)
Este Trabalho de Conclusão de Curso cumpriu seus objetivos e antedeu aos requisitos estabelecidos no início do projeto. Foi elaborada uma biblioteca para sistemas de recomendação de produtos de e-commerces e foi estabelecida uma respectiva análise de desempenho dos algoritmos de recomendação.

A avaliação de desempenho dos métodos propostos na biblioteca deste trabalho verificaram resultados já conhecidos no meio acadêmico. Em particular, a dependência entre qualidade de recomendação e tamanho da lista de sugestões se verificou (impacto de $N$). 

Além disso, mostramos que um banco de dados com maior quantidade de avaliações (impacto de $H$) tem mais relevância que um banco de dados com mais usuários (impacto de $T$). 

Outro resultado do trabalho foi a comprovação do fenômeno de \textit{hidden feedback} (impacto de $M$). Mesmo que construamos métodos embasados na ``avaliação positiva'' dos usuários, esse parâmetro pode não ter tanta influência, visto que a maioria das avaliações dos clientes já são de fato positivas.

Também foi verificada a influência da quantidade de vizinhos mais próximos em algoritmos que usam essa metodologia colaborativa (impacto de $k$). Apesar de influenciar na qualidade da recomendação, esse parâmetro desempenha papel secundário.

Uma outra conclusão importante deste trabalho foi da importância de se escolher \textit{a priori} o conjunto de atributos dos itens (impacto de $\mathcal{F}$). A categorização excessiva dos itens pode ser maléfica para a recomendação, caso as \textit{features} não tenham relevância para os usuários.

Avaliamos também diferentes medidas de distância entre os atributos (impacto de $d_{ij}^f$). A medida da diferença em valor absoluto foi comparada com outros índices, como o índice Jaccard, para uma lista de gêneros, e verificou-se que a distância $L_1$ resulta em melhor qualidade de recomendação. Vale ressaltar a importância da escolha das medidas de distância, visto seu impacto no desempenho dos algoritmos, sobretudo nos algoritmos UI e FW.

Por fim, avaliamos também a quantidade de pesos dos atributos no método FW (impacto de $W$). Vimos que a quantidade de $w_f>0$ não tem grande impacto na recomendação, visto que o valor dos pesos, em si, já é suficiente para alterar a qualidade das sugestões.

Apenas o método UP atingiu os requisitos funcionais em termos de precisão e abrangência, para uma combinação específica de parâmetros, como valores pequenos de $N$. Tanto para esse algoritmo quanto para o método FW, o desempenho é sensivelmente inferior ao relatado nos artigos de referência. O motivo por trás disso é a dissimilaridade entre os bancos de dados. Assim como foi confirmado, o emprego de bases com mais recomendações $r_{ui}$ influencia grandemente na qualidade das recomendações. Para se obter um \textit{benchmarking} mais fiel, seria necessário utilizar o banco de dados dos autores de referência. 

Apesar de, por definição, o método UI ser similar ao método UP, seu desempenho foi sensivelmente inferior ao do algoritmo-base. Isso se deve fundamentalmente a dois motivos: o primeiro é devido ao fato de o algoritmo UP possuir uma etapa de correlação entre vizinhos mais próximos, que cumpre com eficácia o papel de selecionar apenas as melhores recomendações. O segundo é devido à má escolha da função que calcula a correlação usuário-item. A simples multiplicação da correlação usuário-atributo $w_{uf}$ pelo valor numérico do atributo do item $a_{if}$ não tem ligação direta com a qualidade de um item para um determinado usuário. 

Conforme foi exemplificado no Capítulo \ref{chap:resultados}, se um usuário gosta filmes de época (elevado $w_{uf}$), mas apenas de filmes antigos, a multiplicação $w_{uf} \cdot a_{if} $ apontaria que ele se interessa por filmes \textit{atuais} (elevado $a_{if}$). Faz-se necessário, portanto, substituir a multiplicação direta entre $\mathbf{W}$ e $\mathbf{A}$ por uma expressão que correlacione o valor numérico do atributo com o interesse do usuário pela \textit{feature}, seja $\mathbf{W}$ e $g\left(\mathbf{A}\right)$, sendo a função $g$ a se determinar. 

Mesmo com essa falha de concepção, o algoritmo UP mostrou desempenho similar ao algoritmo FW, superando-o inclusive em tempo de processamento. 

\section{Trabalhos futuros} % (fold)
\label{sec:trabalhos_futuros}

A extensão desse Trabalho de Conclusão de Curso pode se dar de diversas maneiras, tanto na área acadêmica quanto na área empresarial. Seguindo o atual encaminhamento do projeto, a principal oportunidade do nosso trabalho é a criação de um serviço de um ``Sistema de Recomendação nas Nuvens''. 

Desejamos eliminar as restrições quanto a entrada e saída de dados, de forma que elas fossem completamente arbitrárias. O objetivo é que o usuário possa informar ao sistema como é formado sua base, e que todo o tratamento preliminar seja feito automaticamente. 

É possível explorar também a construção de um \textit{driver} que possibilite a conexão entre o sistema de recomendação e um banco de dados SQL, sem que seja necessária a etapa intermediária de arquivos \texttt{csv} para aquisição de dados. Em seguida, é importante elaborar um \textit{website} para o sistema de recomendação e exportar toda a lógica para um servidor dedicado. 

Outra melhoria desejada é a melhoria dos métodos e funções, a fim de aprimorar a performance computacional. Dessa forma, o serviço de ``sistema de recomendação nas nuvens'' estaria completo e poderia ser utilizado por e-commerces reais.

Também seria desejável, para uma avaliação mais completa do trabalho, o emprego dos métodos computacionais em um banco de dados de um e-commerce real. Apesar de termos contatado diversas lojas de comércio online, devido a impedimentos administrativos, não obtivemos sucesso em firmar uma parceria com essas lojas. 

No campo acadêmico, há muito espaço para melhorias nos algoritmos de recomendação. Conforme mostrado, a quantidade de atributos, seus pesos e suas medidas de distância tem grande influência na qualidade da recomendação. Seria interessante, portanto, explorar diferentes estratégias de determinação dessas variáveis para todos os métodos. É possível, por exemplo, utilizar algoritmos genéticos ou redes neurais que explorem combinações de parâmetros e pesos a fim de maximizar a precisão e acurácia para uma determinada base.

Além disso, as metodologias de solução de cada um dos sistemas deveriam ser debatidas ao máximo, de modo a explorar casos de uso particulares e a propor mudanças e otimizações. Faz-se necessário responder a perguntas como ``O que acontece com itens ou usuários sem nenhuma avaliação?'' e ``Qual o desempenho dos métodos para outros bancos de dados?''. 