%!TEX root = index.tex
\chapter{Conclusão}
%\chapter{Trabalhos Futuros}
\label{cha:trabalhos_futuros}

\section{Discussão} % (fold)
\label{sec:discuss_o}

% section discuss_o (end)
Este Trabalho de Conclusão de Curso cumpriu seus objetivos e antedeu aos requisitos estabelecidos no início do projeto. Foi elaborada uma biblioteca para sistemas de recomendação de produtos de e-commerces e foi estabelecida uma respectiva análise de desempenho dos algoritmos de recomendação.

A avaliação de desempenho dos métodos propostos na biblioteca deste trabalho verificaram resultados já conhecidos no meio acadêmico. Em particular, a dependência entre qualidade de recomendação e tamanho da lista de sugestões se verificou (impacto de $N$). 

Além disso, mostramos que um banco de dados com maior quantidade de avaliações (impacto de $H$) tem mais relevância que um banco de dados com mais usuários (impacto de $T$). 

Outro resultado do trabalho foi a comprovação do fenômeno de \textit{hidden feedback} (impacto de $M$). Mesmo que construamos métodos embasados na ``avaliação positiva'' dos usuários, esse parâmetro pode não ter tanta influência, visto que a maioria das avaliações dos clientes já são de fato positivas.

Também foi verificada a influência da quantidade de vizinhos mais próximos em algoritmos que usam essa metodologia colaborativa (impacto de $k$). Apesar de influenciar na qualidade da recomendação, esse parâmetro desempenha papel secundário.

Uma outra conclusão importante deste trabalho foi da importância de se escolher \textit{a priori} o conjunto de atributos dos itens (impacto de $\mathcal{F}$). A categorização excessiva dos itens pode ser maléfica para a recomendação, caso as \textit{features} não tenham relevância para os usuários.

Avaliamos também diferentes medidas de distância entre os atributos (impacto de $d_{ij}^f$). A medida da diferença em valor absoluto foi comparada com outros índices, como o índice Jaccard, para uma lista de gêneros, e verificou-se que a distância $L_1$ resulta em melhor qualidade de recomendação. Vale ressaltar a importância da escolha das medidas de distância, visto seu impacto no desempenho do sistema.

Por fim, avaliamos também a quantidade de pesos dos atributos no método FW (impacto de $W$). Vimos que a quantidade de $w_f>0$ não tem grande impacto na recomendação, visto que o valor dos pesos, em si, já é suficiente para alterar a qualidade das sugestões.

\section{Trabalhos futuros} % (fold)
\label{sec:trabalhos_futuros}

A extensão desse Trabalho de Conclusão de Curso pode se dar de diversas maneiras, tanto na área acadêmica quanto na área empresarial. Seguindo o atual encaminhamento do projeto, a principal oportunidade do nosso trabalho é a criação de um serviço de um ``Sistema de Recomendação nas Nuvens''. 

Desejamos eliminar as restrições quanto a entrada e saída de dados, de forma que elas fossem completamente arbitrárias. O objetivo é que o usuário possa informar ao sistema como é formado sua base, e que todo o tratamento preliminar seja feito automaticamente. 

É possível explorar também a construção de um \textit{driver} que possibilite a conexão entre o sistema de recomendação e um banco de dados SQL, sem que seja necessária a etapa intermediária de arquivos \texttt{csv} para aquisição de dados. Em seguida, é importante elaborar um \textit{website} para o sistema de recomendação e exportar toda a lógica para um servidor dedicado. 

Outra melhoria desejada é a reconstrução dos métodos na linguagem de programação C, a fim de melhorar a performance computacional. Dessa forma, o serviço de ``sistema de recomendação nas nuvens'' estaria completo e poderia ser utilizado por e-commerces reais.

No campo acadêmico, há muito espaço para melhorias nos algoritmos de recomendação. As metodologias de solução de cada um dos sistemas deveriam ser debatidas ao máximo, de modo a explorar casos de uso particulares e a propor mudanças e otimizações. Faz-se necessário responder a perguntas como ``O que acontece com itens ou usuários sem nenhuma avaliação?'' e ``Qual o desempenho dos métodos para outros bancos de dados?''. 