%!TEX root = index.tex
\chapter{Conclusão}
%\chapter{Trabalhos Futuros}
\label{cha:trabalhos_futuros}

A avaliação de desempenho dos métodos propostos na biblioteca deste Trabalho de Conclusão de Curso verificaram resultados já conhecidos no meio acadêmico.

Em particular, a dependência entre qualidade de recomendação e tamanho da lista de sugestões se verificou (impacto de $N$). Além disso, mostramos que um banco de dados com maior quantidade de avaliações (impacto de $H$) tem mais relevância que um banco de dados com mais usuários (impacto de $T$). 

Outro resultado do trabalho foi a comprovação do fenômeno de \textit{hidden feedback} (impacto de $M$). Mesmo que construamos métodos embasados na ``avaliação positiva'' dos usuários, esse parâmetro pode não ter tanta influência, visto que a maioria das avaliações dos clientes já são de fato positivas.

Para os trabalhos futuros, iremos realizar a validação cruzada e avaliar se os requisitos funcionais foram estabelecidos. Em seguida, procuraremos melhorar o sistema de recomendação a fim de torná-lo mais genérico. Buscaremos eliminar restrições quanto a entrada e saída de dados, de forma que elas sejam completamente arbitrárias. O objetivo é que o usuário possa informar ao sistema como é formado sua base, e que todo o tratamento preliminar seja feito automaticamente. 

Caso haja tempo, trabalharemos também na construção de um \textit{driver} que possibilite a conexão entre o sistema de recomendação e um banco de dados SQL, sem que seja necessária a etapa intermediária de arquivos \texttt{csv} para aquisição de dados. Planejamos elaborar um \textit{website} para o sistema de recomendação e exportar toda a lógica para um servidor dedicado. Outra melhoria desejada é a reconstrução dos métodos na linguagem de programação C, a fim de melhorar a performance computacional. Dessa forma, o serviço de ``sistema de recomendação nas nuvens'' estaria completo e poderia ser utilizado por e-commerces reais.