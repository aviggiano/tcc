%!TEX root = index.tex
\chapter{Conclusão}
%\chapter{Trabalhos Futuros}
\label{cha:trabalhos_futuros}

TODO arrumar

Para os trabalhos futuros, iremos realizar a validação cruzada e avaliar se os requisitos funcionais foram estabelecidos. Em seguida, procuraremos melhorar o sistema de recomendação a fim de torná-lo mais genérico. Buscaremos eliminar restrições quanto a entrada e saída de dados, de forma que elas sejam completamente arbitrárias. O objetivo é que o usuário possa informar ao sistema como é formado sua base, e que todo o tratamento preliminar seja feito automaticamente. 

Caso haja tempo, trabalharemos também na construção de um \textit{driver} que possibilite a conexão entre o sistema de recomendação e um banco de dados SQL, sem que seja necessária a etapa intermediária de arquivos \texttt{csv} para aquisição de dados. Planejamos elaborar um \textit{website} para o sistema de recomendação e exportar toda a lógica para um servidor dedicado. Outra melhoria desejada é a reconstrução dos métodos na linguagem de programação C, a fim de melhorar a performance computacional. Dessa forma, o serviço de ``sistema de recomendação nas nuvens'' estaria completo e poderia ser utilizado por e-commerces reais.