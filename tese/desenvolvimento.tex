%!TEX root = index.tex
\chapter{Desenvolvimento da biblioteca} % (fold)
\label{cha:desenvolvimento_da_biblioteca}

\section{Recursos acadêmicos} % (fold)
\label{sec:recursos_acad_micos}

Diversos recursos extra-curriculares foram de fundamental importância para o sucesso deste trabalho. Foram aplicados ensinamentos práticos de quatro cursos da plataforma online de \textit{e-learning} Coursera (\url{https://www.coursera.org/}), seja relacionados a teoria dos sistemas de recomendação, seja relacionados a configuração de ambientes de testes em máquinas virtuais dos servidores da Amazon Web Services.

O curso ``Redes: Amigos, Dinheiro e Bytes'' (Networks: Friends, Money, and Bytes -- \url{https://www.coursera.org/course/friendsmoneybytes}), teve papel importante na introdução a temas ligados à rede mundial de computadores. Mais especificadamente, a aula 4 aborda, de maneira simples mas repleta de exemplos, a temática de sugestão de itens através da pergunta ``Como o Netflix recomenda filmes?''. Essa aula ajudou-nos a compreender a teoria por trás do algoritmo de recomendação do Netflix detalhado na Referência \citeonline{lops2011content-chap5}.


Outro curso que influenciou diretamente o nosso Trabalho de Conclusão de Curso foi ``Computação para Análise de Dados'' (Computing for Data Analysis -- \url{https://www.coursera.org/course/compdata}). As quatro semanas de aula ensinaram a leitura de dados formatados em R, o tratamento de dados, a definição de métodos estatísticos, como por exemplo de regressão linear, a aplicação de cálculos vetorizados e a construção de gráficos e tabelas. Aliado a essas aulas, aprendemos também o paradigma funcional, amplamente utilizado em R, durante as sete semanas de ``Princípios de Programação Funcional em Scala'' (Functional Programming Principles in Scala -- \url{https://www.coursera.org/course/progfun}) 

Startup Engineering (\url{https://www.coursera.org/course/startup})

\section{Ferramentas utilizadas} % (fold)
\label{sec:ferramentas_utilizadas}

O desenvolvimento da biblioteca computacional se deu por meio do ambiente de desenvolvimento integrado RStudio versão 0.98.953 (\url{http://www.rstudio.com/}). Esse IDE inclui um console, um editor de texto e um corretor de sintaxe que suporta a execução de código direta, bem como ferramentas para traçar gráfico, histórico de comandos, depuração de erros e gerenciamento de espaço de trabalho. Além disso, o RStudio está disponível via licença de código aberto AGPLv3 para os principais sistemas operacionais (Windows, Mac e Linux).

\section{Métodos computacionais} % (fold)
\label{sec:m_todos_computacionais}

% section m_todos_computacionais (end)

\section{Ambiente de testes} % (fold)
\label{sec:ambiente_de_testes}

% section ambiente_de_testes (end)