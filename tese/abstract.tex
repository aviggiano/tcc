%!TEX root = index.tex
% resumo em português
\setlength{\absparsep}{18pt} % ajusta o espaçamento dos parágrafos do resumo
\begin{resumo}[Abstract]
 \begin{otherlanguage*}{english}

This project's scope is to design and assess a recommender system algorithm and library for e-commerces. The goal of this library is to make the implementation of a generic recommender system simple and easy, so it can be used by the academics and e-commerces willing to automate the suggestion of items, such as in email marketing.

The library was developed using three different recommendation algorithms. The feature weighted is a hybrid method, based on collaborative filtering and content-based filtering, in which a linear regression is calculated from a social-network database, extracting the weights that determine each attribute's importance. The second method, based in user profiles, considers the users' interests in specific features, indirectly calculated by the users' interest in different items. The third method, based in the user-item correlation, is derived from the method based in users' profiles and was developed by the authors. This method searches for them items with the features that are more similar to the attributes that the user has shown interest for.

The comparative assessment of the methods has shown the superiority of the user-profile algorithm in almost all aspects, and has measured the main parameters that affect the recommendation quality. From the empirical results shown in this work, it is possible to establish some guidelines on how to create a recommender system based on the library developed by the authors.

   \vspace{\onelineskip}
 
   \noindent 
   \textbf{Key-words}: Artificial intelligence, Machine learning, e-Commerce, Products.
 \end{otherlanguage*}
\end{resumo}
