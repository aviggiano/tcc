%!TEX root = index.tex
% resumo em português
\setlength{\absparsep}{18pt} % ajusta o espaçamento dos parágrafos do resumo
\begin{resumo}
O objetivo deste trabalho é projetar e avaliar o desempenho de um algoritmo de recomendação e uma biblioteca computacional para sistemas de sugestão de produtos de lojas de comércio online. Essa biblioteca tem finalidade de permitir a fácil implementação de um sistema de recomendação genérico para ser utilizado por acadêmicos e \textit{e-commerces} que desejem automatizar o processo de sugestão de itens, tal como em \textit{email marketing}. 

A biblioteca foi desenvolvida utilizando-se três diferentes algoritmos de recomendação. O algoritmo baseado na ponderação de atributos, que trata-se de um método híbrido entre filtragem colaborativa e filtragem baseada em conteúdo, onde a partir da regressão linear de dados de uma rede social, extrai os pesos que determinam a importância de cada atributo dos itens. O segundo método, baseado em perfil de usuários, leva em consideração o interesse dos usuários por \textit{features}, indiretamente calculado a partir de seu interesse pelos itens. O terceiro método, baseado na correlação usuário-item, é uma variante do método baseado no perfil de usuários e foi desenvolvido pela dupla. Este método busca os itens com \textit{features} mais similares aos atributos pelos quais o usuário se interessa.

A avaliação comparativa dos métodos mostrou a superioridade do algoritmo de perfil de usuários em quase todos os aspectos, e avaliou os principais parâmetros de influência na qualidade da recomendação. A partir dos resultados empíricos mostrados neste trabalho, é possível estabelecer diretrizes para a elaboração de um sistema de recomendação próprio com base na biblioteca elaborada pela dupla.

\textbf{Palavras-chaves}: Inteligência artificial, Aprendizado computacional, Comercio eletrônico, Produtos.
\end{resumo}