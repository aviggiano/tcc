%!TEX root = index.tex
% resumo em português
\setlength{\absparsep}{18pt} % ajusta o espaçamento dos parágrafos do resumo
\begin{resumo}
O objetivo deste trabalho é projetar e implementar uma biblioteca computacional para sistemas de recomendaçãos de produto de lojas de comércio, com a finalidade de permitir a fácil implementação de um sistema de recomendação genérico para ser utilizado na geração de email marketing. Para alcançar esse objetivo foi necessário um estudo sobre os principais métodos de cálculo de recomendações, desde as definições dos relacionamentos entre usuário e item às comparações entre itens e seus diversos atributos.

A biblioteca foi desenvolvida utilizando três diferentes algoritmos de recomendação. O algoritmo baseado na ponderação de atributos, que trata-se de um método híbrido entre filtragem colaborativa e filtragem baseada em conteúdo, onde a partir da
regressão linear de dados de uma rede social, extraem-se os pesos que determinam a importância de cada atributo dos itens. O segundo método, baseado em no perfil de usuários, leva em consideração o interesse dos usuários por \textit{features}, indiretamente calculado a partir de seu interesse pelos itens. O terceiro método, baseado na correlação usuário-item, é uma variante do método baseado no perfil de usuários. Este método busca os itens com \textit{features} mais similares aos atributos pelos quais o usuário se interessa, através dos atributos.


 \textbf{Palavras-chaves}: Inteligência artificial, Aprendizado computacional, Comercio eletrônico, Produtos
\end{resumo}