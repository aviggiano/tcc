%!TEX root = index.tex
\chapter[Metodologia]{Metodologia}
\label{chap:metodologia}

A metodologia de projeto deste Trabalho de Conclusão de Curso foi fundamentada principalmente na Referência \citeonline{pmbok}. Por se tratar de um projeto de Engenharia de Software, foi necessário dar ênfase às etapas iterativas de desenvolvimento dos algoritmos. Esse processo cíclico, com fases de especificação, desenvolvimento e validação, permitiu obter resultados preliminares e os modificar os algoritmos ao longo da disciplina, ajustando detalhes e melhorando o sistema gradativamente \cite{iterative-development}.

A metodologia de execução do projeto, assim como a de avaliação dos resultados, pode ser consolidada da seguinte maneira: 

\section{Definição da Necessidade} % (fold)
\label{sec:defini_o_da_necessidade}

% section defini_o_da_necessidade (end)

Com o crescente número de lojas de comércio online, tornou-se necessário a criação de sistemas que pudessem entender e prever o comportamento de consumidores, a fim de oferecer produtos específicos para cada um deles, aumentando o número de vendas e a satisfação do cliente. Observa-se atualmente que o número de sistemas de recomendação gratuitos, de fácil integração e de código aberto (\textit{open source}) são limitados e não correspondem às necessidades do mercado. Existe, pois, a necessidade da criação de uma biblioteca que possa ser utilizada por e-commerces que desejem estabelecer seu próprio sistema de recomendação ou mesmo por indivíduos interessados na temática da recomendação de itens.

\section{Definição dos Parâmetros de Sucesso} % (fold)
\label{sec:defini_o_dos_par_metros_de_sucesso}

% section defini_o_dos_par_metros_de_sucesso (end)

O sucesso do projeto pode ser medido em duas frentes: a primeira, quantitativa, mede a precisão e a abrangência das recomendações; a segunda, qualitativa, avalia se o sistema responde bem aos problemas recorrentes desse tópico de pesquisa, tais como a escalabilidade, o excesso de especialização e outros. 

\section{Síntese de Soluções} % (fold)
\label{sec:s_ntese_de_solu_es}

% section s_ntese_de_solu_es (end)

Nesta fase do projeto, foram propostas possíveis soluções para o desafio da recomendação. Decidiu-se avaliar dois métodos híbridos do meio acadêmico e um outro elaborado pela dupla. 

\section{Detalhamento da Solução} % (fold)
\label{sec:detalhamento_da_solu_o}

% section detalhamento_da_solu_o (end)

Após a escolha dos métodos de recomendação, as soluções foram detalhadas matematicamente segundo uma mesma notação, e a estrutura dos algoritmos foi descrita e exemplificada. Neste ponto, escolheu-se também a linguagem de programação R e a forma de entrada e saída de dados, por meio de arquivos \texttt{.csv}.

A fim de facilitar o pré-processamento dos dados, estabelecemos que seriam necessários dois arquivos. Um deles deve conter a matriz de atributos $\mathbf{A}$ e o outro, a matriz de avaliações  $\mathbf{R}$. 

\begin{equation} 
\mathbf{A} = 
\begin{bmatrix} 
 a_{i_1 f_1} &  a_{i_1 f_2} &  a_{i_1 f_3}  & \dots   \\
 a_{i_2 f_1} &  a_{i_2 f_2} &  a_{i_2 f_3}  & \dots   \\
 a_{i_3 f_1} &  a_{i_3 f_2} &  a_{i_3 f_3}  & \dots  \\ 
 \vdots &  \vdots &  \vdots  & \ddots   \\
 \end{bmatrix}
\end{equation}


\begin{equation}
	  \mathbf{R} = 
\begin{bmatrix} 
  r_{u_1 i_1} &  r_{u_1 i_2} &  r_{u_1 i_3}  & \dots   \\
 r_{u_2 i_1} &  r_{u_2 i_2} &  r_{u_2 i_3}  & \dots   \\
 r_{u_3 i_1} &  r_{u_3 i_2} &  r_{u_3 i_3}  & \dots  \\ 
 \vdots &  \vdots &  \vdots  & \ddots   \\
\end{bmatrix}
\end{equation}

\section{Estruturação do Banco de Dados} % (fold)
\label{sec:modelamento_e_simula_o}

% section modelamento_e_simula_o (end)


Uma vez determinada a forma de entrada de informações, definiram-se os conjuntos de dados a serem utilizados. 

O primeiro conjunto de dados abertos é proveniente do sistema de recomendações de filmes MovieLens (\url{http://movielens.umn.edu}), e é composto de 100 000 avaliações (valores inteiros de 1 a 5) de 943 usuários para 1682 filmes \cite{movielensdataset}. Além disso, cada usuário (idade, sexo, profissão, logradouro) avaliou pelo menos 20 filmes (categoria, ano de publicação). Nessa base de dados, chamada de 100k, o catálogo de filme faz o papel de catálogo de produtos, e o histórico de compras se refere à avaliação dos filmes feita por cada usuário. 

O segundo banco de dados é extraído do Internet Movie Database (IMDB), e possui 28 819 filmes. Esse banco está presente na biblioteca \texttt{ggplot2} da linguagem de programação R \cite{moviesggplot2dataset}.

Na nossa análise, os bancos de dados 100k e IMDB foram utilizados complementarmente. A união desses dois conjuntos deu origem à base 100k-IMDB, composta por 943 usuários, 1682 itens e 25 atributos. Na biblioteca proposta pela dupla, os dados demográficos de usuários não são utilizados. 

%Os métodos escolhidos foram codificados em R e testados com inicialmente com o banco de dados 100k. Posteriormente, testamos os algoritmos no banco IMDB, a fim de avaliar a qualidade das recomendações mediante a mudanças na base de dados.

Ainda na etapa de implementação, confirmamos a validade de cada um dos métodos aplicando-os nas matrizes-referência (Tabelas \ref{tab:rui_ref} e \ref{tab:aif_ref}). 

\section{Validação Cruzada} % (fold)
\label{sec:prot_tipos_testes}

A fim de realizar um estudo comparativo (\textit{benchmarking}) com os artigos de referência, mantivemos a mesma metodologia de avaliação de qualidade do artigo \citeonline{symeonidis2007feature}.

Em particular, implementamos uma validação cruzada considerando $T=75\%$ do banco de dados como base de treinamento ou aprendizado e os $25\%$ restantes como base de testes. Em seguida, mascaramos $H=75\%$ das avaliações dos usuários-teste, de modo a medir a qualidade do sistema de recomendação em prever os itens positivamente avaliados. Cerca de uma dezena de parâmetros de interesse foram avaliados para cada um dos métodos (Tabela \ref{tab:variaveis}). 

Além disso, não fizemos distinção entre valores não observados (\textit{NA value/NULL value}) e avaliações nulas ($r_{ui}=0$), pois na maioria dos casos essa simplificação é válida. Esse não é o caso, por exemplo, de sistemas em que o usuário pode deliberadamente dar  nota zero para um item.

Sabe-se que a extração de um modelo por meio de uma validação cruzada sobre uma mesma base de dados pode gerar \textit{overfitting} \cite{ng1997preventing}. Para não cair nesse erro e com foco na reprodutibilidade do trabalho, realizamos todas as amostragens em R utilizando o número 2 como semente aleatória (\textit{state seed}). Dessa forma, os parâmetros calculados para os modelos são sempre os mesmos para qualquer teste de qualidade. Evidentemente, caso se deseje avaliar a performance dos métodos para um outro banco de dados, uma validação cruzada rigorosa deverá ser aplicada. 

% utilizamos sempre a mesma base de treinamento para aprendizado do modelo. sendo um deles (100k-IMDB) para avaliação da qualidade das recomendações mediante a mudanças em parâmetros do problema e o outro para avaliação dos algoritmos em uma base totalmente diferente, sem modelagem \textit{a priori}.


Como a complexidade dos algoritmos excede o limite dos computadores pessoais da dupla, foi necessário contratar o serviço de computação nas nuvens Amazon Web Services.

Alugamos duas máquinas virtuais do tipo \texttt{r3.large}, otimizadas para memória. As máquinas, de especificação 2 vCPU, 15 GB de memória RAM e sistema operacional Amazon Linux AMI release 2014.09 x86\_64, baseado em RHEL Fedora, custaram USD 0,175 por hora de uso. Todos os testes foram realizados em aproximadamente 12 horas, custando apenas USD 4,20 (menos de R\$ 12,00). Uma explicação detalhada da configuração do ambiente de testes se encontra na Seção \ref{sec:ambiente_de_testes}.