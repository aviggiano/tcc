%!TEX root = index.tex
\chapter[Metodologia]{Metodologia}
\label{chap:metodologia}

A metodologia de projeto deste Trabalho de Conclusão de Curso foi fundamentada principalmente na Referência \citeonline{pmbok}. Por se tratar de um projeto de Engenharia de Software, foi necessário dar ênfase às etapas iterativas de desenvolvimento dos algoritmos. Esse processo cíclico, com fases de especificação, desenvolvimento e validação, permitiu obter resultados preliminares e os modificar os algoritmos ao longo da disciplina, ajustando detalhes e melhorando o sistema gradativamente \cite{iterative-development}.

A metodologia de execução do projeto, assim como a de avaliação dos resultados, pode ser consolidada da seguinte maneira: 

\section{Definição da Necessidade} % (fold)
\label{sec:defini_o_da_necessidade}

% section defini_o_da_necessidade (end)

Com o crescente número de lojas de comércio online, tornou-se necessário a criação de sistemas que pudessem entender e prever o comportamento de consumidores, a fim de oferecer produtos específicos para cada um deles, aumentando o número de vendas e a satisfação do cliente. Observa-se atualmente que o número de sistemas de recomendação gratuitos, de fácil integração e de código aberto (\textit{open source}) são limitados e não correspondem às necessidades do mercado. Existe, pois, a necessidade da criação de um sistema que possa ser utilizado por e-commerces que desejem estabelecer seu próprio sistema de recomendação ou mesmo por indivíduos interessados na temática da recomendação de itens.

\section{Definição dos Parâmetros de Sucesso} % (fold)
\label{sec:defini_o_dos_par_metros_de_sucesso}

% section defini_o_dos_par_metros_de_sucesso (end)

O sucesso do projeto pode ser medido em duas frentes: a primeira, quantitativa, mede a precisão e a abrangência das recomendações; a segunda, qualitativa, avalia se o sistema responde bem aos problemas recorrentes desse tópico de pesquisa, tais como a escalabilidade, o excesso de especialização e outros. 

\section{Síntese de Soluções} % (fold)
\label{sec:s_ntese_de_solu_es}

% section s_ntese_de_solu_es (end)

Nesta fase do projeto, foram propostas possíveis soluções para o desafio da recomendação. Decidiu-se avaliar dois métodos híbridos do meio acadêmico e um outro elaborado pela dupla. 

\section{Detalhamento da Solução} % (fold)
\label{sec:detalhamento_da_solu_o}

% section detalhamento_da_solu_o (end)

Após a escolha dos métodos de recomendação, as soluções foram detalhadas matematicamente segundo uma mesma notação, e a estrutura dos algoritmos foi descrita e exemplificada. Neste ponto, escolheu-se também a linguagem de programação (R) e a forma de entrada e saída de dados (arquivos \texttt{.csv}).

\section{Modelamento e Simulação} % (fold)
\label{sec:modelamento_e_simula_o}

% section modelamento_e_simula_o (end)

Os métodos escolhidos foram codificados em R e testados com inicialmente com o banco de dados 100k. Posteriormente, testamos os algoritmos no banco IMDB, a fim de avaliar a qualidade das recomendações mediante a mudanças na base de dados.

\section{Validação Cruzada} % (fold)
\label{sec:prot_tipos_testes}

A fim de realizar um estudo comparativo (\textit{benchmarking}) com os artigos de referência, mantivemos a mesma metodologia de avaliação de qualidade do artigo \citeonline{symeonidis2007feature}.

Em particular, implementamos uma validação cruzada considerando $T=75\%$ do banco de dados como base de treinamento ou aprendizado e os $25\%$ restantes como usuários-teste. Em seguida, foram mascarados $H=75\%$ das avaliações dos usuários-teste, de modo a medir a qualidade do sistema de recomendação em prever os itens que haviam sido positivamente avaliados. 




reescrever

A realização de testes será feita com os bancos de dados de centenas de milhares de itens ou de avaliações. Visto que será feita uma validação cruzada, será necessário descartar os dados e reformular a solução caso as recomendações não atinjam os requisitos funcionais. Isso evita que o sistema seja moldado para operar somente com aquele banco de dados específico.






NAME="Amazon Linux AMI"
VERSION="2014.09"
ID="amzn"
ID_LIKE="rhel fedora"
VERSION_ID="2014.09"
PRETTY_NAME="Amazon Linux AMI 2014.09"
ANSI_COLOR="0;33"
CPE_NAME="cpe:/o:amazon:linux:2014.09:ga"
HOME_URL="http://aws.amazon.com/amazon-linux-ami/"
Amazon Linux AMI release 2014.09

Linux 3.14.20-20.44.amzn1.x86_64 x86_64
