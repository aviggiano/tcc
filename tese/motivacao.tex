%!TEX root = index.tex
\section{Motivação} % (fold)
\label{cha:motivacao}

Conforme apresentado, a quantidade de lojas de varejo online cresce em ritmo acelerado no Brasil e no mundo. Motivados pela importância econômica dos e-commerces, bem como pela possibilidade de criar um conjunto de ferramentas \textit{open source} que possam ser utilizadas pela comunidade acadêmica e empresarial, propomos como Trabalho de Conclusão de Curso o desenvolvimento de um algoritmo e biblioteca computacional para sistemas de recomendação de produtos de lojas de comércio online.

O pacote computacional é composto de métodos de leitura de dados de histórico de compras e de informações de clientes e produtos, de cálculo de sugestões de itens com base em algoritmos de recomendação e de análise de desempenho das recomendações. Além disso, propusemos um algoritmo baseado em um método já existente, a fim de avaliar a sua qualidade comparado a outras soluções.

A motivação de se criar uma biblioteca de software decorre principalmente da sua abrangência e capacidade de adaptação, visto que é possível atender a mais casos de uso que um sistema de recomendação completo. De um lado, um sistema de recomendação possui uma finalidade específica -- como por exemplo de sugerir notícias para usuários de internet -- e uma entrada e saída de dados específica -- como por exemplo o fato de as notícias sempre estarem ordenada pelas mais recentes em uma tabela de sugestões. De outro lado, uma biblioteca computacional pode receber qualquer tipo de dados e gerar qualquer saída de dados. 

Caso uma empresa ou um acadêmico queira construir seu próprio sistema de recomendação, basta elaborar a conexão entre o pacote apresentado pela dupla, seu banco de dados e a interface gráfica de apresentação de resultados.

As contribuições científica e tecnológica deste trabalho para a Engenharia Mecatrônica estão sobretudo nos campos de inteligência artificial, de sistemas de informação e de automação de processos.

As competências acadêmicas necessárias para a execução desse trabalho envolvem algoritmos e estruturas de dados (abordados em PMR2300 -- Computação para Automação), documentação e modelagem de sistemas computacionais (explicados em PMR2440 -- Programação para Automação), sistemas de informação e banco de dados (tratados em PMR2490 -- Sistemas de Informação) e inteligencia artificial, com enfase em aprendizado de máquina (aprofundados em PMR2728 -- Teoria de Probabilidades em Inteligência Artificial e Robótica). As competências técnicas abrangem programação estatística e funcional, demonstradas através da linguagem R.

%Os principais desafios do projeto constituem o tratamento da grande quantidade de dados, que serão coletados de diversas bases contendo mais de 100 mil avaliações e 25 mil itens; determinação de medidas de similaridade para as recomendações; análise de performance das implementações propostas; escolha da solução definitiva e, por fim, comparação entre os sistemas de recomendação existentes e o que será apresentado pela dupla.


