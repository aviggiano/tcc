%% abtex2-modelo-trabalho-academico.tex, v-1.9.2 laurocesar
%% Copyright 2012-2014 by abnTeX2 group at http://abntex2.googlecode.com/ 
%%
%% This work may be distributed and/or modified under the
%% conditions of the LaTeX Project Public License, either version 1.3
%% of this license or (at your option) any later version.
%% The latest version of this license is in
%%   http://www.latex-project.org/lppl.txt
%% and version 1.3 or later is part of all distributions of LaTeX
%% version 2005/12/01 or later.
%%
%% This work has the LPPL maintenance status `maintained'.
%% 
%% The Current Maintainer of this work is the abnTeX2 team, led
%% by Lauro César Araujo. Further information are available on 
%% http://abntex2.googlecode.com/
%%
%% This work consists of the files abntex2-modelo-trabalho-academico.tex,
%% abntex2-modelo-include-comandos and abntex2-modelo-references.bib
%%

% ------------------------------------------------------------------------
% ------------------------------------------------------------------------
% abnTeX2: Modelo de Trabalho Academico (tese de doutorado, dissertacao de
% mestrado e trabalhos monograficos em geral) em conformidade com 
% ABNT NBR 14724:2011: Informacao e documentacao - Trabalhos academicos -
% Apresentacao
% ------------------------------------------------------------------------
% ------------------------------------------------------------------------

\documentclass[
	% -- opções da classe memoir --
	12pt,				% tamanho da fonte
	openright,			% capítulos começam em pág ímpar (insere página vazia caso preciso)
	twoside,			% para impressão em verso e anverso. Oposto a oneside 
	% oneside twoside twoside openright openany onecolumn twocolumn
	a4paper,			% tamanho do papel. 
	% -- opções da classe abntex2 --
	%chapter=TITLE,		% títulos de capítulos convertidos em letras maiúsculas
	%section=TITLE,		% títulos de seções convertidos em letras maiúsculas
	%subsection=TITLE,	% títulos de subseções convertidos em letras maiúsculas
	%subsubsection=TITLE,% títulos de subsubseções convertidos em letras maiúsculas
	% -- opções do pacote babel --
	english,			% idioma adicional para hifenização
	french,				% idioma adicional para hifenização
	spanish,			% idioma adicional para hifenização
	brazil				% o último idioma é o principal do documento
	]{abntex2}

% ---
% Pacotes básicos 
% ---
\usepackage{lmodern}			% Usa a fonte Latin Modern			
\usepackage[T1]{fontenc}		% Selecao de codigos de fonte.
\usepackage[utf8]{inputenc}		% Codificacao do documento (conversão automática dos acentos)
\usepackage{lastpage}			% Usado pela Ficha catalográfica
\usepackage{indentfirst}		% Indenta o primeiro parágrafo de cada seção.
\usepackage{color}				% Controle das cores
\usepackage{graphicx}			% Inclusão de gráficos
\usepackage{microtype} 			% para melhorias de justificação
% ---

%força tabelas a nao reposicionarem		
\usepackage{placeins}


% apendice
%\usepackage[titletoc]{appendix}
%\renewcommand\cftappendixname{\appendixname~}
%\AtBeginDocument{\renewcommand\appendixname{New Name}}

%%%CODE
\usepackage{listings}
\usepackage{caption}

\lstset{literate=
  {á}{{\'a~}}1 {é}{{\'e~}}1 {í}{{\'i~}}1 {ó}{{\'o~}}1 {ú}{{\'u~}}1
  {ã}{{\~a~}}1 {ẽ}{{\~e~}}1 {ĩ}{{\~i~}}1 {õ}{{\~o~}}1 {ũ}{{\~u~}}1
  {Á}{{\'A~}}1 {É}{{\'E~}}1 {Í}{{\'I~}}1 {Ó}{{\'O~}}1 {Ú}{{\'U~}}1
  {à}{{\`a~}}1 {è}{{\`e~}}1 {ì}{{\`i~}}1 {ò}{{\`o~}}1 {ù}{{\`u~}}1
  {À}{{\`A~}}1 {È}{{\'E~}}1 {Ì}{{\`I~}}1 {Ò}{{\`O~}}1 {Ù}{{\`U~}}1
  {ä}{{\"a~}}1 {ë}{{\"e~}}1 {ï}{{\"i~}}1 {ö}{{\"o~}}1 {ü}{{\"u~}}1
  {Ä}{{\"A~}}1 {Ë}{{\"E~}}1 {Ï}{{\"I~}}1 {Ö}{{\"O~}}1 {Ü}{{\"U~}}1
  {â}{{\^a~}}1 {ê}{{\^e~}}1 {î}{{\^i~}}1 {ô}{{\^o~}}1 {û}{{\^u~}}1
  {Â}{{\^A~}}1 {Ê}{{\^E~}}1 {Î}{{\^I~}}1 {Ô}{{\^O~}}1 {Û}{{\^U~}}1
  {œ}{{\oe~}}1 {Œ}{{\OE~}}1 {æ}{{\ae~}}1 {Æ}{{\AE~}}1 {ß}{{\ss~}}1
  {ç}{{\c c~}}1 {Ç}{{\c C~}}1 {ø}{{\o~}}1 {å}{{\r a~}}1 {Å}{{\r A~}}1
  {€}{{\EUR~}}1 {£}{{\pounds~}}1
}



\definecolor{dkgreen}{rgb}{0,0.6,0}
\definecolor{gray}{rgb}{0.5,0.5,0.5}
\definecolor{mauve}{rgb}{0.58,0,0.82}

\lstset{frame=tb,
  language=bash,
  aboveskip=3mm,
  belowskip=3mm,
  showstringspaces=false,
  columns=flexible,
  basicstyle={\small\ttfamily},
  numbers=none,
  numberstyle=\tiny\color{gray},
  keywordstyle=\color{blue},
  commentstyle=\color{dkgreen},
  stringstyle=\color{mauve},
  breaklines=true,
  breakatwhitespace=true,    % added this
  tabsize=3
}
%\DeclareCaptionFormat{listing}{\rule{\dimexpr\textwidth+17pt\relax}{0.4pt}\par\vskip1pt#1#2#3}
%\captionsetup[lstlisting]{format=listing,singlelinecheck=false, margin=0pt, font={sf},labelsep=space,labelfont=bf}

\renewcommand{\lstlistingname}{Código}




% ---
% Pacotes de citações
% ---
\usepackage[brazilian,hyperpageref]{backref}	 % Paginas com as citações na bibl
\usepackage[num]{abntex2cite}	% Citações padrão ABNT

% --- 
% CONFIGURAÇÕES DE PACOTES
% --- 
\usepackage[maxfloats=40]{morefloats}

\usepackage{amsmath}
\usepackage{multirow}
% helvetica no titulo dos capitulos
%\usepackage[scaled]{helvet}
\usepackage{tgheros}
\renewcommand{\ABNTEXchapterfont}{\fontfamily{\sfdefault}\selectfont}
\renewcommand{\ABNTEXchapterfontsize}{\HUGE}
%real numbers
\usepackage{amssymb}
%\hyphenation{Recomendação}

% ---
% Configurações do pacote backref
% Usado sem a opção hyperpageref de backref
\renewcommand{\backrefpagesname}{Citado na(s) página(s):~}
% Texto padrão antes do número das páginas
\renewcommand{\backref}{}
% Define os textos da citação
\renewcommand*{\backrefalt}[4]{
	\ifcase #1 %
		Nenhuma citação no texto.%
	\or
		Citado na página #2.%
	\else
		Citado #1 vezes nas páginas #2.%
	\fi}%
% ---



% ---
% Informações de dados para CAPA e FOLHA DE ROSTO
% ---
\titulo{Proposta de algoritmo e desenvolvimento de biblioteca para sistemas de recomendação de produtos de lojas de comércio online}
\autor{Antônio Guilherme Ferreira Viggiano \\ Fernando Fochi Silveira Araújo}
\local{São Paulo, Brasil}
\data{\today}
\orientador{Prof. Dr. Fábio Gagliardi Cozman}
\coorientador{}
\instituicao{%
  Universidade de São Paulo
  \par
  Escola Politécnica
  \par
  Trabalho de Conclusão de Curso}
\tipotrabalho{Trabalho de Formatura}
% O preambulo deve conter o tipo do trabalho, o objetivo, 
% o nome da instituição e a área de concentração 
\preambulo{Trabalho de Conclusão de Curso apresentado ao Departamento de Engenharia Mecatrônica da Escola Politécnica da Universidade de São Paulo com requisito parcial para obtenção do Grau de Engenheiro Mecatrônico.}
% ---
\DeclareMathOperator*{\argmax}{arg\,max}


% ---
% Configurações de aparência do PDF final

% alterando o aspecto da cor azul
\definecolor{blue}{RGB}{41,5,195}

% informações do PDF
\makeatletter
\hypersetup{
     	%pagebackref=true,
		pdftitle={\@title}, 
		pdfauthor={\@author},
    	pdfsubject={\imprimirpreambulo},
	    pdfcreator={LaTeX with abnTeX2},
		pdfkeywords={recsys}{recommendation}{e-commerce}{sistemas de recomendação}{trabalho acadêmico}{varejo on-line}, 
		colorlinks=true,       		% false: boxed links; true: colored links
    	linkcolor=blue,          	% color of internal links
    	citecolor=blue,        		% color of links to bibliography
    	filecolor=magenta,      		% color of file links
		urlcolor=blue,
		bookmarksdepth=4
}
\makeatother
% --- 

% --- 
% Espaçamentos entre linhas e parágrafos 
% --- 

% O tamanho do parágrafo é dado por:
\setlength{\parindent}{1.3cm}

% Controle do espaçamento entre um parágrafo e outro:
\setlength{\parskip}{0.2cm}  % tente também \onelineskip

% ---
% compila o indice
% ---
\makeindex
% ---

% ----
% Início do documento
% ----
\begin{document}

% Retira espaço extra obsoleto entre as frases.
\frenchspacing 

% ----------------------------------------------------------
% ELEMENTOS PRÉ-TEXTUAIS
% ----------------------------------------------------------
% \pretextual

\imprimircapa

\imprimirfolhaderosto

%!TEX root = index.tex
\begin{fichacatalografica}\label{ficha catalografica}
	\vspace*{\fill}					% Posição vertical
	\hrule							% Linha horizontal
	\begin{center}					% Minipage Centralizado
	\begin{minipage}[c]{12.5cm}		% Largura
	
	\imprimirautor
	
	\hspace{0.5cm} \imprimirtitulo  / \imprimirautor. --
	\imprimirlocal, \imprimirdata-
	
	\hspace{0.5cm} \pageref{LastPage} p. : il. (algumas color.) ; 30 cm.\\
	
	\hspace{0.5cm} \imprimirorientadorRotulo~\imprimirorientador\\
	
	\hspace{0.5cm}
	\parbox[t]{\textwidth}{\imprimirtipotrabalho~--~\imprimirinstituicao,
	\imprimirdata.}\\
	
	\hspace{0.5cm}
		1. Sistema de recomendação.
		2. E-Commerce.
		I. Prof. Dr. Fábio Gagliardi Cozman.
		II. Universidade de São Paulo.
		III. Escola Politécnica.
		IV. Desenvolcimento de um Sistema de Recomendação para E-Commerce\\ 			
	
	\hspace{8.75cm} CDU xx.xxx.xxx.x\\
	
	\end{minipage}
	\end{center}
	\hrule
\end{fichacatalografica}
%\pagebreak

%%!TEX root = index.tex
\begin{folhadeaprovacao}

  \begin{center}
    {\ABNTEXchapterfont\large\imprimirautor}

    \vspace*{\fill}\vspace*{\fill}
    \begin{center}
      \ABNTEXchapterfont\bfseries\Large\imprimirtitulo
    \end{center}
    \vspace*{\fill}
    
    \hspace{.45\textwidth}
    \begin{minipage}{.5\textwidth}
        \imprimirpreambulo
    \end{minipage}%
    \vspace*{\fill}
   \end{center}
        
   %Trabalho aprovado. \imprimirlocal, DIA de Mês de 2014:

    \assinatura{\textbf{\imprimirorientador} \\ Orientador} 
   \assinatura{\textbf{Prof. Dr. Lucas Antonio Moscato} \\ Convidado 1}
   \assinatura{\textbf{Prof. Dr. Thiago de Castro Martins} \\ Convidado 2}
   \assinatura{\textbf{Prof. Dr. Arturo Forner Cordero} \\ Convidado 3}
   \assinatura{\textbf{Profa. Dra. Larissa Driemeier} \\ Convidado 4}
      
   \begin{center}
    \vspace*{0.5cm}
    {\large\imprimirlocal}
    \par
    {\large\imprimirdata}
    \vspace*{1cm}
  \end{center}
  
\end{folhadeaprovacao}
%\pagebreak

%%!TEX root = index.tex
\begin{dedicatoria}
   \vspace*{\fill}
   \centering
   \noindent
   \textit{ Dedicamos este trabalho ao Professor Fábio Cozman, pela orientação e apoio } \vspace*{\fill}
\end{dedicatoria}
%\pagebreak

%!TEX root = index.tex
\begin{agradecimentos}
Agradecemos ao professor Fábio Cozman pela sua orientação e apoio durante todo o projeto. Agradecemos também ao professor Thiago Martins e aos demais orientadores das disciplinas PMR2500 e PMR2550 -- Projeto de Conclusão do Curso I e II -- por terem nos guiado na elaboração da monografia e por terem sempre exigido trabalhos de alta qualidade. Esse papel é fundamental na valorização do diploma de Engenharia Mecatrônica da Escola Politécnica. 
\end{agradecimentos}
%\pagebreak

%!TEX root = index.tex
\begin{epigrafe}
    \vspace*{\fill}
	\begin{flushright}
		\textit{Aqui vai uma bela epigrafe}
	\end{flushright}
\end{epigrafe}
%\pagebreak

%!TEX root = index.tex
% resumo em português
\setlength{\absparsep}{18pt} % ajusta o espaçamento dos parágrafos do resumo
\begin{resumo}
Resumo em português

 \textbf{Palavras-chaves}: este é o resumo em português
\end{resumo}
%\pagebreak

%!TEX root = index.tex
% resumo em português
\setlength{\absparsep}{18pt} % ajusta o espaçamento dos parágrafos do resumo
\begin{resumo}[Abstract]
 \begin{otherlanguage*}{english}

This project's scope is to design and assess a recommender system algorithm and library for e-commerces. The goal of this library is to make the implementation of a generic recommender system simple and easy, so it can be used by the academics and e-commerces willing to automate the suggestion of items, such as in email marketing.

The library was developed using three different recommendation algorithms. The feature weighted is a hybrid method, based on collaborative filtering and content-based filtering, in which a linear regression is calculated from a social-network database, extracting the weights that determine each attribute's importance. The second method, based in user profiles, considers the users' interests in specific features, indirectly calculated by the users' interest in different items. The third method, based in the user-item correlation, is derived from the method based in users' profiles and was developed by the authors. This method searches for them items with the features that are more similar to the attributes that the user has shown interest for.

The comparative assessment of the methods has shown the superiority of the user-profile algorithm in almost all aspects, and has measured the main parameters that affect the recommendation quality. From the empirical results shown in this work, it is possible to establish some guidelines on how to create a recommender system based on the library developed by the authors.

   \vspace{\onelineskip}
 
   \noindent 
   \textbf{Key-words}: Artificial intelligence, Machine learning, e-Commerce, Products.
 \end{otherlanguage*}
\end{resumo}

%\pagebreak

% inserir lista de tabelas
% ---
\pdfbookmark[0]{\listtablename}{lot}
\listoftables*
\cleardoublepage
\listoffigures*
\cleardoublepage

% ---


%!TEX root = index.tex
\begin{simbolos}\label{simbolos}
    \item[$k$]  Número de vizinhos mais próximos 
    \item[$N$] Tamanho da lista de recomendação  
    \item[$\mathcal{U}$] Conjunto de todos os usuários 
    \item[$\mathcal{I}$] Conjunto de todos os itens  
    \item[$\mathcal{F}$] Conjunto  de todos os atributos dos itens
    %\item[$\mathcal{C}$] Conjunto  de todas as características dos usuários
    \item[$u, v$] Usuários 
    \item[$i, j$] Itens 
    \item[$f$] Atributos dos itens 
    %\item[$c $] Características dos usuários  
    \item[$\mathbf{X}_{M \times N},~\mathbf{X}$] Matriz de elementos $x_{mn}$ 
    \item[$\mathbf{x}_{N},~\mathbf{x}$] Vetor de elementos $x_{n}$
    \item[$\tilde{x}$] Valor ótimo de $x$
    \item[$\hat{x}$] Valor estimado de $x$
    \item[$|\mathcal{X}|$] Número de elementos do conjunto $\mathcal{X}$
    \item[$\mathbf{R}, r_{ui}$] Avaliação feita pelo usuário $u$ do item $i$
    \item[$\mathbf{A}, a_{if}$] Atributo $f$ presente no item $i$
    %\item[$\mathbf{B}, b_{uc}$] Característica $c$ do usuário $u$   
 %   \item[$\mathbf{T}, t_{uf}$] Correlação entre usuário $u$ e atributo $f$
    \item[$\mathbf{S}, s_{ij}, s_{uv}$] Similaridade entre itens $i$ e $j$ ou entre usuários $u$ e $v$
    \item[$\mathbf{W}, w_{uf}$] Correlação ponderada entre usuário $u$ e atributo $f$ 
    \item[$\mathbf{\Omega}, \omega_{ui}$] Correlação entre usuário $u$ e item $i$ 
    \item[$\mathbf{w}, w_{f}$] Peso do atributo $f$
\end{simbolos}

% ---
% inserir o sumario
% ---
\pdfbookmark[0]{\contentsname}{toc}
\tableofcontents*
\cleardoublepage
% ---

% ----------------------------------------------------------
% ELEMENTOS TEXTUAIS
% ----------------------------------------------------------
\textual


%inteligencia artificial
%aprendizado computacional
% comercio eletronico
% produtos

%!TEX root = index.tex
\section[Introdução]{Introdução}
\begin{frame}
\frametitle{Introdução}
\begin{block}{Definição: Sistemas de recomendação}
``São ferramentas e técnicas de software destinadas a prover sugestões de itens para usuários'' \cite{ricci2011introduction-chap1}
\end{block}
\end{frame}


\begin{frame}
\frametitle{Introdução}
\begin{block}{Etapas principais}
\begin{itemize}
	\item Aquisição dos dados de entrada
	\item Determinação das recomendações
	\item Apresentação dos resultados ao usuário
\end{itemize}
\end{block}
\end{frame}

\begin{frame}
\frametitle{Introdução}
\begin{itemize}
	\item importância econômica de lojas online
	\item criação de ferramentas \textit{open source} para a comunidade
\end{itemize}

\begin{center}
$\Downarrow$ 
\end{center}

\begin{center}
Desenvolvimento de um sistema de recomendação \par{} de produtos para e-commerces
\end{center}

\end{frame}

%\pagebreak
%!TEX root = index.tex
\section{Motivação} % (fold)
\label{cha:motivacao}

Conforme apresentado, a quantidade de lojas de varejo online cresce em ritmo acelerado no Brasil e no mundo. Motivados pela importância econômica dos e-commerces, bem como pela possibilidade de criar um conjunto de ferramentas \textit{open source} que possam ser utilizadas pela comunidade acadêmica e empresarial, propomos como Trabalho de Conclusão de Curso o desenvolvimento de um algoritmo e biblioteca computacional para sistemas de recomendação de produtos de lojas de comércio online.

O pacote computacional é composto de métodos de leitura de dados de histórico de compras e de informações de clientes e produtos, de cálculo de sugestões de itens com base em algoritmos de recomendação e de análise de desempenho das recomendações. Além disso, propusemos um algoritmo baseado em um método já existente, a fim de avaliar a sua qualidade comparado a outras soluções.

A motivação de se criar uma biblioteca de software decorre principalmente da sua abrangência e capacidade de adaptação, visto que é possível atender a mais casos de uso que um sistema de recomendação completo. De um lado, um sistema de recomendação possui uma finalidade específica -- como por exemplo de sugerir notícias para usuários de internet -- e uma entrada e saída de dados específica -- como por exemplo o fato de as notícias sempre estarem ordenada pelas mais recentes em uma tabela de sugestões. De outro lado, uma biblioteca computacional pode receber qualquer tipo de dados e gerar qualquer saída de dados. 

Caso uma empresa ou um acadêmico queira construir seu próprio sistema de recomendação, basta elaborar a conexão entre o pacote apresentado pela dupla, seu banco de dados e a interface gráfica de apresentação de resultados.

As contribuições científica e tecnológica deste trabalho para a Engenharia Mecatrônica estão sobretudo nos campos de inteligência artificial, de sistemas de informação e de automação de processos.

As competências acadêmicas necessárias para a execução desse trabalho envolvem algoritmos e estruturas de dados (abordados em PMR2300 -- Computação para Automação), documentação e modelagem de sistemas computacionais (explicados em PMR2440 -- Programação para Automação), sistemas de informação e banco de dados (tratados em PMR2490 -- Sistemas de Informação) e inteligencia artificial, com enfase em aprendizado de máquina (aprofundados em PMR2728 -- Teoria de Probabilidades em Inteligência Artificial e Robótica). As competências técnicas abrangem programação estatística e funcional, demonstradas através da linguagem R.

%Os principais desafios do projeto constituem o tratamento da grande quantidade de dados, que serão coletados de diversas bases contendo mais de 100 mil avaliações e 25 mil itens; determinação de medidas de similaridade para as recomendações; análise de performance das implementações propostas; escolha da solução definitiva e, por fim, comparação entre os sistemas de recomendação existentes e o que será apresentado pela dupla.



%\pagebreak
%!TEX root = index.tex
\chapter[Objetivos]{Objetivos}
\label{chap:objetivos}

O objetivo do presente Trabalho de Conclusão de Curso é o desenvolvimento de um Sistema de Recomendação de produtos para lojas de comércio online, e respectiva análise de desempenho das recomendações propostas. 

Serão avaliados diferentes algoritmos de recomendação, e será feita uma avaliação comparativa entre cada um dos métodos. A explicação detalhada de cada um deles se encontra no Capítulo \ref{chap:sintese_de_solucoes}.

O sistema a ser desenvolvido será, do ponto de vista da taxonomia tradicional dos sistemas de recomendação \cite{schafer1999recommender}, automático e persistente. Isso significa que as sugestões serão dadas sem a interação do usuário e que as compras anteriores serão levadas em conta. Essas características aproximam o entregável das ferramentas de marketing via e-mail, que sugerem produtos com uma determinada frequência aos usuários com base em seu histórico de compras. 

A qualidade das recomendações será avaliada tanto em termos da similaridade entre os itens efetivamente comprados pelo cliente com aqueles previstos pelo sistema de recomendação, quanto em termos de indicadores de erro tipo I e erro tipo II, como a medida F \cite{sarwar2000analysis}. 

Por meio de uma validação cruzada, analisaremos a influência dos principais parâmetros do problema na qualidade das recomendações, como o tamanho do banco de dados ou a quantidade de informações de itens e clientes utilizadas na recomendação. 

Será discutido o impacto dos principais desafios tecnológicos e científicos dos sistemas de recomendação na nossa proposta de solução, tais como a escalabilidade, a adaptação a novos usuários e a dispersão dos dados \cite{wei2007survey}. Também serão avaliadas as diferentes medidas de similaridade e modelos de predição na qualidade das recomendações. 

Ao final, será possível extrair uma validação experimental das diretrizes fundamentais a serem seguidas por e-commerces que desejem desenvolver um sistema de recomendação próprio, a partir de um banco de dados de clientes, produtos e histórico de compras. 
%\pagebreak
%!TEX root = index.tex
\section[Estado da Arte]{Estado da Arte}
\begin{frame}
\frametitle{Estado da Arte}
\end{frame}

%\pagebreak
%!TEX root = index.tex
\section[Metodologia]{Metodologia}
\begin{frame}{Metodologia}{Estruturação do banco de dados}

\begin{description}
	\item[100k] 100 000 avaliações de 943 usuários \\ para 1682 filmes
	\item[IMDB] 28 819 filmes
	\item[\textbf{IMDB-100k}] 943 usuários, 1682 filmes e 25 atributos
\end{description}
\end{frame}

\begin{frame}[fragile]
\frametitle{Metodologia}
\framesubtitle{Desenvolvimento da biblioteca}


\textbf{Ferramenta utilizada}
	\begin{description}
		\item[RStudio] Editor de texto e console
	\end{description}
\textbf{Estrutura da biblioteca}
\begin{columns}
\column{0.4\textwidth}
\begin{verbatim}
recsys/
|-- db
|   `-- ml-100k
|       |-- u.data
|       |-- u.item
|       |-- u.user
|       |-- ...
|-- methods
|   |-- fw.R
|   |-- ui.R
|   `-- up.R
\end{verbatim}
\column{0.5\textwidth}
\begin{verbatim}
|-- results
|   |-- benchmark.R
|   |-- performance.R
|   `-- run_tests.R
`-- setup
    |-- functions.R
    `-- setup.R
\end{verbatim}
\end{columns}
\end{frame}


\begin{frame}{Metodologia}{Validação cruzada}
\begin{columns}[t] 
\column{.5\textwidth} 
\textbf{Avaliação}
\begin{itemize}
	\item $T=75\%$ base de treinamento
	\item $H=75\%$ dados ``escondidos''
\end{itemize}
\begin{table}[h]
\begin{center}
	\caption{Avaliações $r_{ui}$}
    \begin{tabular}{ | c | c | c | c | c | c |} 
    \hline
     & $i_1$ & $i_2$ & $i_3$ & $i_4$ \\ \hline
     $u_1$ & - & 4 & 3 & 5 \\ \hline
     $u_2$ & 2 & 5 & - & 1 \\ \hline
     $u_3$ & 3 & - & - & 2 \\ \hline
     $u_4$ & (5) & (2) & (3) & 4 \\ \hline
    \end{tabular}
    % TODO color table
\end{center}
\end{table}
\column{.5\textwidth}
\textbf{Ambiente de testes} 
\begin{itemize}
    \item Máquina r3.large
    \item 2 vCPU
    \item 15 GB de memória RAM
    \item Amazon Linux AMI release 2014.09 x86\_64
    \item Custo total R\$ 5,70
 \end{itemize}

\end{columns}
\end{frame}
%!TEX root = index.tex
\section[Requisitos]{Requisitos}
\begin{frame}
\frametitle{Requisitos}
\begin{block}{Requisitos funcionais}
\begin{itemize}
	\item EMA máximo: 
	\item $20\%$ para Precisão
	\item $20\%$ para Abrangência
	\item \textit{Throughput} mínimo
	\begin{itemize}
		\item 100 mil recomendações por hora
	\end{itemize}
\end{itemize}
\end{block}

\begin{block}{Requisitos não funcionais}
\begin{itemize}
	\item Escalabilidade
	\item Sistema genérico
	\begin{itemize}
		\item Padronização dos dados de entrada/saída
	\end{itemize}
	\item Código aberto
\end{itemize}
\end{block}
\end{frame}

%\pagebreak
%!TEX root = index.tex
\chapter{Avaliação de Desempenho} % (fold)
\label{cha:avalia_o_de_desempenho}

% chapter avalia_o_de_desempenho (end)

% chapter avalia_o_do_sistema_de_recomenda_o (end)

% section avalia_o_do_sistema_de_recomenda_o (end)

De modo geral os sistemas de recomendação tem o objetivo de apresentar ao usuário itens pelos quais ele possa se interessar e que, no caso de um e-commerce,  ele vá adquirir. O desempenho de um sistema de recomendação se mede, portanto, na qualidade com a qual ele executa essa tarefa. Essa qualidade pode ser medida de diferentes maneiras, tal como pela medida de distância entre os produtos recomendados $\hat{\textbf{\i}}$ e aqueles que seriam efetivamente comprados $\textbf{i}$ pelo cliente em uma validação cruzada (\textit{cross validation}). Essa medida pode ser, por exemplo, a distância $L_1$ (erro médio absoluto, $\left|\hat{\textbf{\i}} - \textbf{i}\right|$) ou a distância $L_2$ (erro quadrático médio,  $\sqrt{\left|\hat{\textbf{\i}} - \textbf{i}\right|^2}$).

Outras medidas de predição também serão utilizadas, tais como acurácia (\textit{accuracy}), especificidade (\textit{specificity}), precisão (\textit{precision}), abrangência (\textit{recall}) e a medida $F_1$ (\textit{$F_1$-score}). Elas estão sumarizadas na Tabela \ref{tab:avaliacao-predicao}.

%\begin{table}[H]
%\begin{center}
%    \caption{Matriz de confusão}
%    \label{tab:avaliacao-predicao}
%    \begin{tabular}{ | l | l | p{5cm} | p{5cm} | }
%    \hline
%    & & \multicolumn{2}{|c|}{Caso predito} \\ \hline 
%    & & \textbf{Positivo} & \textbf{Negativo} \\ \hline
%    \multirow{2}{*}{Caso real} 
%        & Positivo & Verdadeiro Positivo &  $(VP)$ & Falso Negativo $(FN)$ \\ \hline
%        & Negativo & Falso Positivo $(FP)$ & Verdadeiro Negativo $(VN)$ \\ \hline
%    \end{tabular}
%\end{center}
%\end{table}


%\begin{tabular}{cc|c|c|c|c|l}
%\cline{3-4}
%& & \multicolumn{2}{ c| }{Caso predito} \\ \cline{3-4}
%& & 2 & 3  \\ \cline{1-4}
%\multicolumn{1}{ |c| }{\multirow{2}{*}{Powers} } &
%\multicolumn{1}{ |c| }{504} & 3 & 2 &      \\ \cline{2-4}
%\multicolumn{1}{ |c  }{}                        &
%\multicolumn{1}{ |c| }{540} & 2 & 3 &      \\ \cline{1-4}
%\multicolumn{1}{ |c  }{\multirow{2}{*}{Powers} } &
%\multicolumn{1}{ |c| }{gcd} & 2 & 2   \\ \cline{2-4}
%\multicolumn{1}{ |c  }{}                        &
%\multicolumn{1}{ |c| }{lcm} & 3 & 3   \\ \cline{1-4}
%\end{tabular}


\begin{table}[hp]
\begin{center}
    \caption{Avaliação de sistemas de predição}
    \label{tab:avaliacao-predicao}
    \begin{tabular}{  | >{\arraybackslash} m{3cm} | >{\centering\arraybackslash} m{4cm} | >{\arraybackslash} m{6cm} | }
    \hline
    \textbf{Medida} & \textbf{Fórmula} & \textbf{Significado} \\ \hline
    Precisão &  $\frac{VP}{VP+FP}$ & Porcentagem de casos positivos corretamente preditos. \\ \hline                            
    Abrangência & $\frac{VP}{VP+FN}$ & Porcentagem de casos positivos sobre aqueles que foram marcados como positivos. \\ \hline
    Especificidade & $\frac{VN}{VN+FP}$ &  Porcentagem de casos negativos sobre aqueles que foram marcados como negativos. \\ \hline
    Acurácia & $\frac{VP+VN}{VP+VN+FP+FN}$ & Porcentagem de predições corretas. \\ \hline
    Medida $F_1$ &  $2 \cdot \frac{\mathrm{Precisão}~\cdot~\mathrm{Abrangência}}{\mathrm{Precisão}~+~\mathrm{Abrangência}}$ & Média harmônica entre precisão e abrangência. \\ \hline
    \end{tabular}
\end{center}
\end{table}

Por fim, avaliaremos o desempenho do sistema mediante a mudança nas variáveis de importância do problema, como por exemplo na quantidade de atributos utilizados na recomendação. O tempo de execução também será avaliado em função do tamanho do banco de dados e do algoritmo utilizado.

%e na capacidade de lidar com problemas como o \textit{cold start}
%\pagebreak
%!TEX root = index.tex
\chapter[Detalhamento de Soluções]{Detalhamento de Soluções}
\label{chap:sintese_de_solucoes}

A fim de facilitar a compreensão dos métodos propostos neste trabalho, serão utilizadas as matrizes de avaliações $\mathbf{R}$ e de atributos $\mathbf{A}$ abaixo, adaptadas da Referência \citeonline{debnath2008feature}. Em todos os exemplos, considera-se valor mínimo $M=2$. Os logaritmos são expressos em base 10 e todos os pesos $w_f$, descritos a seguir, são utilizados.

\begin{table}[h]
\begin{center}
    \caption{Avaliações $r_{ui}$}
    \label{tab:rui_ref}
    \begin{tabular}{ | c | c | c | c | c | c | c | } 
    \hline
     & $i_1$ & $i_2$ & $i_3$ & $i_4$ & $i_5$ & $i_6$ \\ \hline
     $u_1$ & - & 4 & - & - & 5 & - \\ \hline
     $u_2$ & - & 3 & - & 4 & - & - \\ \hline
     $u_3$ & - & - & - & - & - & 4 \\ \hline
     $u_4$ & 5 & - & 3 & - & - & - \\ \hline
    \end{tabular}
\end{center}
\end{table}

\begin{table}[h]
\begin{center}
    \caption{Atributos $a_{if}$}
    \label{tab:aif_ref}
    \begin{tabular}{ | c | c | c | c | c | } 
    \hline
     & $f_1$ & $f_2$ & $f_3$ & $f_4$  \\ \hline
     $i_1$ & 0 & 1 & 0 & 0  \\ \hline
     $i_2$ & 1 & 1 & 0 & 0  \\ \hline
     $i_3$ & 0 & 1 & 1 & 0  \\ \hline
     $i_4$ & 0 & 1 & 0 & 0  \\ \hline
     $i_5$ & 1 & 1 & 1 & 0  \\ \hline
     $i_6$ & 0 & 0 & 0 & 1  \\ \hline
    \end{tabular}
\end{center}
\end{table}

\section{Algoritmo baseado na ponderação de atributos (FW)} % (fold)
\label{sec:algoritmo_baseado_na_pondera_o_de_atributos_}

% section algoritmo_baseado_na_pondera_o_de_atributos_ (end)

O primeiro algoritmo que utilizaremos no sistema de recomendação, adaptado da Referência \citeonline{symeonidis2007feature} e denominado ponderação de atributos, \textit{feature weighting} ou FW, trata-se de um híbrido entre filtragem colaborativa e filtragem baseada em conteúdo. A partir da regressão linear de dados de uma rede social (\textit{Internet Movie Database, IMDB}), extraem-se os pesos que determinam a importância de cada atributo dos itens, e é onde ocorre a filtragem colaborativa dos usuários. Após obtenção dos pesos, realiza-se a filtragem baseada em conteúdo para determinar os itens com maior similaridade, que são finalmente recomendados.

Na filtragem baseada em conteúdo, ``cada item é representado por um vetor de atributos ou \textit{features}''. A similaridade $s_{ij}$ entre dois itens $i$ e $j$ é dada pela média ponderada das distâncias entre as \textit{features} dos itens:

\begin{equation} 
\label{eq:sij}
    s_{ij} = \sum_{f}{w_{f} \left(1-d_{fij}\right)}
\end{equation}

As distâncias entre os atributos $d_f$ são determinadas conforme o tipo de dado avaliado e seu domínio, normalizadas no intervalo $\left[0,1\right]$. 

Para atributos literais, como categoria, marca, cor, etc., uma possível medida de distância é o delta de Kronecker descrito em \ref{eq:delta}. A similaridade entre as cores ``azul'' e ``vermelho'' é, nesse caso, 0, e sua distância é 1. O valor da distância é nulo se e somente se os atributos são idênticos.

Para atributos pertencentes a uma coleção finita de itens, tais como os atores participantes de um filme, é possível estabelecer a similaridade entre dois conjuntos a partir do índice Jaccard, descrito em \ref{eq:jaccard}. Neste caso, a similaridade entre os conjuntos \{Al Pacino, Tom Hanks\} e \{Tom Hanks, Marlon Brando\} é $1/3$, e a sua distância é $2/3$.


\begin{equation}
\label{eq:delta}
\delta_{mn} =  
\begin{cases}
1, &\text{se }m=n \\
0, &\text{se }m \neq n
\end{cases} 
\end{equation}

\begin{equation}
\label{eq:jaccard}
J(A,B) ={{|A \cap B|}\over{|A \cup B|}}
\end{equation}

Vale considerar a correlação entre atributos no cálculo das distâncias: a similaridade de duas marcas de calçado, por exemplo, é maior que a de duas marcas de produtos de categorias diferentes, mesmo que as marcas sejam distintas nos dois casos. Em uma primeira análise, todavia, utilizaremos para a maior parte das \textit{features} as medidas de distância do delta de Kronecker \ref{eq:dfij_delta} (Tabela \ref{tab:d_ijf}) e do índice Jaccard \ref{eq:dfij_jaccard}. Isso significa que se os atributos de dois itens são idênticos, a distância é nula e portanto a similaridade é máxima. O sumário de algumas medidas de distância que podem ser utilizadas para casos específicos estão na Tabela \ref{tab:medidas-distancia}.

\begin{equation}
\label{eq:dfij_delta}
\begin{split}
d_{fij} =&~ 1-\delta_{ij}^f \\
    =&~ 1-\delta_{a_{if} a_{jf}}
\end{split} 
\end{equation}


\begin{table}[p]
\begin{center}
    \caption{$d_{ij}^{f}$}
    \label{tab:d_ijf}
    \begin{tabular}{ | c | c | c | c | c | c | c | } 
    \hline
     $f_1$ & $i_1$ & $i_2$ & $i_3$ & $i_4$ & $i_5$ & $i_6$  \\ \hline
     $i_1$ & -  &  0  &  1  &  1  &  0  &  1 \\ \hline
     $i_2$ & 0  &  -  &  0  &  0  &  1  &  0  \\ \hline
     $i_3$ & 1  &  0  &  -  &  1  &  0  &  1 \\ \hline
     $i_4$ & 1  &  0  &  1  &  -  &  0  &  1 \\ \hline
     $i_5$ & 0  &  1  &  0  &  0  &  -  &  0 \\ \hline
     $i_6$ & 1  &  0  &  1  &  1  &  0  &  - \\ \hline
    \end{tabular}
    \quad
    \begin{tabular}{ | c | c | c | c | c | c | c | } 
    \hline
     $f_2$ & $i_1$ & $i_2$ & $i_3$ & $i_4$ & $i_5$ & $i_6$  \\ \hline
     $i_1$ & -  &  1  &  1   & 1  &  1  &  0 \\ \hline
     $i_2$ & 1  &  -  &  1   & 1  &  1  &  0  \\ \hline
     $i_3$ & 1  &  1  &  -   & 1  &  1  &  0 \\ \hline
     $i_4$ & 1  &  1  &  1   & -  &  1  &  0 \\ \hline
     $i_5$ & 1  &  1  &  1   & 1  &  -  &  0 \\ \hline
     $i_6$ & 0  &  0  &  0   & 0  &  0  &  - \\ \hline
    \end{tabular}
    
    \begin{tabular}{ | c | c | c | c | c | c | c | } 
    \hline
     $f_3$ & $i_1$ & $i_2$ & $i_3$ & $i_4$ & $i_5$ & $i_6$  \\ \hline
     $i_1$ & -  &  1  &  0  &  1  &  0  &  1 \\ \hline
     $i_2$ & 1  &  -  &  0  &  1  &  0  &  1  \\ \hline
     $i_3$ & 0  &  0  &  -  &  0  &  1  &  0 \\ \hline
     $i_4$ & 1  &  1  &  0  &  -  &  0  &  1 \\ \hline
     $i_5$ & 0  &  0  &  1  &  0  &  -  &  0 \\ \hline
     $i_6$ & 1  &  1  &  0  &  1  &  0  &  - \\ \hline
    \end{tabular}    
    \quad
    \begin{tabular}{ | c | c | c | c | c | c | c | } 
    \hline
     $f_4$ & $i_1$ & $i_2$ & $i_3$ & $i_4$ & $i_5$ & $i_6$  \\ \hline
     $i_1$ & -  &  1  &  1 &   1 &   1  &  0 \\ \hline
     $i_2$ & 1  &  -  &  1 &   1 &   1  &  0  \\ \hline
     $i_3$ & 1  &  1  &  - &   1 &   1  &  0 \\ \hline
     $i_4$ & 1  &  1  &  1 &   - &   1  &  0 \\ \hline
     $i_5$ & 1  &  1  &  1 &   1 &   -  &  0 \\ \hline
     $i_6$ & 0  &  0  &  0 &   0 &   0  &  - \\ \hline
    \end{tabular}        
\end{center}
\end{table}


\begin{equation}
\label{eq:dfij_jaccard}
\begin{split}
d_{fij} =&~ 1-J^f(i,j) \\
    =&~ 1-J(a_{if},a_{jf})
\end{split} 
\end{equation}

\begin{table}[hp]
\begin{center}
    \caption{Medidas de distância entre alguns atributos}
    \label{tab:medidas-distancia}
    \begin{tabular}{  | >{\arraybackslash} m{3cm} | >{\arraybackslash} m{3cm} | >{\centering\arraybackslash} m{3cm} | } 
    \hline
    \textbf{Atributo} $f$ & \textbf{Domínio} $\mathrm{F}$ & \textbf{Distância} $d_f$ \\ \hline
    Marca & Literal & $1-\delta^f_{ij}$ \\ \hline    
    Esporte & Literal & $1-\delta^f_{ij}$ \\ \hline
    Gênero & Literal & $1-\delta^f_{ij}$ \\ \hline            
    Categoria & Conjunto Literal & $1-J^f(i,j)$ \\ \hline            
    Preço & $\mathbb{R}$ & $ \frac{\left| a_{if}-a_{jf} \right|}{\max_{i,j}{\left| a_{if}-a_{jf} \right|}} $ \\ \hline
    Data & $\mathbb{R}$ milissegundos a partir de \textit{epoch} \cite{epoch} & $ \frac{\left| a_{if}-a_{jf} \right|}{\max_{i,j}{\left| a_{if}-a_{jf} \right|}} $ \\ \hline
    \end{tabular}
\end{center}
\end{table}
 
Os pesos $w_f$ são a priori desconhecidos. A Referência \citeonline{symeonidis2007feature} os determina a partir de uma regressão linear do tipo \ref{eq:regressao-linear}, onde $e_{ij}$ é o número de usuários que se interessam tanto por $i$ quanto por $j$. Esses valores permitem determinar ``o julgamento humano de similaridade entre itens'', e pode ser calculado a partir da matriz de avaliações, conforme a equação \ref{eq:determinacao-eij} (Tabela \ref{tab:eij}). O operador booleano $\mathrm{b}_M$, descrito pela Equação \ref{eq:b0}, nada mais é que uma ferramenta matemática para se poder extrair o número de usuários que avaliaram \textit{positivamente} tanto $i$ quanto $j$ a partir de $\mathbf{R}$. 


\begin{equation}
\label{eq:regressao-linear} 
    e_{ij} = w_0 + \sum_{f}{w_{f} \left(1-d_{fij}\right)}
\end{equation} 


\begin{equation}
\label{eq:determinacao-eij} 
    e_{ij} = \sum_{u}{\mathrm{b_M}\left(r_{ui} ~ r_{uj}\right)}
\end{equation} 

\begin{equation}
\label{eq:b0}
\mathrm{b}_M\left(x\right) = 
\begin{cases}
1, &\text{se }x>M \\
0, &\text{se }x\leq M
\end{cases} 
\end{equation}

\begin{table}[p]
\begin{center}
    \caption{$e_{ij}$}
    \label{tab:eij}
    \begin{tabular}{ | c | c | c | c | c | c | c | } 
    \hline
     & $i_1$ & $i_2$ & $i_3$ & $i_4$ & $i_5$ & $i_6$  \\ \hline
     $i_1$ & - & 0 & 1 & 0 & 0 & 0 \\ \hline
     $i_2$ & 0 & - & 0 & 1 & 1 & 0  \\ \hline
     $i_3$ & 1 & 0 & - & 0 & 0 & 0 \\ \hline
     $i_4$ & 0 & 1 & 0 & - & 0 & 0 \\ \hline
     $i_5$ & 0 & 1 & 0 & 0 & - & 0 \\ \hline
     $i_6$ & 0 & 0 & 0 & 0 & 0 & - \\ \hline
    \end{tabular}
\end{center}
\end{table}

Desta forma, os pesos $w_f$ são determinados a partir resolução do sistema de equações lineares \ref{eq:determinacao-wf} (Tabela \ref{tab:w_f}). Apenas os pesos positivos e com valor absoluto expressivo (maior que um piso arbitrariamente escolhido a posteriori) são utilizados na recomendação. 

\begin{equation}
\label{eq:determinacao-wf} 
    w_0 + \sum_{f}{w_{f}  \left(1-d_{fij}\right)} = \sum_{u}{\mathrm{b_0}\left(r_{ui} ~ r_{uj}\right)},~\forall i \neq j 
\end{equation} 

\begin{table}[p]
\begin{center}
    \caption{$w_f$}
    \label{tab:w_f}
    \begin{tabular}{ | c | c | c | c | c | } 
    \hline
     $w_0$ & $w_1$ & $w_2$ & $w_3$ & $w_4$   \\ \hline
     0.41 & -0.22 & -0.34 & -0.03 & -   \\ \hline
    \end{tabular}
\end{center}
\end{table}

Calcula-se a matriz de similaridade $\mathbf{S}$ pela Equação \ref{eq:sij} (Tabela \ref{tab:sij}) e recomendam-se os itens similares àqueles já comprados, segundo \ref{eq:ifw} (Tabela \ref{tab:i_u_fw}).

\begin{table}[p]
\begin{center}
    \caption{$s_{ij}$}
    \label{tab:sij}
    \begin{tabular}{ | c | c | c | c | c | c | c | } 
    \hline
     & $i_1$ & $i_2$ & $i_3$ & $i_4$ & $i_5$ & $i_6$  \\ \hline
     $i_1$ & - &  0.44 & 1.00 & 0.93 &  0.51 &  0.17 \\ \hline
     $i_2$ & 0.44 &         - & 0.51 & 0.44 &  1 & -0.32  \\ \hline
     $i_3$ & 1.00 &  0.51 &        - & 1.00 &  0.44 &  0.24 \\ \hline
     $i_4$ & 0.93 &  0.44 & 1.00 &        - &  0.51 &  0.17 \\ \hline
     $i_5$ & 0.51 &  1.00 & 0.44 & 0.51 &         - & -0.25 \\ \hline
     $i_6$ & 0.17 & -0.33 & 0.24 & 0.17 & -0.25 &         - \\ \hline
    \end{tabular}
\end{center}
\end{table}



\begin{equation}
\label{eq:ifw} 
    \hat{\imath}_u = \argmax_{i \in \left\{i~|~r_{ui} > 0\right\}, j}{s_{ij}}
\end{equation} 


\begin{table}[p]
\begin{center}
    \caption{$\hat{\imath}_u$ (FW)}
    \label{tab:i_u_fw}
    \begin{tabular}{ | c | c | c | c | } 
    \hline
     $u_1$ & $u_2$ & $u_3$ & $u_4$   \\ \hline
     3 & 5 & 3 & 4  \\ \hline
    \end{tabular}
\end{center}
\end{table}


\section{Algoritmo baseado no perfil de usuários (UP)} % (fold)
\label{sec:algoritmo_baseado_no_perfil_de_usu_rios_}

% section algoritmo_baseado_no_perfil_de_usu_rios_ (end)

O segundo algoritmo, adaptado da Referência \citeonline{debnath2008feature}, é um hibrido entre filtragem colaborativa e filtragem baseada em conteúdo. Os atributos dos itens são ponderados no cálculo de similaridade, com pesos extraídos de um modelo de perfil de usuários, denominado \textit{user profile} ou UP. Esse perfil leva em consideração o interesse dos usuários por \textit{features}, indiretamente calculado a partir de seu interesse pelos itens. 

Para se determinar a relevância de $f$ para $u$, deve-se levar em conta não somente a frequência com a qual uma característica aparece, mas também o fato de algumas características estarem contidas na maioria dos itens. Determina-se, então, os pesos $w_{uf}$, que mostram a relevância de $f$ para $u$, a partir da medida estatística TF-IDF (\textit{term frequency--inverse document frequency}), presente em formulações de recuperação de informação e mineração de dados (Equação \ref{eq:w-tfidf}). 

Em nosso caso, TF ou \textit{feature frequency} é a ``similaridade intra-usuários'', igual ao número de vezes em que a \textit{feature} $f$ aparece no perfil do usuário $u$ (Equação \ref{eq:tf}, Tabela \ref{tab:tf_uf}). Se o usuário avaliou \textit{positivamente} algum item $r_{ui}$, tal que $r_{ui}$ é superior a um valor mínimo $M$, considera-se que $u$ tem interesse $\mathrm{TF}_{uf}$ nos atributos $f$ dos itens $i$, representados por $a_{if}$. 

\begin{equation}
\label{eq:tf} 
    \mathrm{TF}_{uf}  = \sum_{i}{\mathrm{b}_M\left(r_{ui}~a_{if}\right)} 
\end{equation} 

\begin{table}[p]
\begin{center}
    \caption{$\mathrm{TF}_{uf}$}
    \label{tab:tf_uf}
    \begin{tabular}{ | c | c | c | c | c | } 
    \hline
     & $f_1$ & $f_2$ & $f_3$ & $f_4$   \\ \hline
     $u_1$ & 2 & 2 & 1 & 0  \\ \hline
     $u_2$ & 1 & 2 & 0 & 0  \\ \hline
     $u_3$ & 0 & 0 & 0 & 1  \\ \hline
     $u_4$ & 0 & 2 & 1 & 0  \\ \hline
    \end{tabular}
\end{center}
\end{table}

O termo IDF ou \textit{inverse user frequency} é a ``dissimilaridade inter-usuários'', relacionada com o inverso da frequência de um atributo $f$ dentro de todos os usuários (Equação \ref{eq:iuf}, Tabela \ref{tab:idf_f}).

\begin{equation}
\label{eq:iuf} 
    \mathrm{IDF}_{f} = \log \left( \frac{\left|~\mathcal{U}~\right|}{\sum_{u}{\mathrm{b}_0\left(\mathrm{TF}_{uf}\right)}} \right)
\end{equation} 

\begin{table}[p]
\begin{center}
    \caption{$\mathrm{IDF}_{f}$}
    \label{tab:idf_f}
    \begin{tabular}{ | c | c | c | c | } 
    \hline
     $f_1$ & $f_2$ & $f_3$ & $f_4$   \\ \hline
     0.30 & 0.12 & 0.30 & 0.60  \\ \hline
     \end{tabular}
\end{center}
\end{table}

Os pesos $w_{uf}$, obtidos na TF-IDF \ref{eq:w-tfidf} (Tabela \ref{tab:w_uf}), são utilizados para calcular a similaridade $s_{uv}$ entre dois usuários $u$ e $v$, conforme as Equações \ref{eq:suv} e \ref{eq:fuv} (Tabela \ref{tab:s_uv}).

\begin{equation}
\label{eq:w-tfidf} 
    w_{uf} = \mathrm{TF}_{uf}~\mathrm{IDF}_{f}
\end{equation} 

\begin{table}[p]
\begin{center}
    \caption{$w_{uf}$}
    \label{tab:w_uf}
    \begin{tabular}{ | c | c | c | c | c | } 
    \hline
     & $f_1$ & $f_2$ & $f_3$ & $f_4$   \\ \hline
     $u_1$ & 0.60 & 0.25 & 0.30 & 0  \\ \hline
     $u_2$ & 0.30 & 0.25 & 0 & 0  \\ \hline
     $u_3$ & 0 & 0 & 0 & 0.60  \\ \hline
     $u_4$ & 0 & 0.25 & 0.30 & 0  \\ \hline
    \end{tabular}
\end{center}
\end{table}

\begin{equation}
\label{eq:suv}
    s_{uv} = \frac{\sum\limits_{f \in \mathcal{F}_{uv}}{w_{uf}~w_{vf}}}{\sqrt{\sum\limits_{f \in \mathcal{F}_{uv}
    }w_{uf}^2} \sqrt{\sum\limits_{f \in \mathcal{F}_{uv}}w_{vf}^2}} 
\end{equation} 

\begin{table}[p]
\begin{center}
    \caption{$s_{uv}$}
    \label{tab:s_uv}
    \begin{tabular}{ | c | c | c | c | c | } 
    \hline
     & $u_1$ & $u_2$ & $u_3$ & $u_4$   \\ \hline
     $u_1$ & - & 0.96 & 0 & 1  \\ \hline
     $u_2$ & 0.96 & - & 0 & 1  \\ \hline
     $u_3$ & 0 & 0 & - & 0  \\ \hline
     $u_4$ & 1 & 1 & 0 & -  \\ \hline
    \end{tabular}
\end{center}
\end{table}

\begin{equation}
\label{eq:fuv}
\begin{split}
    \mathcal{F}_{uv} &= \mathcal{F}_u \cap \mathcal{F}_v \\
    \mathcal{F}_u &= \left\{ f \in \mathcal{F}~|~t_{uf} > 0 \right\}
\end{split}    
\end{equation} 

Dispondo-se de $\mathbf{S}$, selecionam-se os $k$ vizinhos mais próximos $v_k^u$ com maior similaridade $s_{uv}$, dentre todos $v \neq u$.  Posteriormente, determina-se o conjunto $\mathcal{I}_{v_k^u} = \left\{ i ~|~ r_{v_k^u i} > M\right\}$ de itens $i$ avaliados positivamente por $v_k^u$. Em \ref{eq:frf} avalia-se a frequência total $\mathrm{f}_{uf}$ dos atributos $f$ para os itens de $\mathcal{I}_{v_k^u}$ (Tabela \ref{tab:f_uf}). 

\begin{equation}
\label{eq:frf} 
\mathrm{f}_{uf} = \sum_{i \in \mathcal{I}_{v_k^u}}{\mathrm{b}_0\left(a_{if}\right)}
\end{equation} 

\begin{table}[p]
\begin{center}
    \caption{$\mathrm{f}_{uf}$}
    \label{tab:f_uf}
    \begin{tabular}{ | c | c | c | c | c | } 
    \hline
     & $f_1$ & $f_2$ & $f_3$ & $f_4$   \\ \hline
     $u_1$ & 0 & 2 & 1 & 0  \\ \hline
     $u_2$ & 1 & 3 & 2 & 0  \\ \hline
     $u_3$ & 1 & 2 & 0 & 0  \\ \hline
     $u_4$ & 2 & 3 & 1 & 0  \\ \hline
    \end{tabular}
\end{center}
\end{table}

Por fim, a partir da Equação \ref{eq:wi} calcula-se o peso $\omega_{ui}$ (Tabela \ref{tab:omega_ui}) de cada item e gera-se a lista dos \textit{top-N} produtos a serem recomendados para o usuário $u$, conforme \ref{eq:iup} (Tabela \ref{tab:i_u}). 

\begin{equation}
\label{eq:wi} 
    \omega_{ui} = \sum_{f}{a_{if}~\mathrm{f}_{uf}}
\end{equation} 


\begin{table}[p]
\begin{center}
    \caption{$\omega_{ui}$ (UP)}
    \label{tab:omega_ui}
    \begin{tabular}{ | c | c | c | c | c | c | c | } 
    \hline
     & $i_1$ & $i_2$ & $i_3$ & $i_4$ & $i_5$ & $i_6$ \\ \hline
     $u_1$ & 2 & 0 & 3 & 0 & 0 & 0 \\ \hline
     $u_2$ & 3 & 0 & 5 & 0 & 6 & 0 \\ \hline
     $u_3$ & 0 & 3 & 0 & 2 & 0 & 0 \\ \hline
     $u_4$ & 0 & 5 & 0 & 3 & 6 & 0 \\ \hline
    \end{tabular}
\end{center}
\end{table}

\begin{equation}
\label{eq:iup} 
    \hat{\imath}_u = \argmax_{i \in \left\{i~|~r_{ui}~=~0\right\}}{\omega_{ui}}
\end{equation} 

\begin{table}[p]
\begin{center}
    \caption{$\hat{\imath}_u$ (UP)}
    \label{tab:i_u}
    \begin{tabular}{ | c | c | c | c | } 
    \hline
     $u_1$ & $u_2$ & $u_3$ & $u_4$   \\ \hline
     3 & 5 & 2 & 5  \\ \hline
    \end{tabular}
\end{center}
\end{table}

\section{Algoritmo baseado na correlação usuário-item (UI)} % (fold)
\label{sec:algoritmo_baseado_na_correla_o_usu_rio_item_ui_}

Este método se trata de uma variante da solução UP, e também está embasado no cálculo da preferência do usuário por \textit{features}, medida através do seu interesse pelos itens. O algoritmo UI utiliza as matrizes de correlação ponderada entre usuários e atributos $\mathbf{W}$ e a matriz de atributos dos itens $\mathbf{A}$ no cálculo da correlação usuário-item.

A lista dos $N$ produtos a serem recomendados decorre portanto do cálculo de $\omega_{ui}$ (Equação \ref{eq:wui}, Tabela \ref{tab:omega_ui_ui}) e da escolha dos itens que maximizem essa variável para cada usuário (Equação \ref{eq:iup}, Tabela \ref{tab:i_u_ui}).

\begin{equation}
\label{eq:wui} 
    \omega_{ui} = \sum_{f}{w_{uf}~a_{if}}
\end{equation} 


\begin{table}[p]
\begin{center}
    \caption{$\omega_{ui}$ (UI)}
    \label{tab:omega_ui_ui}
    \begin{tabular}{ | c | c | c | c | c | c | c | } 
    \hline
     & $i_1$ & $i_2$ & $i_3$ & $i_4$ & $i_5$ & $i_6$ \\ \hline
     $u_1$ & 0.25 & 0.85 & 0.55 & 0.25 & 1.15 & 0 \\ \hline
     $u_2$ & 0.25 & 0.55 & 0.25 & 0.25 & 0.55 & 0 \\ \hline
     $u_3$ & 0    & 0    & 0    & 0    & 0    & 0.60 \\ \hline
     $u_4$ & 0.25 & 0.25 & 0.55 & 0.25 & 0.55 & 0 \\ \hline
    \end{tabular}
\end{center}
\end{table}


\begin{table}[p]
\begin{center}
    \caption{$\hat{\imath}_u$ (UI)}
    \label{tab:i_u_ui}
    \begin{tabular}{ | c | c | c | c | } 
    \hline
     $u_1$ & $u_2$ & $u_3$ & $u_4$   \\ \hline
     3 & 5 & - & 5  \\ \hline
    \end{tabular}
\end{center}
\end{table}

Ao passo que o método \textit{UP} recomenda itens a partir dos $k$ vizinhos mais próximos, o algoritmo \textit{UI} busca os itens com \textit{features} mais similares aos atributos pelos quais $u$ se interessa, diretamente através da matriz de atributos. 

Espera-se que esse tipo de recomendação forneça sugestões de qualidade similar ao algoritmo original, pois os dois tem a mesma fundamentação inicial. Pode-se observar que, para o exemplo-base, ambos algoritmos forneceram a mesma recomendação para três de quatro usuários.


% subsection variante_correla_o_usu_rio_item_ (end)

%\pagebreak
%\pagebreak
%!TEX root = index.tex
\chapter{Desenvolvimento da biblioteca} % (fold)
\label{cha:desenvolvimento_da_biblioteca}

\section{Recursos acadêmicos} % (fold)
\label{sec:recursos_acad_micos}

A principal contribuição da Escola Politécnica para o projeto veio das disciplinas de programação para automação (PMR2300 -- Computação para Automação e PMR2440 -- Programação para Automação) e de banco de dados (PMR2490 -- Sistemas de Informação). Além disso, a disciplina optativa PMR2728 -- Teoria de Probabilidades em Inteligência Artificial e Robótica abordou a temática do aprendizado de máquina e dos sistemas de recomendação.

Além das disciplinas do curso de Engenharia Mecatrônica, diversos recursos extra-curriculares foram de fundamental importância para o sucesso deste trabalho. Foram aplicados aprendizados práticos de quatro cursos da plataforma online Coursera (\url{https://www.coursera.org/}), sejam relacionados a teoria dos sistemas de recomendação, sejam relacionados a configuração de servidores na Amazon Web Services.

O curso ``Redes: Amigos, Dinheiro e Bytes'' (Networks: Friends, Money, and Bytes -- \url{https://www.coursera.org/course/friendsmoneybytes}), teve papel importante na introdução a temas ligados à rede mundial de computadores. Mais especificadamente, a aula 4 aborda, de maneira simples mas repleta de exemplos, a temática de sugestão de itens através da pergunta ``Como o Netflix recomenda filmes?''. Essa aula ajudou-nos a compreender a teoria por trás do algoritmo de recomendação do Netflix detalhado na Referência \citeonline{lops2011content-chap5}.


Outro curso que influenciou diretamente o nosso Trabalho de Conclusão de Curso foi ``Computação para Análise de Dados'' (Computing for Data Analysis -- \url{https://www.coursera.org/course/compdata}). As quatro semanas de aula ensinaram a leitura de dados formatados em R, o tratamento de dados, o uso de modelos estatísticos, como por exemplo métodos de regressão linear e polinomial, a aplicação de cálculos vetorizados e a construção de gráficos e tabelas. 

Aliado a essas aulas, aprendemos também o paradigma funcional, amplamente utilizado em R, durante as sete semanas de ``Princípios de Programação Funcional em Scala'' (Functional Programming Principles in Scala -- \url{https://www.coursera.org/course/progfun}). Todo o pacote foi construído em torno desse padrão de programação, visto que usuários de bibliotecas desejam utilizar métodos e funções genéricas para construção de resultados específicos.

Por fim, o curso de doze semanas de duração ``Engenharia de Startup'' (Startup Engineering -- \url{https://www.coursera.org/course/startup}) nos ensinou a trabalhar com diversas ferramentas de software necessárias para a realização dos testes de desempenho dos algoritmos. Utilizamos máquinas virtuais, linha de comando Unix, versionamento de código em \texttt{git} e editores de texto sem interface gráfica (tais como vi e Emacs). Além disso, o \textit{setup} de máquinas virtuais na Amazon Web Services também era abordada no curso, facilitando a configuração do ambiente de testes e a automatização desse processo. 

\section{Ferramentas utilizadas} % (fold)
\label{sec:ferramentas_utilizadas}

A programação da biblioteca computacional se deu por meio do ambiente de desenvolvimento integrado RStudio versão 0.98.953 (\url{http://www.rstudio.com/}). Esse IDE inclui console, editor de texto e corretor de sintaxe que suporta a execução de código direta, bem como ferramentas para traçar gráficos, histórico de comandos, depuração de erros e gerenciamento de espaço de trabalho. Além disso, o RStudio está disponível via licença de código aberto AGPLv3 (Affero General Public License version 3) para os principais sistemas operacionais (Windows, Mac e Linux).

\section{Métodos computacionais} % (fold)
\label{sec:m_todos_computacionais}



\subsection{Estrutura da biblioteca} % (fold)
\label{sub:estrutura_da_biblioteca}

A biblioteca está estruturada em quatro seções principais: \texttt{db}, onde está o banco de dados MovieLens 100k, \texttt{methods}, onde estão os algoritmos de recomendação, \texttt{results}, onde estão os métodos de avaliação de qualidade e \texttt{setup}, onde estão codificadas funções diversas, tais como leitura de banco de dados e cálculo de medidas de distância.

\begin{lstlisting}[caption=Estrutura da biblioteca]
recsys/
|-- db
|   `-- ml-100k
|       |-- u.data
|       |-- u.item
|       |-- u.user
|       |-- ...
|-- methods
|   |-- common
|   |   `-- up_ui_w.R
|   |-- fw.R
|   |-- ui.R
|   `-- up.R
|-- results
|   |-- benchmark.R
|   |-- performance.R
|   `-- run_tests.R
`-- setup
    |-- functions.R
    `-- setup.R
\end{lstlisting}

As principais funções da biblioteca estão descritas na documentação do Apêndice \ref{cha:documenta_o_da_biblioteca}, e o código pode ser obtido através do endereço \url{https://github.com/aviggiano/tcc/tree/master/recsys}.

% subsection estrutura_da_biblioteca (end)

\subsection{Algoritmo baseado na ponderação de atributos (FW)} % (fold)
\label{sub:algoritmo_baseado_na_pondera_o_de_atributos_fw_}

O algoritmo de ponderação de atributos possui cinco etapas: 

\subsubsection{Determinação de $e_{ij}$} % (fold)
\label{ssub:determina_o_de_e__ij_}

Em forma matricial, a Equação \ref{eq:determinacao-eij} pode ser descrita da forma \ref{eq:eij_matricial}.

\begin{equation}
\label{eq:eij_matricial}
\mathbf{E} = \mathrm{b}_M\left(\mathbf{R}^T\right) \mathrm{b}_M\big(\mathbf{R}\big)
\end{equation}

Em R, isso se traduz por uma simples instrução:

\begin{lstlisting}[caption=Determinação de $e_{ij}$]
e = b(t(r),M) %*% b(r,M)
\end{lstlisting}


\subsubsection{Determinação de $d_{ij}^f$} % (fold)
\label{ssub:determina_o_de_d__ij_f_}

Para o cálculo genérico de FW, a medida de distância padrão utilizada é a distância $L_1$ entre as duas \textit{features} (Equação \ref{eq:dijf2}, Código \ref{lst:dijf2}). Para o cálculo específico do banco de dados 100k-IMDB, as medidas de distância (Equação \ref{eq:dijf_especifico}, Código \ref{lst:dijf_especifico}) são determinadas segundo a Tabela TODO. 

\begin{equation}
\label{eq:dijf2}
d_{ij}^f = \left|a_{if} - a_{jf}\right|
\end{equation}

\begin{lstlisting}[caption=Determinação de $d_{ij}$ genérico,label=lst:dijf2]
d[i,j,] = abs(a[i,]-a[j,])
\end{lstlisting}

\begin{equation}
\label{eq:dijf_especifico}
\begin{split}
d_{ij}^f &= \left|a_{if} - a_{jf}\right|,~f \in \left\{1,21,23,24,25\right\} \\
d_{ij}^f &= J(a_{if},a_{jf}),~f = 2, \cdots, 20
\end{split}
\end{equation}


\begin{lstlisting}[caption=Determinação de $d_{ij}$ específico,label=lst:dijf_especifico]
d[i,j,1] = abs(a[i,21]-a[j,21])  
d[i,j,2] = abs(a[i,22]-a[j,22])  
d[i,j,3] = abs(a[i,23]-a[j,23])  
d[i,j,4] = abs(a[i,24]-a[j,24])  
d[i,j,5] = abs(a[i,25]-a[j,25])  
d[i,j,6] = abs(a[i,1]-a[j,1])    
d[i,j,7] = 1-jaccard(a[i,2:20],a[j,2:20])
\end{lstlisting}

\subsubsection{Determinação de $w_f$} % (fold)
\label{ssub:determina_o_de_w_f_}

O vetor $\mathbf{w}$ é obtido através da regressão linear de Equação \ref{eq:determinacao-wf}. Na linguagem R, basta aplicar o método \texttt{lm}, conforme mostra o Código \ref{lst:wf}.


\begin{lstlisting}[caption=Determinação de $w_f$,label=lst:wf]
# linear fit e ~ w0 + w (1-d)
D = 1-d
lm.W = lm(e ~ D, x=FALSE, y=FALSE, model=FALSE, qr=FALSE)
W = as.vector(lm.W$coefficients)
\end{lstlisting}

\subsubsection{Determinação de  $s_{ij}$} % (fold)
\label{ssub:determina_o_de_s__ij_}

Para o cálculo da matriz de similaridade, escolhemos os pesos $w_f>0$ e aplicamos a Equação \ref{eq:sij}, para todo $f>0$.

\begin{lstlisting}[caption=Determinação de $s_{ij}$]
if(W[f] > 0) s = s + W[f] * (1-d[,,f-1])
\end{lstlisting}


%\begin{enumerate}
	%\item Determinação de $e_{ij}$
	%\item Determinação de $d_{ij}^f$
	%\item Determinação de $w_f$
	%\item Determinação de  $s_{ij}$
	%\item Determinação de $\hat{i}_u$
%\end{enumerate}

\subsection{Algoritmo baseado no perfil de usuários (UP)} % (fold)
\label{sub:algoritmo_baseado_no_perfil_de_usu_rios_up_}
\subsection{Algoritmo baseado na correlação usuário-item (UI)} % (fold)
\label{sub:algoritmo_baseado_na_correla_o_usu_rio_item_ui_}

% subsection algoritmo_baseado_na_correla_o_usu_rio_item_ui_ (end)

% subsection algoritmo_baseado_no_perfil_de_usu_rios_up_ (end)

% subsection algoritmo_baseado_na_pondera_o_de_atributos_fw_ (end)

\section{Ambiente de testes} % (fold)
\label{sec:ambiente_de_testes}

Inicialmente, realizamos os testes de qualidade nos nossos próprios computadores pessoais. Todavia, a execução de testes sucessivos exigia muita capacidade computacional, principalmente quanto a memória virtual.

A alocação de objetos e matrizes na memória RAM é muito custosa, principalmente na etapa de determinação de medidas de distância $d_{ij}^f$ para o algoritmo FW (Equação \ref{eq:sij}). Uma matriz de dimensão $\left|\mathcal{I}\right| \times\left|\mathcal{I}\right| \times\left|\mathcal{F}\right|$ possui $1682 \times 1682 \times 25$ elementos (cerca de 71 milhões), e ocupa aproximadamente 500 MB de memória. No total, 4 GB de memória RAM são utilizadas durante todos os testes, fazendo-se necessário o uso máquinas dedicadas.

Por essa razão, realizamos todas as etapas de recomendação e avaliação de qualidade em máquinas \textit{memory-optimized} nos servidores da Amazon Web Services (\url{http://aws.amazon.com/}). Visto que o serviço é cobrado por hora-máquina, desenvolvemos um \textit{script} de inicialização para instalar todos os pacotes de programação e execução imediata dos testes, permitindo assim reduzir os custos da análise.   

As etapas de configuração do ambiente de testes envolvem o cadastro na Amazon Web Services, a criação de uma máquina virtual, a instalação das ferramentas de programação e o descarregamento do código de testes.

Após o cadastro na AWS, deve-se seguir os seguintes passos para a criação de uma máquina virtual:

\begin{enumerate}
\item Login na plataforma (Figura \ref{fig:aws_servicos});
\item Acesso ao serviço de máquinas virtuais Elastic Compute Cloud ou EC2. Para criar uma nova máquina, basta clicar em \textit{Launch Instance} (Figura \ref{fig:aws_ec2});
\item Escolha da configuração do software da máquina. É possível escolher entre diversos sistemas operacionais, versões e distribuições. Para avançar, deve-se clicar em \textit{Select}. No nosso caso, escolhemos a configuração \textit{Amazon Linux AMI 2014.09.1 (HVM)}, em virtude da facilidade de se instalar pacotes adicionais  (Figura \ref{fig:aws_setup_ec2});
\item Escolha do tipo de máquina. No nosso caso, escolhemos uma máquina otimizada para memória RAM. Em seguida, avançamos em \textit{Next: Configure Instance Details}. Nos \textit{Steps} 3, 4 e 5 do serviço, não modificamos nenhuma opção pré-configurada (Figura \ref{fig:aws_setup_ec2_maquina});
\item Criação da chave privada a ser utilizada para conexão com a máquina. Depois da criação de uma nova chave, pode-se descarregá-la através de \textit{Download Key Pair} e finalmente iniciar a máquina em \textit{Launch Instances} (Figura \ref{fig:aws_setup_keypair});
\item Inicialização da máquina e do DNS público. Esse é o endereço de conexão com a máquina, e pode ser obtido na seção \textit{Description} $\rightarrow$ \textit{Public DNS} (Figura \ref{fig:aws_setup_dns}).
\end{enumerate}

\begin{figure}[htp]
    \begin{center}
    \includegraphics[width=1\textwidth]{img/aws_servicos}
    \end{center}
    \caption{Serviços da Amazon Web Services}
    \label{fig:aws_servicos}
\end{figure}

\begin{figure}[htp]
    \begin{center}
    \includegraphics[width=1\textwidth]{img/aws_ec2}
    \end{center}
    \caption{Serviços de Elastic Compute Cloud (EC2)}
    \label{fig:aws_ec2}
\end{figure}

\begin{figure}[htp]
    \begin{center}
    \includegraphics[width=1\textwidth]{img/aws_setup_ec2}
    \end{center}
    \caption{Escolha do tipo de configuração do software da máquina}
    \label{fig:aws_setup_ec2}
\end{figure}

\begin{figure}[htp]
    \begin{center}
    \includegraphics[width=1\textwidth]{img/aws_setup_ec2_maquina}
    \end{center}
    \caption{Escolha do tipo máquina}
    \label{fig:aws_setup_ec2_maquina}
\end{figure}

\begin{figure}[htp]
    \begin{center}
    \includegraphics[width=1\textwidth]{img/aws_setup_keypair}
    \end{center}
    \caption{Criação da chave privada}
    \label{fig:aws_setup_keypair}
\end{figure}

\begin{figure}[htp]
    \begin{center}
    \includegraphics[width=1\textwidth]{img/aws_setup_dns}
    \end{center}
    \caption{Criação da máquina e do DNS público \textit{(Public DNS)}}
    \label{fig:aws_setup_dns}
\end{figure}



Uma vez criada a máquina, deve-se utilizar a chave privada para realizar um \textit{secure shell} e conectar-se remotamente ao EC2. A partir de um computador pessoal dotado de um interpretador de comandos \texttt{bash}, utilizamos a seguinte instrução:

\begin{lstlisting}[caption=Secure shell para conexão com a máquina virtual EC2]
ssh -i ~/Downloads/key_name.pem ec2-user@ec2-54-88-186-106.compute-1.amazonaws.com
\end{lstlisting}

Em seguida, para a automatização do ambiente de testes e rápida configuração de novas máquinas, criamos um arquivo \texttt{script.sh} na linguagem de programação \texttt{bash}. Esse \textit{script} instala os pacotes \texttt{R} e \texttt{git} e cria a chave pública necessária para acessar o servidor onde o código da biblioteca está hospedado. Em virtude de sua popularidade, utilizamos o serviço de hospedagem de códigos abertos GitHub (\url{https://github.com/aviggiano/tcc}). 

\begin{lstlisting}[caption=\textit{Script} de configuração do ambiente de testes]
#!/usr/bin/env bash
sudo su						# login como administrador. necessario para instalar pacotes no sistema operacional
yes | yum install R		# instala o pacote R no linux
yes | yum install git	# instala o git para descarregar os metodos de recomendacao
ssh-keygen -t rsa			# gera a chave para conectar-se ao GitHub
cat ~/.ssh/id_rsa.pub	# imprime a chave publica, que deve ser adicionada nas configuracoes do GitHub
\end{lstlisting}

A saída do \textit{script} é uma chave no seguinte formato:

\begin{lstlisting}[caption=Chave pública]
ssh-rsa AAAAB3NzaC1yc2EAAAADAQABAAABAQC/bxcszoyHzwyqtTXp4fl1Q3OT58Lsb7QLx+7nQ6
y0OIoWhK+r5ynSVi0BpTC+2hMrlg1rZTC1ED7Nb+SI9bRvf+1UYWOiVUXtwAVColMNBdIfE7QCWbJm
TmmBLcv9PIoCAvCfrxBh+flW3hXG388/LIEjZJckPYogho0jPAnFv3IXAGtVniV6cBcTTfKPUnX+np
6xiqnf4tYQpmPW/mnxk9s3bbEmcE1eYJkrE2IWdzy6EBnR9D4cBW5D8/VMM54xMJzugWZO//sIjLLT
0oFTTrroiwr+OX2DxqFdgCy8Agx1WZTeGhBAW1nvIVr5WVcWVBSzBCZfg8mYe+zYnbwl ec2-user@
ip-10-168-40-38
\end{lstlisting}

Após a obtenção da chave, deve-se cadastrá-la na página correspondente do GitHub (\url{https://github.com/settings/ssh}), como mostra a Figura \ref{fig:github_key}.

\begin{figure}[htp]
    \begin{center}
    \includegraphics[width=1\textwidth]{img/github_key}
    \end{center}
    \caption{Cadastro da chave pública no GitHub}
    \label{fig:github_key}
\end{figure}

Uma vez tendo habilitado a máquina virtual da AWS para manipulação do repositório da biblioteca, pode-se descarregar o código e executar o \textit{script} de testes de qualidade:

\begin{lstlisting}[caption=Script de execução dos testes de qualidade]
#!/usr/bin/env bash
git clone git@github.com:aviggiano/tcc	# clona o repositorio
cd tcc && Rscript recsys/results/run_tests.R 		# executa o script de testes
\end{lstlisting}
%\pagebreak
%!TEX root = index.tex
\chapter[Resultados]{Resultados}
\label{chap:resultados}

Até o presente momento, os resultados deste Trabalho de Conclusão de Curso concentram-se na definição de necessidades, de parâmetros de sucesso e de síntese de possíveis soluções. 

Visto que a primeira etapa de um sistema de recomendação é a coleta e manipulação de dados, definimos que a aquisição de dados será feita a partir de um banco de dados genérico, que deverá alimentar o sistema por meio de arquivos de texto com valores separados por vírgulas (\texttt{.csv}). A fim de facilitar o pré-processamento dos dados, exigem-se três arquivos, cada um com uma tabela de itens e seus atributos $\mathbf{A}$, clientes e suas características $\mathbf{B}$ e histórico de compras ou avaliações $\mathbf{R}$. Caso existam outras tabelas no banco de dados, o sistema deverá ser alterado para levar em conta o processamento dos arquivos suplementares.

\begin{equation} 
\begin{split} 
\mathbf{A} &= 
\begin{bmatrix} 
 a_{i_1 f_1} &  a_{i_1 f_2} &  a_{i_1 f_3}  & \dots   \\
 a_{i_2 f_1} &  a_{i_2 f_2} &  a_{i_2 f_3}  & \dots   \\
 a_{i_3 f_1} &  a_{i_3 f_2} &  a_{i_3 f_3}  & \dots  \\ 
 \vdots &  \vdots &  \vdots  & \ddots   \\
 \end{bmatrix} \\
\mathbf{B} &= 
\begin{bmatrix} 
 b_{u_1 c_1} &  b_{u_1 c_2} &  b_{u_1 c_3}  & \dots   \\
 b_{u_2 c_1} &  b_{u_2 c_2} &  b_{u_2 c_3}  & \dots   \\
 b_{u_3 c_1} &  b_{u_3 c_2} &  b_{u_3 c_3}  & \dots  \\ 
 \vdots &  \vdots &  \vdots  & \ddots   \\
 \end{bmatrix}\\
  \mathbf{R} &= 
\begin{bmatrix} 
  r_{u_1 i_1} &  r_{u_1 i_2} &  r_{u_1 i_3}  & \dots   \\
 r_{u_2 i_1} &  r_{u_2 i_2} &  r_{u_2 i_3}  & \dots   \\
 r_{u_3 i_1} &  r_{u_3 i_2} &  r_{u_3 i_3}  & \dots  \\ 
 \vdots &  \vdots &  \vdots  & \ddots   \\
\end{bmatrix}
\end{split} 
\end{equation}

Em alguns bancos de dados, a tabela de histórico também contém outras informações adicionais $\theta$, tais como o método de pagamento, a data da compra, data de entrega, etc., e é denominada $\mathbf{H}$.

\begin{equation} 
\mathbf{H} =
\begin{bmatrix} 
 r_{u_1 i_1} &  \theta_{h_1 1} &  \theta_{h_1 2} & \dots   \\
 r_{u_1 i_2} &  \theta_{h_2 1} &  \theta_{h_2 2} & \dots   \\
 r_{u_1 i_3} &  \theta_{h_3 1} &  \theta_{h_3 2} & \dots   \\
 \vdots &  \vdots &  \vdots  & \ddots   \\
 \end{bmatrix} \\
\end{equation}


Uma vez determinada a forma de entrada de dados, definiu-se a escolha do conjunto de dados a serem utilizados. O primeiro conjunto de dados abertos é proveniente do website de recomendações de filmes MovieLens (\url{http://movielens.umn.edu}). Nessa base de dados, o catálogo de filme faz o papel de catálogo de produtos pelos quais os usuários possam se interessar, e o histórico de compras se refere à avaliação dos filmes feita por cada usuário. Outros conjuntos de dados também serão  explorados pela dupla, tais como os dados de classificação de músicas do serviço Yahoo! Music (\url{http://webscope.sandbox.yahoo.com}) ou de dados anônimos de e-commerces.

Por fim, as possíveis soluções do projeto abrangem o cálculo das medidas de similaridade entre itens, para os conjuntos de dados que não foram previamente tratados. Determinar um valor numérico entre dois produtos distintos, tais como uma camiseta e uma prancha de \textit{surf}, é uma tarefa complexa e sujeita a erros humanos. Após reflexão e leitura de referências, definimos possíveis maneiras de realizar esse cálculo: 

\begin{itemize} 
	\item Grupos de similaridade: a similaridade dos itens seria definida por pelas características do item (como ``esporte radical'', ``corrida'', ``filme de aventura'', etc). Seria necessário classificar manualmente esses atributos (por exemplo, determinar que ``esporte radical'' tem similaridade de $3/5$ com ``corrida'' e $1/5$ com ``produtos de limpeza'').
	\item Histórico de compra da comunidade: quanto mais usuários comprarem a mesma dupla de itens, maior a similaridade entre estes itens. A classificação se faz pelo histórico mas o índice de similaridade pertence aos itens e não entre os usuários.
	\item Ranking por arestas: adaptado do sistema \textit{edge rank}, de classificação de posts no Facebook, este sistema levaria em conta a multiplicação de diversas características do item. Características como: popularidade da marca, número de compras por visualização do item, popularidade da marca para o usuário em questão (quantos itens ele comprou desta marca), há quanto tempo o item foi lançado e a relação do item com as compras anteriores do usuário (se este tipo de item já fez sucesso com este usuário).
\end{itemize}

Para os itens que já foram avaliados, como no caso dos conjuntos de dados de classificação de filmes ou músicas, essa etapa não é necessária, pois a similaridade entre dois itens provém diretamente da avaliação do usuário. Dizer que um filme $A$ tem avaliação $4/5$ e um filme $B$ tem avaliação $5/5$ equivale a dizer que os dois são interessantes para aquele usuários, e por isso vale recomendar filmes similares a $A$ e $B$. Nesse caso, passa-se  diretamente para a etapa das recomendações.

Os resultados práticos de cálculo de similaridade ou descrição completa do banco de dados serão apresentados no relatório final do trabalho, em conjunto com os demais resultados da evolução do projeto. 
%\pagebreak
%!TEX root = index.tex
\chapter{Conclusão}
%\chapter{Trabalhos Futuros}
\label{cha:trabalhos_futuros}

A avaliação de desempenho dos métodos propostos na biblioteca deste Trabalho de Conclusão de Curso verificaram resultados já conhecidos no meio acadêmico.

Em particular, a dependência entre qualidade de recomendação e tamanho da lista de sugestões se verificou (impacto de $N$). Além disso, mostramos que um banco de dados com maior quantidade de avaliações (impacto de $H$) tem mais relevância que um banco de dados com mais usuários (impacto de $T$). 

Outro resultado do trabalho foi a comprovação do fenômeno de \textit{hidden feedback} (impacto de $M$). Mesmo que construamos métodos embasados na ``avaliação positiva'' dos usuários, esse parâmetro pode não ter tanta influência, visto que a maioria das avaliações dos clientes já são de fato positivas.

Para os trabalhos futuros, iremos realizar a validação cruzada e avaliar se os requisitos funcionais foram estabelecidos. Em seguida, procuraremos melhorar o sistema de recomendação a fim de torná-lo mais genérico. Buscaremos eliminar restrições quanto a entrada e saída de dados, de forma que elas sejam completamente arbitrárias. O objetivo é que o usuário possa informar ao sistema como é formado sua base, e que todo o tratamento preliminar seja feito automaticamente. 

Caso haja tempo, trabalharemos também na construção de um \textit{driver} que possibilite a conexão entre o sistema de recomendação e um banco de dados SQL, sem que seja necessária a etapa intermediária de arquivos \texttt{csv} para aquisição de dados. Planejamos elaborar um \textit{website} para o sistema de recomendação e exportar toda a lógica para um servidor dedicado. Outra melhoria desejada é a reconstrução dos métodos na linguagem de programação C, a fim de melhorar a performance computacional. Dessa forma, o serviço de ``sistema de recomendação nas nuvens'' estaria completo e poderia ser utilizado por e-commerces reais.
%%!TEX root = index.tex
\section[Cronograma]{Cronograma}
\begin{frame}{Cronograma}
\begin{itemize}
	\item[09/07] Pré-tratamento do banco de dados
 	\item[16/07] \textbf{Programação} do método Ponderação de Atributos
 	\item[23/07] \textbf{Programação} do método Perfil de Usuários
 	\item[30/07] Análise comparativa dos dois algoritmos
 	\item[]
 	\item[13/08] Relatório de atividades de implementação
 	\item[27/08] Primeiros testes com o sistema \\(\textbf{precisão e acurácia} para uma base de testes)
 	\item[]
 	\item[03/09] Testes com o sistema (\textbf{validação cruzada})
 	\item[24/09] Melhorias incrementais e relatório de atividades
 	\item[]
 	\item[15/10] \textbf{Relatório aprofundado} de atividades
 	\item[]
 	\item[05/11] Elaboração da apresentação e finalização dos relatórios
 	\item[12/11] Melhorias incrementais
\end{itemize}
\end{frame}
%\pagebreak
%%!TEX root = index.tex
\chapter[Andamento do Projeto]{Andamento do Projeto}
\label{chap:andamento_do_projeto}

%Tal relatório deverá conter 
%	descrição minuciosa 
%		do andamento do projeto
%		das etapas cumpridas, 
%		do planejamento do trabalho no segundo semestre, 
%		dos resultados alcançados  , 
%		das modificações em relação à proposta inicial, 
%	bem como uma avaliação precisa e detalhada 
%		dos resultados alcançados neste semestre , 
%		dos objetivos do próximo semestre 
%		da viabilidade de alcance do resultado programado  no fim do segundo semestre. 

Ao longo do semestre, o escopo deste Trabalho de Conclusão de Curso se alterou no que tange as possíveis soluções e aquilo que será entregue como produto.

De início, pensamos fazer um sistema de recomendação utilizando algoritmos de filtragem colaborativa baseada em itens, principalmente motivados pela leitura inicial de \cite{linden2003amazon}, que mostra as vantagens desse método comparado à filtragem colaborativa baseada em usuários. 

Todavia, percebemos que grande parte dos e-commerces estruturam seus bancos de dados em torno da descrição dos itens à venda e das informações dos clientes. As tabelas de itens podem possuir dezenas de atributos, dependendo do ramo de negócios da loja, tais como marca, esporte, categoria, cor, preço ou outros. Pouco detalhe é dado à interação entre esses dois grupos, visto que a tabela de histórico de compras se limita a informações como data e método de pagamento. Dessa forma supusemos que métodos de filtragem colaborativa, fundamentados na avaliação dos itens por parte dos usuários, teriam pior desempenho que métodos baseados em conteúdo, que exploram as características dos itens na recomendação. 

Ao considerarmos os possíveis algoritmos de recomendação, decidimos  não apenas testar um método de sugestão de produtos, mas sim fazer uma análise comparativa entre dois ou mais estratégias de recomendação. 

Os métodos do sistema de recomendação a ser desenvolvido serão, portanto, híbridos baseados em variantes de dois diferentes algoritmos, descritos nas Seções \ref{sec:algoritmo_baseado_na_pondera_o_de_atributos_} e \ref{sec:algoritmo_baseado_no_perfil_de_usu_rios_}, inspirados em \cite{symeonidis2007feature} e \cite{debnath2008feature}.

O primeiro artigo determina a similaridade de dois itens a partir de medidas de distância para cada um dos atributos dos itens, ponderadas por pesos determinados na regressão linear de uma equação descrita pelo interesse dos usuários em cada \textit{feature}. O segundo texto parte do princípio que os usuários estão interessados nos atributos dos itens, traçando correlações entre esses dois elementos até chegar nos pesos que servirão de base para o cálculo da similaridade inter-usuários, utilizada na recomendação pelo método da vizinhança (\textit{nearest neighbors}). Ambos estão descritos com maior detalhe no Capítulo \ref{chap:sintese_de_solucoes}.
%\pagebreak
\appendix
%!TEX root = index.tex
\chapter{Documentação da biblioteca} % (fold)
\label{cha:documenta_o_da_biblioteca}

% chapter documenta_o_da_biblioteca (end)
\lstset{language=C}
\begin{lstlisting}[caption=\texttt{setup}.R]
read.history 
	input				filename, separator, header, col.names
	output			history
	description		Lê o arquivo de histórico de compras e retorna uma matriz correspondente. Os parametros separator e header indicam a formatação do arquivo e são opcionais. Caso header seja FALSE, col.names deve conter um arranjo de palavras indicando os nomes das colunas da matriz.

read.item
	input				filename, separator, header, col.names
	output			item
	description		Lê o arquivo de descrição dos itens e retorna uma matriz correspondente. Os parametros separator e header indicam a formatação do arquivo e são opcionais. Caso header seja FALSE, col.names deve conter um arranjo de palavras indicando os nomes das colunas da matriz.

read.user
	input				filename, separator, header, col.names
	output			user
	description		Lê o arquivo de descrição dos usuários e retorna uma matriz correspondente. Os parametros separator e header indicam a formatação do arquivo e são opcionais. Caso header seja FALSE, col.names deve conter um arranjo de palavras indicando os nomes das colunas da matriz.

get.r
	input				history
	output			r
	description		Le a matriz de histórico de compras e retorna a matriz de avaliação r_ui

get.a
	input				item
	output			a
	description		Le a matriz de descrição de itens e retorna a matriz de atributos dos itens a_if
\end{lstlisting}



\begin{lstlisting}[caption=\texttt{performance}.R]
hide.data
	input				r, Utrain.Utest, HIDDEN, random=FALSE, has.na=TRUE 
	output			matriz r com dados mascarados
	description		mascara os dados da matriz r para os usuários de teste da lista Utrain.Utest. HIDDEN é o percentual de avaliações a serem mascaradas. random é o modo de operação; caso seja TRUE, os dados são mascarados aleatoriamente para os usuários-teste. has.na indica se os itens não avaliados em r são indicados como NA ou como 0.

divide.train.test
	input				r, TRAIN
	output			lista contendo U.train e U.test
	description		divide os usuários em duas bases, uma de treinamento de uma de testes, segundo a proporção TRAIN

performance
	input				a, r, M=2, k=10, N=20, norm=TRUE, remove=FALSE, method, TRAIN=0.75, HIDDEN=0.75, W=FALSE, repick=FALSE
	output			lista contendo precisão, abrangência, medida F1 e tempo de execução do método
	description		norm indica se a matriz de avaliações deve ser normalizada. W, caso diferente de FALSE, indica a quantidade de pesos de atributos a serem utilizados no método FW
\end{lstlisting}

\begin{lstlisting}[caption=\texttt{fw}.R]
fw
	input				a, r, rtrain.rtest, Utrain.Utest, M, k, N, W
	output			iu
	description		Retorna uma lista de recomendação para todos os usuários Utest, após obter o modelo a partir dos usuários Utrain
\end{lstlisting}

\begin{lstlisting}[caption=\texttt{ui}.R]
ui
	input				a, r, rtrain.rtest, Utrain.Utest, M, k, N
	output			iu
	description		Retorna uma lista de recomendação para todos os usuários Utest, após obter o modelo a partir dos usuários Utrain
\end{lstlisting}

\begin{lstlisting}[caption=\texttt{up}.R]
up
	input				a, r, rtrain.rtest, Utrain.Utest, M, k, N
	output			iu
	description		Retorna uma lista de recomendação para todos os usuários Utest, após obter o modelo a partir dos usuários Utrain
\end{lstlisting}



\begin{lstlisting}[caption=\texttt{functions}.R]
b
	input				x, y
	output			1 se x > y ou 0 se x <= y
	description		A definição da função b_M é diferente da do artigo de referência, em que b_Pt(x) é  tal que x >= Pt

delta
	input				m, n
	output			1 se m == n ou 0 se m != n
	description		Delta de Kronecker

jaccard
	input				xs, ys
	output			número entre 0 e 1
	description		Índice Jaccard entre os conjuntos xs e ys	

h
	input				matrix, N=6
	output			imprime as primeiras N linhas e N colunas da matriz
	description		Caso a entrada seja um vetor, imprime os N primeiros elementos. 

top.N
	input				xs, N=10
	output			N maiores elementos da lista xs
	description		Usado na construção da lista de itens top-N 

index.top.N
	input				xs, N=10, ys.remove=NULL
	output			Índice dos N maiores elementos da lista xs
	description		ys.remove é uma lista em que se deseja excluir os elementos da lista top.N

normalize
	input				matrix, columns=FALSE
	output			matriz normalizada
	description		Caso columns seja TRUE, as colunas da são normalizadas dependendo do maior valor absoluto de cada uma delas
\end{lstlisting}
%\pagebreak
%\input{sumario-final}
% ----------------------------------------------------------
% ELEMENTOS PÓS-TEXTUAIS
% ----------------------------------------------------------
\postextual
% ----------------------------------------------------------

% ----------------------------------------------------------
% Referências bibliográficas
% ----------------------------------------------------------
\bibliography{bibliografia}
\end{document}
