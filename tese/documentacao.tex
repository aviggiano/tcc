%!TEX root = index.tex
\chapter{Documentação da biblioteca} % (fold)
\label{cha:documenta_o_da_biblioteca}

% chapter documenta_o_da_biblioteca (end)
\lstset{language=C}
\begin{lstlisting}[caption=\texttt{setup}.R]
read.history 
	input				filename, separator, header, col.names
	output			history
	description		Lê o arquivo de histórico de compras e retorna uma matriz correspondente. Os parametros separator e header indicam a formatação do arquivo e são opcionais. Caso header seja FALSE, col.names deve conter um arranjo de palavras indicando os nomes das colunas da matriz.

read.item
	input				filename, separator, header, col.names
	output			item
	description		Lê o arquivo de descrição dos itens e retorna uma matriz correspondente. Os parametros separator e header indicam a formatação do arquivo e são opcionais. Caso header seja FALSE, col.names deve conter um arranjo de palavras indicando os nomes das colunas da matriz.

read.user
	input				filename, separator, header, col.names
	output			user
	description		Lê o arquivo de descrição dos usuários e retorna uma matriz correspondente. Os parametros separator e header indicam a formatação do arquivo e são opcionais. Caso header seja FALSE, col.names deve conter um arranjo de palavras indicando os nomes das colunas da matriz.

get.r
	input				history
	output			r
	description		Le a matriz de histórico de compras e retorna a matriz de avaliação r_ui

get.a
	input				item
	output			a
	description		Le a matriz de descrição de itens e retorna a matriz de atributos dos itens a_if
\end{lstlisting}



\begin{lstlisting}[caption=\texttt{performance}.R]
hide.data
	input				r, Utrain.Utest, HIDDEN, random=FALSE, has.na=TRUE 
	output			matriz r com dados mascarados
	description		mascara os dados da matriz r para os usuários de teste da lista Utrain.Utest. HIDDEN é o percentual de avaliações a serem mascaradas. random é o modo de operação; caso seja TRUE, os dados são mascarados aleatoriamente para os usuários-teste. has.na indica se os itens não avaliados em r são indicados como NA ou como 0.

divide.train.test
	input				r, TRAIN
	output			lista contendo U.train e U.test
	description		divide os usuários em duas bases, uma de treinamento de uma de testes, segundo a proporção TRAIN

performance
	input				a, r, M=2, k=10, N=20, norm=TRUE, remove=FALSE, method, TRAIN=0.75, HIDDEN=0.75, W=FALSE, repick=FALSE
	output			lista contendo precisão, abrangência, medida F1 e tempo de execução do método
	description		norm indica se a matriz de avaliações deve ser normalizada. W, caso diferente de FALSE, indica a quantidade de pesos de atributos a serem utilizados no método FW
\end{lstlisting}

\begin{lstlisting}[caption=\texttt{fw}.R]
fw
	input				a, r, rtrain.rtest, Utrain.Utest, M, k, N, W
	output			iu
	description		Retorna uma lista de recomendação para todos os usuários Utest, após obter o modelo a partir dos usuários Utrain
\end{lstlisting}

\begin{lstlisting}[caption=\texttt{ui}.R]
ui
	input				a, r, rtrain.rtest, Utrain.Utest, M, k, N
	output			iu
	description		Retorna uma lista de recomendação para todos os usuários Utest, após obter o modelo a partir dos usuários Utrain
\end{lstlisting}

\begin{lstlisting}[caption=\texttt{up}.R]
up
	input				a, r, rtrain.rtest, Utrain.Utest, M, k, N
	output			iu
	description		Retorna uma lista de recomendação para todos os usuários Utest, após obter o modelo a partir dos usuários Utrain
\end{lstlisting}



\begin{lstlisting}[caption=\texttt{functions}.R]
b
	input				x, y
	output			1 se x > y ou 0 se x <= y
	description		A definição da função b_M é diferente da do artigo de referência, em que b_Pt(x) é  tal que x >= Pt

delta
	input				m, n
	output			1 se m == n ou 0 se m != n
	description		Delta de Kronecker

jaccard
	input				xs, ys
	output			número entre 0 e 1
	description		Índice Jaccard entre os conjuntos xs e ys	

h
	input				matrix, N=6
	output			imprime as primeiras N linhas e N colunas da matriz
	description		Caso a entrada seja um vetor, imprime os N primeiros elementos. 

top.N
	input				xs, N=10
	output			N maiores elementos da lista xs
	description		Usado na construção da lista de itens top-N 

index.top.N
	input				xs, N=10, ys.remove=NULL
	output			Índice dos N maiores elementos da lista xs
	description		ys.remove é uma lista em que se deseja excluir os elementos da lista top.N

normalize
	input				matrix, columns=FALSE
	output			matriz normalizada
	description		Caso columns seja TRUE, as colunas da são normalizadas dependendo do maior valor absoluto de cada uma delas
\end{lstlisting}